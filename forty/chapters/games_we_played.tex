\chapter{Games We Played}
Children have a built-in desire and zest to play and have fun, provided
they are placed in a safe and loving environment. They laugh and giggle
and entertain themselves with a variety of games depending on where and
when they grow up. 

When I see our grand children play games, blissfully not on the tablets or
phones but mostly outdoors physical games and interactive or educational
indoor games, faint memories start waking up. When I hear their non-stop
laughter and noise, I reminisce about the games we used to play in Panipat
in the 1950s. They are so vivid as if I can reach out and touch the
participants. 

We have simple yet engaging team games, that fill the long days growing up
in Panipat. 

Immediately after the partition, a life changing major event as we
understand it now, but completely oblivious at that time. Our parents
never mentioned to us the catastrophic event and its effects on their
lives. They just wanted for us to have carefree childhood. They were busy
in their innumerable chores. Their reassuring presence engulfed us as we
went around, free of care, planning and playing out our days of childhood. 

There were no computers or television, let alone the hand held devices. 

Most of the times we play outdoors with friends and the brothers and
occasionally indoor interactive games with family members. Our brother,
Prem who lives in Madras, brings suitcases full of fun and educational
toys. He has also taught us, never-heard-of, card games. The whole family
eagerly waits for his visits. His lively personality, love, laughter and
games paint our lives with bright colors. 

Even though we are poor, we never feel deprived. Childhood for us is
a never ending fun. Even after several decades I can see images as if I am
still the same child. 

We have competition by deciding who will make the longest jump from the
swing set strategically placed outside the Puja/store rooms. The landing
is on a grassy soft patch of land under the shade of most delicious
Dussehri mango tree lovingly planted by Pita Ji in 1950 near the main
entrance of the house. 

The game of Lakeeran (Lines) is loved by all.  Gleefully, children sing
"Cheecho cheech galerian, do teriaan do meriaan.” Hearing this all
children get together. 

Two teams are formed, each assigned roughly an equal area around the
house. Both teams draw small lines with a pencil, a chalk or a piece of
soft charcoal on pieces of papers, stones, bricks, or wood in a fixed
time. These objects are carefully hidden in hard to detect places; under
the leaves, bushes, stones, bricks or whatever else we find as
a camouflage. Even though prohibited, we draw lines in poorly visible
parts of the pale concrete walls of the house or the perimeter wall of the
house.  A mental note is made about the hiding locations. At the end of
the set time the teams go into the opponent's territory, hurriedly search
and uncover the objects bearing the lines. At the end of a fixed search
time, the children on both sides pull out the undetected objects marked
with lines. The team with maximum number of undetected lines is the
winner. The lines are washed or erased preparing the area for future
games. 

On Sundays during school days and 3 to 4 times a week during the three
months of summer vacation, a group of our friends go for a four miles walk
in the open fields dotted with keekar, neem and peeple tress. The goal is
to reach the Nehr (Canal) located southwest of our home. Dressed in
knickers, shirt and slippers the trek starts before sunrise to take
advantage of the cool breeze before the ground started baking. We break
twigs of neem or keekar tree, pluck the leaves, chew the bark on one end,
let the juice rinse the mouth while exposing the bristles in the shape of
a micro broom to be used as a tooth brush. Bitter initially, the taste
slowly became mild, the twig was our tooth brush. Shamelessly spitting
whenever and wherever needed, we talk, sing and walk leisurely to the
Nehr. The canal is too big and deep for our comfort.

Use of our primitive tooth brush never bothers us even when we see some
well-to-do children use tooth paste and proper brushes. Neem tooth pastes
start showing up in the market. Now we understand the possible reason of
never having the need to see a dentist in our childhood. 

A naala (creek) feeds the Nehr. The good swimmers jumped into the river,
the lesser ones bathe in the creek and non swimmers enviously looked on.
Sometimes urged by the will to be part of the group and sometimes
mischievously pushed into the naala, the non swimmers go through some
'gothe'  (getting little bit of water in the windpipe or the lungs) and
struggle out of the water gasping for air. Only once we heard of a boy
getting washed away by the currents and occasional whirlpool that used to
be fun to watch but we knew it could be deadly. 

We are lucky that there is a play ground near the entrance gate. There is
no development behind our house, just flat open field. There is ample
ground to play all known outdoor games. 

Piththu is a very popular team game in northern India. We collect 8 to 10
pieces of rocks, some flat and some irregular in shape. A flat one is
placed on the ground and other irregular ones carefully stacked as
a column. It is intentionally made easy to crumble and difficult to
reconstruct. Theekree is a variation where all the rocks are flat and easy
to re-stack.

Two teams are formed. One player at a time from the offense team holds
a rubber ball in hand, stands about 10 feet away from the rock column.
One player from the defense team stands behind the column to catch the
ball and the rest of the members from both sides spread out in the field.
The object is for the pitcher to throw the ball and hit the column making
the rocks scatter. He gets three attempts to hit the column and if he
fails, other players in the team take turns. If all fail, the defense team
becomes the offense team. When the column gets hit and crumble, the
offense team players run up to the site, collect all the rocks and
recreate the column. The defense team tries to hit the opponents with the
ball. The offense team players scream, run, twist and turn, jump to avoid
getting hit while at the same time continue the attempt to recreate the
column. If the ball touches any player he sits out decreasing the number
of column makers, making the hard job even harder. Eventually either the
team successfully recreates the column or all their players are out after
getting hit by the ball. 

At the end of one game the teams reverse the role. The game goes on and on
till Mata Ji (Mother, with Ji added as a sign of respect) call us in,
announcing the dinner. 

Hungry, exercised bodies need the nutrition. Without making any fuss we
run in, wash hands, say a quick mandatory prayer, read Ramayan, sit down
on the floor outside the kitchen and eat freshly cooked vegetables and
fluffed fresh roti (round flat bread made of whole wheat flour.) made on
the hot tava (Skillet). There is always a friendly argument about who gets
the first roti. Having quickly devoured the food, washed hands and rinsed
mouths, the boys are eager to go out and ready for more games. 

It was only in later years we found out that Mata Ji had served all the
vegetables to us and ate her own roti with salt, pickle and onions!

After the sun goes down, we play Lookan Mitti (hide and seek). It is not
the kind we play with our grand children these days where a few rooms and
closets are chosen to silently hide and excitedly wait as the seekers are
heard searching nearby. 

Our hiding places are a set of about 30 neighborhood houses. No one is
allowed to go inside their homes. We can hide on their rooftops, in their
yards, gardens, tree tops or any outdoor locations. 

One time we got into trouble as we went into an open outhouse which was
being used by an older woman. Oops! 

Every grown up person looks old to our young eyes. We are barred forever
from going to that house again. 

After 15 minutes or so the seekers go out hunting for the hiders. Once
a while the hiders sends signals by whistling or shrieking. Like all
games, this also comes to an end when the message comes that it is time to
come back and sleep. 

A glass of sweet milk, sometimes with bread pieces soaked in it works as
a night cap. After a restful sleep the body is ready for school and then
more games. 

One very popular and apparently unique game of India is Gulli Danda. Gulli
is constructed from about 4 inches piece of round wood. The two ends are
chiseled like the ends of a sharpened pencil. The danda is about twenty
inches long round tree branch which has been made smooth by sanding. 

Gulli Danda is played in two ways. The common way is to have two teams of
5 to 6 boys on each side. A slit like burrow, khutti, is dug in the
ground, the gulli is placed across it and the danda touching the ground is
kept  behind it. One player at a time from the offense team takes turn
till they were all out. He pushes the gulli in the air. The defense team
lines up a short distance in front of this starting point and tries to
catch the gulli as it soars in the air. If it is caught, the boy is out
and next players comes on. If if is not caught, the starting player keeps
the danda across the slit. Offense team throws the gulli back, the goal is
to hit the danda.  If it touches the danda, the player is out and if not,
the player strikes one end of the gulli with the danda making it rise in
the air. As it rises up in the air he hits it away with the danda, the
farther the better. If he cannot not hit the gulli he is out. If it gets
the hit and gulli gets caught, as it flies in the air, by the now spread
out defense team, the player is out. If not, the defense player throws it
back with an attempt to hit the danda placed across the khutti. If it
fails the player hits the gulli from where it lands. The game goes on till
all the players had a chance to hit the gulli. Number of hits decide which
team wins. 

Once all the players on the offense team out the teams reverse the role. 

The other way these two tools are used is called Daayra (Circle). A large
circle is drawn on the ground with chalk powder. One player from one team
stands inside the circle with a danda in his hand. The opponent team
players throws gulli in the air over the Daayra. The player inside the
circle hits the gulli as it descends with the danda. If missed and it
lands in the circle or if the gulli gets hit but caught he is declared
out. Victory is decided by the number of hits. We never thought of eye
injuries. 

Riding bicycle with groups of friends is another fun exercise. We cut out
a medium hard card board piece and stick it along the spokes of wheels in
such a way that it creates an intermittent sound when the moving cycle
makes the card board strike the spokes. Faster ride makes the beat faster
to a level when you hear a continuos sound. Bike racing on the road is
safe as there are no cars or scooters in our neighborhood. There is no
concept of helmets or knee guards. 

Rerha (rim of bicycle wheel) is another physical game children play.
A metal rod is bent at one end to create a U shape that hold the rim of
wheel, not too tight and not too loose. Having thus engaged the wheel in
this contraption we run with it on the road. At times we have races
running with the rehrha. 

Sound of tho tho tho or hey kabbadi kabbadi kabaddi without taking the
next breath can still be heard on command in my mind. Kabaddi, India's
national game, is played by having two teams of about 5 to 10 players on
each side. A dividing straight line is drawn between the teams with chalk
powder and a square boundary line drawn around both sides to demarcate the
playing field. Clothes except underwear are taken off, body is soaked with
sarson ka tel (mustard oil) to make it slippery and difficult to grab. One
player from one side goes into the opponent's territory saying tho tho....
or kabaddi  kabaddi ..... in one breath. He tries to touch as many players
as possible while they attempt to escape getting touched. Their goal is to
grab the invader and hold him down within their territory till he takes
next breath. If successful, the invader is declared out but if the invader
comes, touches the opponents and successfully retreats home, the boys he
touched are out. Then a boy from the opposite side repeats the process.
Victory is decided when all players on one side are out and the opponent
still have players on the field. 

We play Kabaddi mostly at the Shakhas (Physical training classes arranged
by RSS—Rashtrya Swayam Sewak Sangh.” We wear khaki shirts, knickers and
carry a bamboo stick. After physical exercises, bamboo stick fight
training, a game of kabaddi is played. I must admit I was not cut out for
such games. 

British gave India the game of Cricket along with a common language,
railway system, a common enemy and constitution while at the same time
making Indians their slaves and subordinates for over two hundred years.
They left an independent India and newly carved out country of West and
East Pakistan in 1947. They vanished but Cricket stayed on as the most
popular sport for the masses. It is played at the National, State, and
school levels. 

It is played in proper gear and attire on a well prepared ground. Inside
a demarcated large oval field, a rectangular hard pitch, 66 feet long and
10 feet wide, is not covered with grass whereas rest of the field is. On
each side of the pitch are three wickets made out of round wooden stumps
with a horizontal groove on the top for the bails. A flat bat with a round
handle is used to hit the red hard ball. 

The game is also played by boys on the streets and local play grounds. We
play either in the oval grounds in front of our house when playing with
friends, in the yard between house number 2 and 3 with siblings and later,
during the Family Get Togethers, on the front brick lined courtyard. The
wickets is made of tree branches or vertically stacked bricks. Street
clothes without protective gear are the norm. The ball is either a cheap
hard ball or bouncy rubber one. Ground is irregular grass, asphalt or
bricks. Somehow we manage to get a proper bat. Most likely a gift from
Prem. Rules of the games are followed as close as possible. Our brother,
Krishan, is an exceptionally good cricket player as a batsman and
a bowler. He is captain of Sanatan Dharam School cricket team. We walk
with big smile and heads held  high during days of inter-school cricket
matches. When asked the reason of our joy, we would say with pride "I am
Krishan's brother!”

Minor injuries are common in cricket and occasionally major ones too.
Pajamas with large round bell bottoms are in fashion these days. 

While running, six foot tall, lanky Virinder once gets his foot stuck in
the flaying bottom of the pajama, trips and fractures his  left forearm.
A local masseuse fixed it. 

Bunte ( Marbles ) is one of the most popular games. We happily play them
under shade of trees or any available cool spot. They come in different
colors and sizes. We buy them, win and sometimes acquire them by doing
some work for others in return for marbles. They are our prized
possessions, carefully counted and stored in glass bottles. Missing ones
are detected, become cause for finger pointing and denials settled by
occasional fist fights. 

There are two ways we play them. In both versions we make a round shallow
hole in the ground--khutti. Each player rolls a banta from a distance of
about 4 to 5 feet toward the khutti. The one closets to it starts the
game. Each player gives two bante to the player who rolls them toward the
khutti. Some land in it. In one variation we squat and hold one banta
between left and right index finger with right thumb resting on the
ground. With the flick of right finger the banta is shot toward the one
pointed by the opponent. If he is successful in hitting it, the player
keeps the ones in the khutti, otherwise the next player takes turn. When
all the marbles in play have been won the next game begins. 

Sometimes a sore looser pees in the khutti. A common saying was "na khelan
ge na khelan diyan ge, khutti vich mootan ge.” It meant: neither we will
play nor let anyone else play, will pee in the khutti".  

Second version was to hit the pointed banta, while standing 4 to 5 feet
away, with a banta almost twice the size of a normal one. Winners and
losers are determined in the same manner as in the squatting version. 

We love playing bante so much that sometimes we miss school. As much as we
love having a large family of 8 siblings, the downside is that each one
had a circle of friends. None of us can get away with mischief. Kanchan's
friend reports to PitaJi when she found me playing bante during school
hours. Everyone knows PitaJi's temper when it comes to education of his
children. Puppy love becomes a bark. But we all knew that soon he would
feel guilty of making his babies cry and reward them with sweets from Bosa
Ram. 

Patang udhana aur Peche  ladhana (flying kites and get the string criss
crossed with other flyers) is one of the passions in early spring.  Major
festival of Makar Sakranti around middle of January is celebrated by
almost every household. We don’t celebrate the religious aspect of it but
enthusiastically participate by flying kites. 

The grounds and fields are open and endless, unlike now when every spot of
land is covered with residential quarters. 

The process of flying kites is a major production. We go to market weeks
ahead of the season and purchase crepe paper, hard bamboo shoots, spools
of cotton thread, glue, and a charkhari (a cylindrical piece of wood with
decorative flat round ends larger than the cylinder connected with one
round 4 inches long half inch diameter piece of wood attached on each
side. This is the spool on which the thread is wound.  

Old glass bottles at home are crushed using a stone mortar and pastel to
convert glass to extremely fine powder. Old pajamas or sheets are cut as
thin strips, tied together to make tails for the kites. Depending on the
financial situation we occasionally buy pre-made paper kites but almost
all are made our own hands. Paper is cut as squares, the bamboo sticks are
cut with a sharp knife to make thin yet sturdy strips of same size as the
kite. One is glued vertically down the two pointed ends. The second is
arched horizontally, the ends at the horizontal tips with the highest
point of arch near the upper tip of paper. These are glued to the paper
and set aside to let them dry. 

Glue is mixed with the glass powder and some wheat flour. This is called
manjjha. It is used to coat the cotton thread. We tie one end of the
thread to a tree or a post, rest is wound between posts or trees at
several levels, finally tying the other end.  Manjjha is applied by hand
to the thread and dried in the ever present sun. Sometimes a sharp glass
piece makes blood appear on hand. We just suck the site or patch it with
manjjha till the bleeding stops. 

The sun-dried manjjha coated thread (Dor) is wound around the charkhari
(spool)

The kites have also dried up by this time. With a pin we make two holes
around the junction of the vertical and the arched bamboo strip and two
holes on either side of the lower end of the vertical bamboo strip.
A Y shape of two threads is created by using knots. One end of the Y is
passed though the upper set of holes and  the other end through the lower
set. They are secured with tight knots. 

We are almost there and visualize the kite in the sky. We toe other end to
the thread on the spool. The kite is balanced by tying small cloth pieces
on the lighter side. The tail is tied at the lower end of the vertical
stick. The sticks are bent and an arch is formed and finally the kite is
ready to up and away in the hot blue sky. 

On a windy day, standing in an open field or on roof top, one boy takes
the kite about 15 feet from the flyer. When a gust of wind arrives he lets
the fly go up while at the same time the flyer pulls in the dor making the
kite soar up. This was called 'kanni dena'. Deftly, by giving dheel
(letting go the dor) and intermittent pulling (tunka) the kite rises in
the sky. 

Sometimes we insert the dor end through a 3 to 4 inches square paper
before tying it to the kite. We write our names or a random message on it.
With each tunka the paper rises higher and higher till it reaches the
kite. 

 Other boys around are raising their kites joining hundreds of others
 which paints the blue sky. They look like numerous birds flying. Kites of
 variety of colors, sizes, and tails fill the air. With efforts we engage
 dor of our kite with other’s and give a sharp tunka. Glass in our manjjha
 is hopefully sharper than the other’s. The end result is that one of the
 dor gets cut and kite gradually undulates as it descends, carried farther
 by the wind. Several boys run to catch the free floating prize. They run
 after the loose kite, on the ground and on the roof tops. On lucky days
 we retrieve our own or steal other's. Sometimes it get stuck in overhead
 electric wires and is considered lost. Sometimes the kite lands on
 a tree. Fearless children climb the trees attempting to be the first one
 to retrieve the kite. Falls from the trees and roof tops are natural
 hazard but youth trumps all fears. 

Laughter, sounds of running foot steps, and fights over who caught the
lost kites first fills the sky and earth. During the festive season it is
a fair game and no one fights. 

My fear of phobia of heights is traceable to seeing Shoki stepping down
from the roof on to the concrete awning of a high window at 30 Railway
colony. He was on his way to retrieve a kite stuck in an adjacent tall
tree. He did get the kite but I held my breath, afraid that he might fall
down anytime. Since then I cannot look down any building 6 floors or
higher. I also get nauseous just seeing someone else do it. 

Kho kho is another outdoor game I rarely play. It is a strenuous tag game
between two teams comprising of 9 players in one and 12 players in the
other team. 

Football, called soccer in USA, is sometimes played in the playground. One
of our friends, Bhashi uses his head a lot to hit the ball and is called
Takroo. He also uses his head in occasional serious fights with other
boys. 

The girls play their own games. Boys never play with them. I don't recall
any reason for it...it is just not done. They play Shataapu, Bante in
a different version, dolls, and other games I don't  remember. They help
mothers in the household chores. 

Boys are obviously spoiled but we never thought about it till now when
I see our sons-in-law helping with household chores. We are creatures of
our times and environments and become what we see and perceive as the norm
for that era. Children become what they notice and not what they are told.
I don't recall PitaJi doing any household work other than taking care of
the yard and garden, procuring items for food and all other outdoor
necessities. At no time Mata Ji asks him or us to help with cooking or
laundry. 

The only time Pita Ji cooks is when he can afford to buy a chicken. He
sets up a chulha outside the house because MataJi does not allow cooking
meat in her kitchen. 

As if the outdoor games were not enough, we had many indoor games to keep
us entertained and educated. My earliest memories go back to our paternal
grandfather, Lalaji plays sweep with his friends. Once in a while he says
“Ve mundiyo (O boys) ithe aao (come here). We eagerly run up to him,
a rare thing for us. 

He teaches us how to play card games --Sweep and Dussar. PitaJi also loves
playing sweep with his friends after he retired at age fifty six, in 1962.
This coincided with the last son leaving for college. Occasionally if one
of his partners does not show up, we get a chance to play with them. They
take the game seriously and tempers flare up at times. Mata Ji complains
that they are having fun and she has to make tea for them. 

Snakes and ladders on one side and Ludo on the other side of the board are
fun to play by boys and girls. But we soon grow out of the childish game. 

Chess never really caught on in our family; it needs too much brain and is
too quiet for our family. Games have to be noisy and filled with fun and
laughter. 

Prem has taught us a very enjoyable game called Aaloo matar gobhi. The
whole family sits in a circle and have equal number of cards dealt.  The
stack of cards is kept face gown in front of each player. Each one of us
is given name of a vegetable or fruit. Everyone tries to get a hard name.
By turn each player quickly opens the top card from front end and lays it
down. Other players follow suit. When the last opened card matches one
open card of another player, the two loudly call out each other's given
name. First one to call correct name gives their open pile of cards to the
other. In the event of a dispute the remaining players become referees.
The aim is to get rid of all the cards. All kinds of names get yelled out.
If in a hurry, when the cards don’t match and a  player calls out a name,
the opened cards of named person are given to the one who is too eager.
There is lot of noise, wrong list of names blurted out, disputes about who
was first to call out the correct name. 

It gets confusing when the number of players approaches 5 or more and
invariably it is more than 10. It is loud, friendly, noisy but always
filled with lot of laughter. Once Laxman Jeejaji got so excited he almost
got his glasses thrown off. It is not just mouths screaming, hands are
flaying too. 

In-Between is another game Prem brought to the family. A pool of money
created by equal amount contributed by every player is collected in the
center. Each player is given two open card. By turn the banker/dealers
gives players one additional card, if wanted by the player. The recipient
has to bet a sum of money, amount of bet depends on the chances of the
next card to be in between the two cards in hand. Ace being the highest
and 2 being the lowest creates the best chance of next card to be in
between these two. The bet placed in this case  is high. The least chance
to win is if the two cards are like 7 and 9 or jack and king. There is no
chance to win if the two cards are same. In this case the player may pass. 

One also has to keep in mind how many cards have been dealt and what are
the chances for the next card to be in between. Maximum bet is the amount
of money in the pool and how much money the player has. 

This drama is deeply ingrained. Bhushan Jeejaji has ace and 2, giving him
the best chance for the next card to be in between. He has very little
money to bet and the pool is humongous. Pita Ji is the banker. Jeejaji
bets his watch against the advices from many. But greed and Hope make us
deaf. There is suspense in the air. Pita Ji deliberately delays dealing
the next cats. Finally he opens the next card. It turns out to be ace.
Jeeja Ji loses the watch to the banker—Father-in-law! PitaJi has choice of
keeping the watch with an upset son-in-law or breaking the rule and
returning the watch to a disappointed loser. He wisely returns the watch.
Peace is more important than greed. 

Prem also brought to the family a game called 21 or  pontoon which we
found later was same as Black Jack. Bets are extremely small not by choice
but by reality. We hardly have money to buy food, let alone gambling. Good
thing is that winning and losing stays within the family. The bughee
system of pooling family money evens out the winning and losing. 

Rummy dealing 7, 10 or 13 cards is played depending on how many players
are in the game. This again is a rather quiet game and therefore never
popular. 

Whenever Virmani Mamaji visits, a foursome including Suraj, Prem and
Pitaji is made to play Bridge. The younger three are never included. Three
tail Enders don’t care for the serious game. Virinder sits out and watches
very carefully. Whenever a player is needed to make the foursome they let
Virinder play. Once he made one mistake, he was scolded, as it is commonly
done during the postmortem of the Bridge game. Mama Ji scold him “Tenu
bridge khelni kadi nahin aaye gee” (You will never learn how to play
Bridge). He took criticism to heart and made a vow to learn this game as
best as possible. At age 76 he played championship matches and invariably
is first among a group of three to four hundred players !. Now at age of
86, he is very sharp and has patiently taught it to me also. But he is
kind and never scolds me and he is happy to see me less and less mistakes.

At our annual FGTs, he accepts me, a novice, as partner against Suraj and
Prem who refuse to be my partner. Virinder and I lose every time because
of me. But not once he scolds me. Instead, during Covid days in 2020 and
2022 he taught the game to me on Zoom and now at age 86 he and I play once
a week on line. He is kind, patiently points out the correct play and
never scolds me. He is happy to see his student make less and less
mistakes.

Scrabble was brought by Prem. It becomes increasingly popular as our
vocabulary improves in school or through the game. Bhushan Jeejaji is
invariably the winner with the best vocabulary in the family. Only once
I was able to beat him by making a seven letter word—rosette. It must have
been a rare sweet victory because even the word is in my memory bank after
over sixty years. 

One by one, the birds flew out of the nest. Noise and laughter got
replaced by quiet and solitude. 

Pita Ji played Sweep and Dussar when he could collect other retirees. When
the groups dwindled, finally PitaJi played Solitaire, all alone. Seeing
the scenes of laughter-filled games of the glorious days. A quiet tear of
joy and sadness was felt in the throat, eyes and nose. 

When I write about these games, I know that hardly any of these are played
by the children in USA. I also wonder how many children in India play them
these days. 

Times change, lifestyles change and so do the means of entertainment and
physical fitness. We all have our memories of our childhood. Hope the
children today live their childhood fully and when they look back at this
golden period of life they exclaim—Wow those were really great days. 

Today's wonders become wonderful memories for tomorrow. 

August 21, 2014

