\chapter{More Festivals}
Most of the times we simply follow the traditions and festivals handed
down to us by our parents and the society. The meanings behind the
symbolism of the events are uncovered, if ever and correctly, much later.
Time has a strange way to alter the traditions as the distance increases
between the time of the occurrence of the event and the time we experience
their memories. 

Festivals do continue to carry their main essence as they are passed on
from one generation to the next. The same Festival may also be celebrated
in different regions with completely different meanings, rituals and the
reasons behind the celebrations. 

In Panipat, Haryana, India in the 1950s, we celebrated festivals through
out the year. Following Dussehra and Diwali, the most elaborate ones in
the pleasant months of October and November, chill starts setting in
Panipat. In the absence of any external means of heating, bed sheets and
thin razaais (cotton filled covers) give way to thick ones to cope up with
cold winters. 

Outdoor beddings get moved to tight but cozy sleeping areas indoors on the
beds and on the floors. I don't recall ever feeling cramped. In fact our
house felt palatial. 

Paternal grandparents traditionally live with the sons. Our  Dadaji
(Father's father) Lala Gokal Chand and Nani (Mother's mother) Mrs. Kesar
Bai lived with us. Since mothers did not traditionally live in daughter's
house, she insists on paying rent. She feels psychological better that she
is not a burden to her daughter. Nani has money which rest of us can use
for essentials but more importantly for splurging on supplies for the
Festivals. 

Our home has four rooms in a line with one small kitchen at each end. The
farthest left room is occupied by Dadaji, fondly called Lalaji and next
one is used by Naniji. By this time the eldest two brothers, Suraj and
Prem have left home for studies followed by jobs. 

Remaining eight comprising of parents, two daughters and four sons live in
the two rooms and the attached veranda. The verandas on the two sides
originally were open but soon were enclosed to create additional rooms. 

In this set up the traditional celebrations of the Festivals were carried
on despite difficult emotional and financial times. 

Under one roof there was the unending energy and curiosity of the children
and knowledge of the elders who knew the customs transmitted over decades
or centuries, and now ready to be passed on to the next generation. 

Children, especially the younger ones have less responsibility and more
time for fun. 

Festival of Lohrhi comes along on January 13 to mark the end of the brutal
winter and celebration of Rabi crop. This is an event spread out over
several days, although not as elaborate as Dussehra and Diwali. 

Starting about 10 days before Lohri, after school and a quick gulp-down
snack, we along with a group of friends assemble on the street. Boys and
girls have separate groups. 

We start collecting logs of wood, corn, sesame covered jaggery and
never-to-refuse money from neighbors. Some householders consider receiving
money as begging and discourage children from accepting money. For us this
is the most essential and not considered begging. Every evening we chart
out a territory to visit. We knock on the door of every house in the
selected area of the day. One boy leads the chorus while others say "Hoi"
in unison. Sounds of "Sunder mundriye-- Hoi, tera kaun vichaara --Hoi,
Dulla bhathi vala, dulle dee dhee viyahi, ser shakkar paai, kurhi da salu
paata, salu kaun samete,...Each line followed by loud Hoi. We do not know
its meaning, significance or its connection to the Lohrhi, but we sing it
loudly with great enthusiasm. This is sung to the owners of the house who
have stepped out to greet and join the fun. 

At the end, home owners give to the boys a handful of revrian made of
dried refined white or brown sugar coated with Sesame seeds, Phuliyan
(puffed rice), roasted moongphali (peanuts) in the shells, kernels of
corn, few pieces of dry wood or money. If the home owners are generous,
all the boys jump commensurate with the amount received. Generous ones
receive a chant from us "Ganga bhai Ganga , eh ghar changa." (first three
words are purely for rhyming purposes, the last three mean this house is
nice.) Stingy ones get "Hukka bhaai Hukka, eh ghar bhukha" (this house is
miserly)

The girls make their own groups, have their own Lohrhi song. They do not
venture out as far as the boys are allowed. It is not considered safe for
the unaccompanied girls. 

The caravan of boys keeps going around door to door, a scene somewhat
similar to the Halloween in USA, without any special costumes.  All the
neighborhoods are very safe. Boys go around worry-free without any parents
watching or protecting. When it gets dark, we all run back, put away the
collections of fun-filled hours of hard work. The piles of wood, revrhis,
corn, peanuts and money grow every day till the day of Lohrhi. 

On the day of Lohri we get dress up and in the evening, we pull out all
the wood that the neighbors have given us plus the ones we had collected
from the woods. The small twigs are set at the base to be used as
starters. Bigger logs are stacked at an angle carefully leaving air spaces
between them. Older members of the family and neighbors sit on the floor
or on the few chairs that have been pulled put. 

With deafening noise and contagious enthusiasm the fire is lit. We cheer
as the flames get higher and higher. Boys and girls run around the fire.
The arms are stretched out to get warmth to the hands from a safe distance
and put on cold cheeks. We keep doing it many times. Corn kernels are
thrown into the fire. Cracklings snd popping sounds  associated with
flying hot red particles are fun to hear and watch but may also be the
source of burnt spots in the shirts or saaris. Revrian and phullian mixed
are placed on a platter and moongphalis in bags. They are passed around to
everyone enjoying the fire.   Everyone picks up the eatables by handful
and munch away. Shells of peanuts are tossed into into the fire. It is fun
to see their edges glow red and then the shell turn into a ball of fire to
ashes. Corn seeds are also thrown in the fire. 

More firewood is added till the fire from last log turns to charcoal.
Buckets of water are splashed over the simmering charcoal. Crackling sound
from the hot coal and steam swirling in the air announces the end of the
show. 

Reluctantly everyone gets up, leaving the warmth of fire and fun, to get
tucked under the thick razaai and use the body heat to warm the cold bed. 

By March the weather improves significantly. Greenery starts filling the
trees, plants and fields; flowers start blooming which make the landscape
come alive with colors. To match the nature, we are getting ready for the
Festival of colours, Holi. 

Once again preparation start several days before the big day. Our total
inventory of clothes to wear is very limited but even among those clothes
there are some that have become torn and unwearable. They are separated
and kept aside for Holi. 

The markets become alive and vibrant with shopkeepers displaying various
colored powders in open sacs or large round metal containers. Once again
Pitaji gives us money to go and buy the colors. Feeling rich, off we run
to the Bosa Ram Chowk. Ghan Shyam Das, the retailer in the shop next to
Bosa Ram, has the largest supply in our neighbourhood. Trembling with
excitement we debate which colors to buy and finally settle for what and
how much we can afford. Red, green, and  yellow are the most popular
colors. Ghan Shyam weighs them on a scale and fills up bags made of
recycled newspapers. 

The stockpile keeps getting bigger as the day of Holi approaches. 

We also buy Pichkaris, (water guns),  which are made out of large hollow
bamboo pole about a foot long with a solid piston. The front end is
chiseled to make it pointed. Pichkaris looked like large wooden syringes.
In later years metal and plastic pichkaris start appearing in the markets. 

The colors are used as dry powder or mixed with water in buckets. Even
large drums are filled with different colored water. 

Excitement of the upcoming day of Holi  keeps mounting. On the big day we
get up before sunrise, quickly eat breakfast and change into clothes which
we will discard after the event. 

The army is ready to go out with our armamentarium of packets of powder,
bucket full of colored water and full pichkari ready to shoot. 

Other friends are already on the street or about to join. Everyone is
shouting "Holi hai" and with that we throw colors at one another, smear
the exposed parts of body including hair and then shoot out coloured water
at the willing victims. 

Everyone, known person or stranger, who steps out of the house is a fair
game. Some try to run away or try to get back into the houses but the
young army is swift and catches the victim. If they are too far to catch
they are caught by the long reach of water being shot from the pichkari. 

Several groups run into one another and if they see the participants
already unrecognizable from the colors, they save the supplies for other
victims. Some groups join each other and the Tola (group) gets larger. 

Everyone is jumping, laughing, shouting "Holi hai" on seeing the next
clean person. They happily get colored, squirm, laugh, run, get angry,
resist or even fight according to their temperament. But the mob
overpowers and right before our eyes, within a matter of minutes, a clean
body wearing clean cloths is turned into a multicolored, unrecognizable
person. 

Due respect is shown to females. Girls are drenched in water color from
a distance. Older women, if they asked to just get a speck of color on
their forehead, cheek, hair, or edge of cloth, they point out the area and
one or two boys follow instructions with full respect. Some women do not
care and are covered all over. 

Widows wear white clothes and do not play Holi. If we know someone is
a widow and even if seen by us, we do not throw colors at them. The older
ones do not mind the exclusion but some young widows, deep inside their
hearts, wish to take part but the pressures of society keeps them from
enjoying the festivities. No such rule apply to widowers. There are not
too many widowers as they frequently get remarried whereas widows are
cursed by the society to stay isolated singles till they die. It never
feels fair. 

We run home to refill the buckets from the drums, pick up more bags of
colored powder and run back to join our groups. 

Some boys have access to trucks or open tempos. Sometimes we are allowed
to ride in them which takes us to areas beyond the reach of our feet. From
that height we can use our pichkari to reach those just standing outside
their doors within the boundary walls of their houses. Fun loving ones
stay and get wet while others quickly run inside. The doors and walls pick
up the colors. 

The action goes on till past noon or when the supplies run out. With
clothes completely soaked, body including hair painted with all types of
color, we come home. We all look alike, fully rainbow colored bodies. 

Exhausted from walking, running and laughter we are ready for a long bath.
Large drums had been filled with water which would no longer be coming
from the faucets at that time of the day. Hand pumps are in full use.

The clothes are thrown away after attempts to clean and save for next Holi
fail. Faces and arms are clean but some color has soaked deeper and will
take few more days to shed off. 

A nice big meal with deserts of halva and kheer has been prepared by
Mataji. It is quickly devoured by the hungry growing boys. Girls also play
Holi but not as wildly compared to boys. 

We hear that in some homes they crush poppy seeds and make an intoxicating
drink out of it. No one in our house does that or even talks about it. 

April 13 is a much celebrated day in Punjab. Haryana used to be part of
Punjab but it became a State by itself in 1966 as Punjab was split based
on linguistics. We are Punjabi speaking people but now got rooted in the
Jaat majority State whose basis of formation is Hindi as their spoken
language. Mataji used to stay "Asi tha hameshan refugees hee rawan ge”(We
will always stay refugees.”) At first Pakistan kicked us out and now
Panjab has disowned us. 

Punjab and Haryana are mainly an agrarian society and are considered the
food baskets of India. 

Vaisakhi, as is known in Punjab and also called Baisakhi elsewhere,
heralds the beginning of harvest season. April 13 is considered as an
auspicious day because it is the opening day of harvest season. This
brings prosperity when the farmers start selling their crops. 

Home delivery with help from Daai (an experienced woman for deliveries) is
the common way babies see the light of the day. No records of births are
recorded. With large number of children, parents have hard time keeping
tracks of their birthdays. Use of lunar month complicates remembering the
date on solar calendar. We never had birthday parties. Therefore the date
of birth is not required till the admission to the 5th grade in a school. 

One day Pitaji takes me to Sanatan Dharam High school in Panipat for
admission. The Principal asks for my date of birth. Not knowing the answer
he thinks of an auspicious date and said "Thareekh thaan yaad nahi aye.
Thusi 13 Aprayle likh chadho. Sun shaayad 1944 see." ( I don't remember
the date. You may write 13 April. The year most likely was 1944). That is
how my official birthday became April 13, 1944. 

Mataji’s version, corroborated by Kanchan, the memory champion of the
family, is that my birth was in the evening of no moon night in late
November of 1943. This happens to be around November 24, 1943.  She was
playing outside when her friends ran up to her and told her "Larhka hoi
hai" (boy is born). She then ran into the house to greet the new addition,
saw him and went back to climbing the tree. 

Vaisakhi is a festival mostly celebrated by Hindus and Sikhs and
especially farmers. It was on the Vaisakhi day in 1699 when Guru Gobind
Singh initiated the Panj Pyaare and started the Khalsa Panth. We are not
Sikhs but do have some in our genealogy, Ram Singh being one of them, and
our Dadaji has the Sikh holy book, Guru Granth Sahib in our home which he
reads everyday. He keeps the back door open and welcomes any friend or
stranger come in and read the holy book any time. 

We are not farmers after we reached Panipat following the partition.
Before that Pitaji was a full time farmer. After partition he was allotted
a small piece of rural land near Panipat where vegetables for home use and
for the workers are grown. 

After bathing, Mata Ji makes halva and poori for the breakfast. We put on
new, if affordable, but surely clean washed most colourful clothes from
our meagre wardrobe. It is festive around us but we do not have any
special Bhangra parties as are organized by the Sikhs in our
neighbourhood. 

In Khanewal the family used to go to a nearby creek and take bath in it
early in the morning. 

The Festival of Rakhi, also known as a Panjabi version and used in our
home--Rakhdhi or more formally called Rakshabandhan (Raksha meaning
protection and bandhan meaning tie or tie a knot) is celebrated every year
in the month of August. The date of this function varies because it is
performed on the full moon day in the fifth month of lunar calendar,
Shravan which got abbreviated to Saavan, corresponding to between July 23
to August 22. Lunar calendar has fewer days than the Solar calendar in
each month. Therefore the Rakhi date changes every year. 

It symbolizes the love between brother and sister. Times change,
necessities change but traditions continue. Historically women and young
females were in danger of getting harassed or kidnapped by the invaders.
Sisters prayed for long life of their brothers and brothers promised to
protect the sisters. The threats and kidnappings by the invaders are long
gone and forgotten but the tradition of symbolism and reminder of love
between sisters and brothers continues to be celebrated. 

Weeks before the day of Rakhdhi, many stores in the markets start
displaying a varieties of colorful, ornate bands made of cotton or silk
thread, called Rakhdhi or Rakhi. Some are plain weaved threads. Red color
is the most common choice. Some are decorated with colorful stones and
beads. They are laid out on white cloth covered tables or covered charpais
(Four legged portable beds).  Bosa Ram is again a happy man because sweet
mithaai is a significant part of the celebration. 

Sisters buys one Rakhdhi for each brother. If she does not have a real
brother she ties Rakhi to a cousin brother. 

Our mother, Mataji, was her parent's only child. She used to tie Rakhi on
our Mamaji (mother's brother), Chuni Lal Virmani’s hand. Chuni Lal was son
of MataJi's chacha (Father's brother). If a female did not have brother or
cousin brother, she would pick a boy who also agreed to become her brother
with all its responsibilities such as protect her, perform some ceremonial
functions and provide utensils to the sister's daughter when she gets
married. 

Occasionally some boys and girls take advantage of the system. Dating is
a taboo these days in IndiqIf a boy and girl love each other and are
afraid of being seen together by the parents or society, they pretend to
have Rakhi tying relation to be able to openly meet without getting
reprimanded. A few known ones get taunted -- Din mein behan bhaai , raat
ko banee lugaai. (during the day brother and sister and becoming lovers at
night). It is a vrare occurrence. 

On the day of Rakhi  everyone takes bath and wears freshly washed clothes.
After saying a prayer the sister ties the rakhi on brother's right arm
just above the wrist while praying for and wishing long life for him.
Brother promises to protect his sister as long as he lives. He hands her
money and/or a gift. They feed each other a piece of sweet mithaai. "Rakhi
mubaarak, may God give you long life”, they hug and say to each other. As
children we are less interested in the Rakhi than sweets. 

Once during FGT at Panipat, all six brothers and two sisters were present.
6 rakhis were tied by each sister. 6 forearms were arranged in a circle
and photographed. Krishan died that year on October 12,1987 at the age of
44 from a heart attack in Shimla. Since then we do not take collective
pictures of the rakhi clad arms. 

September is the month to take a breather. It is time to start thinking of
and planning for the big one—Dussehra and Diwali. 

Amidst good and bad times Festivals and celebrations continue at house
number 2 Model Town, Panipat. 

But all good things come to an end, only to be reborn again, somewhere. As
the circle of life moves, young ones grow up, move out and with them move
out the festivals, but the sweet memories of vibrant joyous days linger.
The providers stay back and reminisce about the golden days played out
under their wings. In their mind they can see what a beautiful life they
created for their children, relatives and friends. Ever so grateful, the
recipients carry the memories and plant the festivals in their own
gardens. Cycle continues although the intensity declines. 
