\chapter{Getting From Here To There}
Living in the USA for forty eight years, we along with millions of others,
use cars, trains, buses and planes to get from one place to the other.
Apart from major metropolis areas, very few people are seen walking or
using bicycles. Once we reach our destination, we get into all types of
buildings and homes. Streets are devoid of people. It seems that the vast
land has very few inhabitants.

Born in Khanewal, Multan, India, now in Pakistan, I grew up in Panipat,
India following the partition of our country. When I was three years old,
we moved to the old city, Panipat in 1947. In 1950 we moved to House
number 2, Model Town, Panipat in a newly constructed development to
accommodate some of the millions of refugees who got uprooted. Panipat is
located about fifty miles north west of Delhi.

Being the children of partition, we have only heard stories and seen
pictures of caravans of dust covered, starving people walking, if they
could not carry all children some choosing whether to carry across boy or
girl-the boy is preferred, some too weak to walk succumb to starvation,
dehydration and heat. Some are riding on horses, bullock carts carrying
whatever material goods that could be salvaged, and trains, some full of
slaughtered bodies, trying to get from there to here and here to there
between the newly carved Pakistan and the parent country India.

Once established in our new home, we did not move around much beyond the
limits of Panipat.

I can see my smaller version and shorter legs moving around in Panipat and
an occasional trip out of town. Full of joy and energy, totally oblivious
of the extraordinary historical trauma my family and the whole region and
the country had gone through.

Walking is the commonest mode used by us. Our little legs make us feel
that the distances are much farther than what we see now when we go back
for our annual visits. The brain calculates distance by number of steps we
take and not the actual distances.

Trains and buses connect us to the rest of the country. Only a few rich
people have cars. For us they are a novelty to watch and just touching
them is a thrill. But never a grain of “Oh, we wish we had one.” We didn’t
have much but we had abundant and were happy in what we had. Beyond
contented. Once a while we see an airplane making us wonder how it flies
and who can possibly afford to fly in the sky.

There is no local public transport or taxis. All local travels have to be
organized on our own within the means.

Among the 10 family members, comprising of three generations, we have only
one vehicle. That is our prized possession—an Atlas bicycle, manufactured
in the neighboring town of Sonepat, Haryana, the site of largest bicycle
factory in India.

It is a regular sized black bicycle with a handle bar ahead, connecting
bar to the seat and a carrier on the back. The carrier has a snap latch to
keep contents secure. The U shaped stand is at the rear end. It is folded
under the rear wheel when bicycle is not in use. A shiny silver bell on
the right side of the handle produces jingling sound when the lever is
pressed by right thumb. Pita Ji, in Khanewal, used to collect fireflies,
collect them in a glass jar which was tied at the middle of handle bar.
Bicycle would be visible to others at night. We never tried that in
Panipat.

The chain has no cover. Loose pajama ends sometimes get caught in the
chain. The thin tires have rubber tubes in them.

Riding on rough roads and meeting discarded nails, punctured tire and tube
is fairly common. In frustration we say "phir puncher ho gaya" (got
punctured again). Every Chowk (major crossing) has a bicycle repair shop.
The mechanic, usually a boy working under supervision of an older man,
takes off the tire from the rim and pulls out the tube from inside the
tire. The boy inflates the tube and puts it in a tub full of water. Rising
bubbles detect the culprit hole. He cuts an appropriate size square piece
from a pre-glued rubber sheet and sticks it to the punctured area on the
tube. The patch is then firmly attached to the tube by pressing it with
a hot smooth metal for an airtight seal. The boy inflates and rechecks it
under water. No bubbles is a moment of joy and laughter. Tube is deflated
and deftly placed inside the tyre. The combination is placed over the
metal rim with inflating valve stuck out of the hole in the rim of the
wheel. Wiping sweat from his forehead the boy inflates it by pumping
laboriously the long hand pump, supported by feet placed on the two metal
flanges near the lower end of the pump. Once the tyre feels solid, it is
ready to go, knowing fully well that the next trip back is not too far in
the future.

Seeing 8 or 10 patches on a tube is very common. It is cheaper to repair
than replace is the rule, not only with the bicycle but everything else.
That is one thing that confounded us when we moved abroad, where just the
reverse applied. Till today the old mentality of saving and one day
somehow fixing it results in piles of clutter. Due to extremely limited
resources, throwing away anything, including the bicycle tubes covered
with numerous patches, is just not done in Panipat.

Bicycle is used for any trip beyond Bosa Ram chowk, named after the famous
supplier of sweets and fried salty dishes such as samosas, pakoras and
mathian…

In early sixties , the prettiest sight in Amritsar is Dolly on her bicycle
riding down the road between Medical college and Dental college. She rides
on her bicycle as I wait for her to pass by for a glance only to be
ignored. That, however, did not last too long.

After our marriage in Chandigarh, we purchased a blue/grey Lambretta
scooter which had been booked several years ago by Dolly's father.

The two wheeler scooter is how Dolly and I get around. Then came along our
daughter, Namita. Initially she sits in Dolly's lap as Dolly sits sideways
on the back seat. All are without helmet. Later Namita sits between the
us. When she becomes sturdy and steady on her feet she stands in front of
me on the small platform between the driver seat and control handle. Once,
while overtaking a car, we barely escaped getting hit by an oncoming
truck.

Jobs take us from Chandigarh to Ambala to Jind and Rohtak. Our favorite
Lambretta takes us to various places safely.

PitaJi, as we fondly call our father, uses the bicycle to go to his newly
allotted land about 20 miles south of Panipat along Grand Trunk (GT) road.
The road was originally built in about 300 BC during Maurya empire. It had
been improved by a Muslim king, Sher Shah Suri, in mid 1500s to connect
his sprawling empire, which he had won from the second Mughal emperor,
Hamayun in 1540. The road connects Kabul, Afghanistan to Calcutta in
Bengal, India. It is a two lane road with no painted dividing or the edges
line or paved shoulders.

One has to ride the bicycle on the dusty unpaved shoulder of the one of
busiest roads in the country. The road is occupied by trucks, buses, cars,
tractors, three wheel tempos stuffed with passengers, some hanging out
from three sides, motor cycles, bullock carts, horse drawn tongas,
rickshaws, camels loaded with large jute bags full of hay on both sides,
occasional elephants and, of course, cows, buffaloes, donkeys, mules,
goats, dogs and people walking on foot.

All are trying to get to their destination as fast as possible except for
some cows and buffaloes. They use the flat ground for an afternoon
jugali-- regurgitation and mastication of hurriedly stuffed food earlier;
the white foam lines their lips and later followed by a snooze. Traffic
slows down and goes around them. Cows are holy, having been tended to by
Lord Krishna

Horns blare constantly, the issue of right of way is decided by the might
is right rule. Seeing three vehicles, two on the road and one on the
unpaved and ill defined shoulder with no guardrails is not uncommon.
Overtaking from any available or forcibly created space is the norm.
Flashing headlights means “Watch out, I am coming through, slow down or
move over if you don't want an accident.” Rule of the jungle dictates who
flashes the lights first and if both do it simultaneously, might is right
rule applies. It is not uncommon for two opposing forces to play
psychological warfare, sharply swerving at the last second to avoid the
collusion. Slight tilts of the vehicles and near misses are taken for
granted. Near misses sometimes become collisions or overturned vehicles.

Overturned or lying sideways trucks in the adjoining ground dot the length
of GT road due to the speed, overtaking and use of alcohol. Desi sharab
(locally distilled alcohol) shops are set up along the road. Most of these
have now been removed and been replaced by Dhabas (informal eating places)
and fancy restaurants. Two lane road has grown to 6 lanes divided highway
with a barrier in the middle and guardrails on the sides. There are paved
shoulders and adjacent service roads for slower traffic, animals and
walkers.

PitaJi has no such luxuries. Wearing a hard khaki hat, loose cotton
clothing, he makes these perilous trips on GT Road to and from the
allocatedaand fortunately without any accident. He never complains about
the ordeals he has to go through to shelter, clothe, feed, educate, marry
off; quietly serving the ten occupants in different ways. Including
himself, the ten included his father, Lala Gokal Chand; mother in law,
Kesar Bai, who apparently had brought money from Pakistan and pays rent
for staying in her daughter's house; two daughters, Kanta and Kanchan and
four sons--Virinder, Krishan, Juginder (Gindi) and Ashok (Shoki). The two
older sons, Suraj and Prem, have moved out for studies and job.

In addition to the bicycle being used to go to the farm, vegetable market,
the old town, and ice factory, it is also used by the boys for joy rides.
There are no helmets and knee guards. Falls and resultant cuts and bruises
are common. Once my right thigh got caught in the spinning spokes of the
wheel. We just covered the cuts and bruises with milk from wildly growing
plant called Ak (milk weed) and plain dirt. We have never heard the word
Tetanus. Later in medical college we had a tetanus ward and learnt that it
was a prevalent and potentially fatal disease. But we turned out to be the
lucky ones.

Sanatan Dharam High School where Virinder, Krishan, I and Shoki studied,
is located about one and half miles east of our home. We walk on the dirt
tracks, along ditches filled with water during rainy season, escaping
swarms of mosquitoes by swaying our hands or note books, cross the railway
line and then walk to the school located just across the GT road. I am not
particularly fond of going to the school and prefer to play. Some days
Virinder takes me to the school on the bicycle and see to it that I have
entered the gate. Some days, to his surprise, I run home faster than he
can return on the bicycle!

No one in our section of the Model town has a car. Everyone walks or uses
bicycle. Usual destinations are the school, market, one another's home,
temple, and playgrounds. There are three playgrounds. Oval one is across
our home, flat one near Bosa Ram chowk and doonga (deep) in a large ditch,
near house number 100, where our Bhabi (Brother's wife), grew up.

Rickshaws, three wheeler vehicles, pedaled by thin, sun tanned muscular
men, are used by people at the railway stations or bus stops. One or two
passengers sit on the vinyl covered seats with or without the pull up
shade depending on the weather. Riksha vaala pedals on the flat roads and
gets off to pull rickshaw when he has to go uphill. Some considerate
passengers get off and walk along to give him respite. Some are not so
kind and want to get full value for their money.

Every time a rickshaw comes to our house, it means an unannounced welcome
guest or ever-so-missed children from college have arrived. Ways of
communications are rudimentary to non existent. "A crow was crowing this
morning. I was wondering who will be coming to our house today,” MataJi
says and luckily crows crows a lot in 2 Number.

Another way of travel is Tonga. It is a horse-drawn colorful wooden
carriage with large wooden wheels, pulled by one horse. This is used as
a taxi. Like rickshaws, they line up outside the railway station and bus
stand. These are more spacious and faster than the rickshaws. You also do
not feel as guilty, especially when going uphill. The sound of metal
studded hoofs of the horse on the road makes us think of movie songs
directed by O. P. Nayyar. The driver sits on the front seat with leash in
one hand and a long narrow stick or a whip in the other. Two passengers
can sit on the back seat and one on the front seat with the driver.
Sitting on the front seat provides the view ahead but is perilous because
horse often lifts the tail up and provides manure to the ground. Luggage
is kept near the feet of back seat passengers.

Once in a while we are enamored by a dark blue Fiat car that stops in
front of our house. This means that our loving Virmani Mama Ji has arrived
from Delhi. Chuni Lal Virmani is the son of our Nanaji's brother. MataJi
is the only child and at a young age she had started tying Rakhi to Chuni
Lal, which made him her brother and our Mamaji. Car is a novelty and
a rare luxury. The children line up and with stretched necks examine the
inside through the windows. We are careful not to touch it with our dusty
hands. Mamaji is very loving. He opens the doors and sometimes let us into
the car. To our amazement, once he let me sit in his lap for a drive
around the park. Shoki sometimes sits on the driver's seat and pretends to
be the owner.

The next car we experience this close is when Bhushan Jeeja Ji comes on
his Standard car carrying a Gold Flake round cigarette container--both
signs of luxury and unapproachable high living from where we stand.

Not much traveling is done by the children or MataJi. Some summer
vacations are spent in Ludhiana at our Narang Bhaaiya and Sumitra Bhuaji's
house, while during other summers they come to Panipat for three months.

This journey is done by train. Packing is not difficult as there are not
too many possessions and in summers one does not need anything more than
two cotton shirts, a pant and two knickers. One set of chappals and one
pair of shoes complete the packing. No tooth brushes are needed because
neem and keekar trees grow everywhere. One end of a branch of these trees
is chewed to make it like a tooth brush.

We get to the station at least half an hour before departure time, go to
the counter to buy a rectangular beige colored ticket made of card board
for the third class compartment. This one has wooden benches, in two
tiers. We have never been inside of second or first class compartment. We
do sometimes peek in to see the other world.

The platform is a busy place with people are carrying round cloth bound
packings. Some bring steel or leather suit cases. Hold all is used to pack
and roll beddings. Some people are sitting on benches, some sleeping on
the floor. Tea and snacks stands are busy with people and flies. Stray
dogs are sniffing around for left overs or food sticking to the scattered
papers on which food was served, on the platform and on the rail tracks.
Books and magazine stores are busy with browsers and some buyers.

Trains rarely came on time. It is a common joke that once a train got in
on time, the excited passengers later got disappointed when they found out
that it was from yesterday's schedule. Generally they are several minutes
to several hours late and there is no way to find out which is true. We
all wait patiently, occasionally looking to the left to hope for a glimpse
of smoke emitted from the engine or a down signal.

At last the train arrives, pulled by a large black, smoke emitting steam
engine. Most compartments are almost full and the existing passengers are
reluctant to open the doors while the incomers have to get in somewhere,
somehow. If some passenger has to get down in Panipat, the door of that
compartment has to open. That is like winning a lottery for the eager
entrants. Everyone rushes to any voluntarily opened or pushed open door.
After several minutes of shoving and pushing with occasional fight
erupting, the people and luggage settles down. There are no assigned
seats. After the storm there is a lull. One by one people find a spot:
a full seat, a partial one shared by the prior occupant, floor, vestibule
near the door and some hold the entry door railing enjoying fresh air.

Some pull out food, wrapped in a cloth piece or metal tiffin carrier. Food
is generally fried pranthas, dry aalu bhaji ( Potaoes), hard boiled eggs
and achaar (pickle). It is customary to offer it to the neighbors. With
full stomachs some doze off, some pull out book or newspaper and some pull
out playing cards for the leisurely long journey.

The melodious sounds of a steam engine train starting to move are
unmatched. Green flag waved by the conductor and down signal gets the
driver to move the train. A long shrill whistle followed by chuk chuk chuk
starting at a higher pitch and slower beat gradually reversing the pattern
as the speed increased, whistles of the engine blare to alert late
boarders to hurry up and warn people crossing the railway lines in front
of it. Plume of smoke swirls up from the top chimney, steam gushed out
intermittently from the sides and the train slowly pulls out with people
waving to the loved ones on the platform and vice versa. Some walk and
then run along till they can not keep up any more.

Coal particles fly out making many a people cry in addition to the
emotional tears of separation. PitaJi is not the one to cry and sometimes
would make an appearance and excuse of getting a piece of coal in his eye
to cover up the emotional tears that rolls down as he watches his children
leave home one by one.

The trips are long and pangs/joy of separation from the loved ones is
a sad/happy occasion depending on which side of the fence one stands.
Parents warn children "Don't stick your head out looking towards the
engine. You will get coal in your eyes." Mostly we listen but the
excitement to look out and ahead into newly opening panorama overpowers
the caution and we pay for the mistake.

The trains have three levels of seating. First class, which we never see
from inside, apparently has individual cubicles with plush seating. Second
class has open light blue cushion seatings. Third class has open hard wood
benches. We never notice the hard seating as we are busy sitting, walking,
hanging by the handles of open doors, eating, playing cards and just
chatting. One favorite pass time is to watch the telephone lines as they
go from pole to pole making a pattern of sagging down, then climbing
predicting arrival of the next pole.

The bus journey is the other way to go out of town. Being not as
comfortable, we ride the bus for rare short trips to Delhi. Itihad bus
service runs buses between Panipat and Delhi. We walk to the bus station,
buy a paper ticket and wait for the next bus. Just like trains, the buses
are also packed. The luggage is kept on roof top by climbing the steps on
the back of the bus. Sometimes we are lucky to get a seat but mostly one
just stands, holding onto the leather straps hanging from the central
metal rod on the ceiling. Some boys and men barely have their feet on the
entry platform and hold onto to the handle bars for survival. Some sit on
roof top and some hang onto to the ladder used for carrying luggage to the
roof top.

Women find bus journey particularly hard as pushing and pinching by some
men is common. A few tell the men off but most suffer in silence. Bus
journey is demeaning and source of life time scars.

Sumitra Narang, our Bhuaji from Ludhiana was visiting Delhi to attend
someone's funeral. She traveled by a local bus. Driver has no way to see
the entry/exit doors. While she was getting off, the bus moved too soon
and she fell hitting her head on the road. She left behind five young
children and a grieving husband. In my memory that is the first time
I cried.

Pursuit of education and my father’s dream takes me to Rupar, Panjab in
1959 at the age of 15. The journeys to and from college are made by train.
My brother, Krishan is one year senior and already studying there. We have
no bicycle or other vehicle here. We walk everywhere. Admission to medical
college takes me to Amritsar in 1961, where again we depend on our feet
for moving everywhere. Out of the class of 100, only two boys have cars
and a few have bicycles.

Pursuit of further education takes us to Post Graduate Institute,
Chandigarh. Walking and Lambretta carry us around.

Lufthansa takes me away to UK. At the airport, on March 28, 1974, over ten
family members see me off. Dolly and Namita hoping to join soon. MataJi is
wiping her nose with end of while saari. PitaJi cannot find the excuse of
coal particle getting into his eye. His dream and his doctor son boards
the jumbo jet to get there from here. He just let tears flow, wondering if
he would see again his son and, soon to follow, his daughter-in-law and
grand daughter. Alas, he never did. Buried by the excitement of the
future, it did not hit me then but the scene of separation is vividly
alive. A tear wells up and I taste their tears.
