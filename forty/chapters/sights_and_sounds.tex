\chapter{Sights Sounds and Smells}
"Khuda ka vaasta tumhen. Aap ko apne bachchon ki kasam. (For God's sake.
You swear upon your own children). Please put your sword back. I have two
wives and five small children to take care of,” stuttered the man, wearing
a round white cap, pleading for his life. His trembling body, the betrayed
fear of imminent death in his eyes, the hands clinging to the feet of
a man with three white lines painted across his forehead, begged for
mercy. 

Before bringing the sword down, which would create a limp body immersed in
a pool of its own blood, the Hindu said "Don't make me swear by my
children. They were mercilessly killed by your brothers in Pakistan last
week. Right in front of my own eyes! My pregnant wife tried to save them.
The long sword simultaneously pierced through her and our one year old son
she was holding close to her chest. I was made to witness this torture,
tied to my cot. Their plan was to put the cot on fire at the end of the
carnage, but an urgent call to kill another fleeing large family took the
killers away. Besides, the striking strip of the match box had become wet
by the spurted blood from the little neck of my child. The first son saved
his own father's life! Should have been the other way. I squirmed and
desperately pulled the rope trying to free myself to save my family, my
blood running through the small bodies. I begged for mercy offering my
life in return of saving my wife and children fell on deaf ears. The guilt
will kill me little by little every day, every moment for as long as
I will breathe. And don't worry about your two wives and five children; we
severed their heads from their sinful bodies in the other room five
minutes ago. They are at peace.” 

Then all was quiet. The severed head, now capless, jerked around for
a while till it settled next to the quivering body soaked in its own
blood. The three lined foreheads and the bearded, turban wearing men,
brandishing their blood tinged shiny swords, moved viciously onto the next
target. They had vowed to take revenge for losing their wives, children,
old parents and all the material possessions they had owned for
generations. 

We occasionally heard such stories from our relatives and other elders,
but read many more similar incidents much later in life. The dark shadows
resulting from the partition of India were kept away from our innocent
minds. 

The sounds and sights of the hue and cry of the departing Muslims and the
arriving, displaced Hindus and Sikhs in Panipat must have been experienced
by me, a three year old boy in 1947. The screams of the victims, the
insane fury of the perpetrators and sights of mangled bodies on the trains
coming from and going to the newly formed Pakistan must be buried deep in
my grey matter. All these horrific events transpired following the
division of India and creation of the new country, Pakistan, on August 14,
1947. 

Like a river, life has the tenacity and need to keep flowing regardless of
the current or preceding turbulent obstacles, small or large. Survival
mode kicks in, time for regrets and counting the losses is placed on the
back burner. 

As children, we did not fathom the mammoth obstacles and turbulences our
elders were going through, enduring their effects and myriad ways of
surmounting them. We were busy enjoying our childhood. Surely our parents
dealt with the numerous problems, but they kept the younger ones insulated
from the trauma. 

"Suraj and Prem, you guard the house from 6 AM to 4 PM. Virinder, you
watch the gate from 4 to 7. I will stay up all night", Kundan Lal, our
father, known as Pitaji, ordered his three elder sons, who at that time
were 19, 16, and 11 years old respectively. They were the oldest of the
six sons. In addition to the three youngsters, one only few months old,
two sisters need full time protection as well.

One day there was a scare in the household as Kanchan, the younger of the
two, wandered off without telling anybody. Unaware that the whole family
was frantically searching for her, with fears of every horrific
possibility in their minds, she was merrily playing with some friends.
Such was the frightful environment in the early days after the partition.
All the elders had their hands full immediately after the family's sudden,
rushed arrival in Panipat. 

Pitaji's paternal uncle, Dr. Gopi Chand, who, with his family, was passing
through Panipat on his way to the safety of Delhi, had earlier sent an
urgent message to Pitaji "I have possessed and am holding an abandoned
house, vacated by a fleeing Muslim family. It will be perfect for your
large family. Come as quickly as possible before someone else pushes us
out.”

Our whole family, including my pregnant mother, for the tenth and the last
time, had come from our permanent home in Khanewal, District Multan to our
summer home in Sabathu, a cantonment town near Simla, in May 1947, for the
three months of summer vacation. The family comprised of my parents, seven
siblings including the one who was yet to be born in August, 1947, and
Lalaji, my paternal grandfather. Maternal grandparents had stayed back in
Sargodha, Punjab, and migrated with great difficulties after the
partition. Our family's plan was to return to our permanent home in
September, 1947. 

All types of information was circulating about the possible and impending
partition of the country. No one knew for sure when or if it would really
happen. But then, without much warning, partition was suddenly announced
in June, 1947 and history of the region and lives of its people changed
instantly. Our family abandoned our land, home, and almost all material
possessions, but was safe, together except for Nanaji and Naniji. 

Pitaji immediately led our family from Sabathu to this abandoned house,
deep in the city of Panipat in the fall season of 1947. The house must
have been vacated in a rush because some pots, filled with partially burnt
vegetables, were still sitting over the cold ashes of wood burning
chulhas. There was no electricity; scorpions and occasional appearance of
snakes came with the territory. We called it our first permanent home in
the Independent India. This was the shelter which protected us from death
and destruction. It had to be guarded around the clock. 

In 1950 we moved to the new development called Model Town in house number
2. This became the permanent home to live and die in for Mr. Kundan Lal
and Mrs. Vidyawati Luthra, my parents. This was the safe haven where they
happily struggled to raise a large family in Panipat. 

Sights, sounds and smells of Panipat in the 1950s started registering in
a recollectable manner in my memory bank. Activities of a busy household
occupied by twelve persons, ranging in the ages from 3 to 80 years,
surrounded my young senses and mind. 

The home is filled by sounds of laughter and crying; scents of food, rain,
flowers, and odors of open latrines; and constant flurry of activities
beginning before the sunrise and ending past the sunset. Mata Ji and Pita
Ji’s grit and foresight had this package delivered to us in an envelop of
security and much love. 

The city wakes up with the hazy sun filtering through dust, smoke from the
burning coal and wood, and cool fog. Crowing of the roosters and chirping
of the birds in the abundant trees are our wake up calls. 

Mornings are also announced by prayers and hymns being sung over loud
speakers. The newly arrived Hindu and Sikh refugees have brought their
religious practices with them. Many brave Muslims have stayed back in the
old city. No Muslims reside in the newly developed Model Town. Each sect
started their days with competing sounds of Allah U Akbar, Gurbani, chants
of Hare Rama, Hare Krishna or recitation of Hanuman Chalisa to invoke the
god Hanuman. Hindus and Sikhs set up Temples and Gurdwaras in the newly
constructed houses. Several old mosques are present in the old city. 

Our family's place for prayers, Shri Ram Sharnam does not participate in
this loud morning ritual. Swami Satya Nand Ji had separated from the Arya
Smaaj, taught that the enunciation of the word Ram is the sound of the
unseen God. Small gatherings started at the home of Shakuntla Behanji in
house number 76, and later number 9, both of which became too small for
the escalating large gatherings of devotees. A large hall named Shri Ram
Sharnam was inaugurated on October 9, 1960 at 588 Model Town, which has
been growing ever since, now under the Guruship of Darshi Ma. Melodious
Bhajans, Amrit Vaani, Sarv Shakti Mate Parmaatmnein Shri Rama E Nama and
discourses are recited over a low volume speaker system, during the
evenings prayer sessions. 

Once the fires of partition cooled down and extinguished, the three
communities live in harmony, peace and non judgmental acceptance. Every
religious sect or group and individual follows their respective religious
practices without questions or fights over My religion is better than
yours. Survival is the common Mantra. 

Roosters crow, dogs bark, crows croak, men and women with water-filled
Lotas in their hands make their way to the open fields. Some people are
seen taking baths in cold water, shivering, and chanting Ram Ram Ram or Om
Namo Shivaaye. Shopkeepers start raising shutters of their shops. Tea
stalls start preparing hot coal for making much needed morning tea. Smoke
starts to fill the air with its acrid smell and grey haze. Vegetable
sellers are busy filling their stalls with several fresh vegetables just
arrived from the Mandi. Chewing their daatan, spitting as they go along
the streets, cloth bags in hand, men and women are making their way to the
market to buy the supplies for the day. Milk vendors, newspaper sellers
peddle around on bicycles. Children, in their respective school uniforms,
start walking to their schools; some playfully happy and some still
yawning, missing their cozy beds. The whole town is waking up. Life of
Panipat is becoming alive. 

A large body of stagnant water, called Talaab, west of the railway line
produces an odor, which spreads out beyond its borders. It also provides
a fertile home for the mosquitos to hatch, multiply, bite and spread
malaria. 

The washermen (Dhobi) and women wash clothes at the edges of this body of
water. Dhobis soak the filthy clothes in water and soda and put them in
large containers, simmering overnight, over a light coal fire. The process
loosenes the dust from the clothes. The dirt-filled wet clothes, soaked in
soap, are beaten by wooden flat bats or swung hard onto flat rocks. They
are rinsed in the water and squeezed by twisting over and over till they
look like braids. Multicolored rows of clothes are hung on clothes lines
or laid out on the adjoining land along the Talaab to dry in the ever
present sun. Soap -filled water flows back into the Talaab. 

Dogs, half submerged cows and buffaloes bathe, urinate in and drank the
same water. This is where people and cattle take baths together. People
fill their metal container (Lota) with water for washing after defecating
in the open fields along the railway lines; many do not have outhouses or
latrines yet. 

Our Mataji, as we call our mother, and Ganga Devi, a regular hired help,
wash the clothes at home in a similar manner, except the water is from the
city water pipes or hand drawn pumps and the loosening of the dust process
is done from 5 AM to noon in a metal container called peepa. Locally made
pale colored, square shaped bar of desi soap for the sole purpose of
washing clothes is used. It is too harsh for the skin during bathing.
Commercially made, red Lifebuoy soap is used for bathing. Brown,
translucent Pearce soap is too expensive for us. Mata Ji and Ganga Devi
use about three inch wide bat with a round handle to beat the dirt out of
the clothes. A single steel wire, stretched across the whole yard, is used
to hang the clothes for drying. 

A newly established Sugar Mill is located at the southern tip of Panipat.
It is notorious for announcing the approaching Panipat to the people
traveling on the Grand Trunk Road, commonly called G T Road, and in the
trains. The characteristic pungent odor greets the visitors, making them
cover their noses till they go past the city or get accustomed to it.
Residents are surrounded by it, breathe it all the time and no longer
notice it. They are thankful for the much needed jobs provided by the Mill
to the thousands of refugees who have abandoned everything behind in
Pakistan. What is pungent to the outsiders is a sweet smell of livelihood
and life to the residents. 

Open sewers, lined between the homes and paved streets, carry a slowly
moving black sludge of solid waste from kitchens, bathrooms and latrines.
It is prodded along by the scant water and it was not uncommon for the
drains, called Naaliyan, to get blocked. The homeowners use water in
baaltiyaan (buckets) to move the smelly stuff beyond the boundary of their
own house. Private or City employed Jamadars (Sweepers) use a U shaped
metal piece, connected to one end of a wooden shaft, to pull out the black
sludge blocking the open drains, onto the sides of drains. The foul smell
and sights of these intermittent dark piles surrounded by flies are
perceived to be normal by us. 

In those days of poverty, I remember once throwing a one Rupee coin in
this sludge filled naali. The whole family spent, in vain, half a day
searching for the lost treasure with scoops and bare hands. Most likely,
that day we ate Roti with salt, onions and achaar. And surely I got
a thappadh or a smack with a broom. 

The sewage system is meant to carry off the rain water as well. It is
obviously insufficient for the job, as is evident after every rainfall.
The shallow playground gets filled and turned into a swimming pool for the
squeaking children and heat exhausted animals.

In addition to various fruit bearing plants, Pitaji has planted a variety
of perennial flowering bushes. Raat Ki Raani, (queen of the night) opens
its thin, elongated white flowers, spreading its unique scent all over the
1200 square yard. It also permeates into the house. It is said that the
smell attracted snakes; children approach the plant carefully watching for
any slithering movements. During the days, white Jasmine flowers, orange
and yellow marigolds, purple Bougainville provided color and fragrance.
Mango tree forms clusters of pale flowers, called boor. Anarkali, the
flower of yet to be born pomegranate, is one of the prettiest flowers in
our yard. Covering the center of the wall, between the two grey painted
entrance metal doors, is a pinkish white flower bearing climber plant.
Flowers from okra, eggplants, large yellow ones on the vines, guavas
plants and others mentioned above invite honey bees and bumble bees which
helps cross pollination. Butterflies of different colors and sizes hop
from one flower to the other. 

Once the fruits start forming, a large number of green, and some
multi-colored, chirping parrots provide constant beauty and nuisance. They
fully or partially eat the guavas. We take out our sling shots and aim
small stones at them. Mataji sews small cloth pouches to protect the
guavas and also grapes hanging from the grape wines. Pitaji gets blood
from the local butchers and pours it at the roots of the vines, a good
source of nitrogen. 

Crowing of crows, looking for unattended or discarded food, are connected
with superstition of arrival of unannounced guests. Their coarse sounds
contrasts the sweet sound of 'Koyal'. Anyone who sang a melodious song is
complimented "Bilkul Koyal jaisi awaaz hai" (Sound is just like that of
a Koyal)

Buzzing sound of the abundant houseflies during the day and mosquitoes at
night fills the air during summer season. Flies are ignored or just
smacked by hand or hand held jute fans. There are too many to kill. Flit
is sprayed in every room in the evening and the doors are closed for
a while. After that we enter the rooms closing the doors quickly before
the next platoon of mosquitoes follow. Darting tongues of the crawling
lizards, magically holding onto the walls and ceiling, take care of any
remaining insects. We look at them as friendly chipkali, but they spook
our children during our annual visits during the 1980s and beyond. 

Groups of stray dogs bark and roam the streets. No one has a pet dog or
cat. People throw a piece of roti or stripped and marrow sucked bones to
them. They become nuisance during the night and early hours of the
mornings, when they engage in loud barking fights over the prized
leftover, unattended or discarded bread or bones. Loud barking ends with
some dogs whimpering, declaring the winners and losers. Losers generally
tuck their tails and meekly walked away. The term Dum daba ke bhagna comes
from such scenes. 

Visits to Bosa Ram, the sweets makers is mesmerizing. The sights of barfi,
freshly made jalebi, laddoos make us salivate then and even now just
writing about it. 

Our home is about half a mile west of the busy train station and the
tracks, called railway lines. Another half mile further east is the main
national highway, GT Road. Dotted along and east of the GT road is the
crowded old city. 

The High Schools were built along this road. Everyone living in the Model
Town has to cross the railway lines and GT Road to get to the schools, bus
stand, public offices and markets. 

There is only one overhead bridge on Assandh road. It was constructed in
early 1950s to facilitate traffic between the newly created Model Town and
the rest of the city. It is used by pedestrians, bicycle riders, trucks,
bullock carts, tractors, occasional cars and buses. There is a small
pre-partition underpass south of it, which is used by pedestrians,
bicycles and rickshaws. It was the only safe way across the railway lines
till the overhead bridge was created. Now the underpass is mostly used
when the railway phatak (crossing gate) is closed. Phatak closes before
the arrival or after departure of the trains at the train station.
A variety of vehicles create long lines on the steep road on both sides of
the crossing. Pedestrians and bicyclists squeeze through and keep crossing
the lines till there is just no chance of escape from the whistling and
rushing trains. Some are not so lucky. 

Most of us do not go over or under the safe passages across the railway
lines because of the distance from home. We simply look both ways and
walked across the eight open railway lines as there are only four tracks.
At times we squeezed through in between the dubbas of long stationary
goods trains, hoping that they do not start moving while we are still
under them. 

Panipat Junction is on the main route connecting Delhi with north and
northwestern parts of India. The railway engines scream loud whistles at
all hours. This is partially to announce arrival and departure of the
passenger and goods trains. The main reason is to prevent accidental
crushing of people who continue their hops across the railway lines even
in the face of oncoming trains, betting their lives that their legs will
win the race against the lunging train. 

Endless blaring of the horns from the increasing number of cars, buses and
trucks is a normal phenomenon on the GT Road. Ringing of the bells by
bicyclists and ricksha pullers adds to the noise. "BauJi, hut ke" (Mr.,
please move), rickshaw pullers and tonga drivers say to the constant
stream of pedestrians competing for the space on the roads. 

Model Town is quiet; hardly anyone can afford a car. Sugar cane loaded
trucks and containers pulled by tractors, followed by the children running
and pulling out the sugar canes, drive on our roads on their way to the
sugar mill. 

"Bol Jamooore, peeche lambe baalon vaale sahib ki pocket mein kya hai,
chashme vaale sahib ke hat ka rung kya hai?" The man in the middle of the
encircling crowd asks the sheet-covered boy, jamoora, lying on the ground.
In a high pitch, the boy under the cover of the sheet, to the amazement of
the crowd, accurately describes a pen, a comb and the color of the hat. He
even describes what another person was planning to do that day, another's
future. A large crowd stands in a circle around the master and jamoora.
Eager ones push to come closest possible. The impressed crowd is willing
to pay a chavanni or duanni to the man for asking the jamoora their future
or a solution to their problems. Little do we know that the persons
described in the audience are planted and are part of the team. They
collect money from the fools and move on to the next village. 

Proponents of Shilajit attract another set of crowd standing in a circle.
Promise of better performance in the bedroom prompts many to part with
precious little money they have. 

Some are touting powders, pills and potions to cure bawaseer,
(hemmorrhoids), upset stomach, Kala motia (glaucoma), safed motia,
(cataract), infertility, curing all dental problems and everything else in
between. Such fake doctors try to sell their cure for all diseases in the
trains as well. 

It was not uncommon to see lay persons sit on the roadside pulling rotted
teeth with pliers. Without anesthesia and of course without license. 

Billboards and painted signs cover every visible wall along the GT Road
showing names of Hakims and RPMs-- Registered Medical Practitioners.
Bawaseer and infertility were the commonest issues. A big man, with a big
mustache, a Hakim appeared on many walls. 

Trained monkeys and bears perform tricks to amuse the crowd, who throw
coins into a tin container. We love when a stick holding male monkey
marries a sari clad female monkey and lead her away, apparently to his
home. 

Another crowd of people surround a couple of men, one playing the musical
instrument called been, while the other uncovers the jute baskets. A cobra
crawls out, it's head standing at right angle to the body, black beady
eyes follow the movements of the been, and at times ominously darts
forward, it's tongue sticking out to strike the been player. We get
scared, very nervous and worried for the player. Later we learnt that the
venom sacs had been removed and the snake bites, if they happen, are
harmless. 

Pythons crawl out of bigger containers and wrap around the bodies of the
performers. Some bold spectators are also given the chance for this feat.
For a fee, of course. To our horror, some bold ones take on the challenge. 

Many signs on the walls, some with the pictures of donkeys, saying--
Dekhiye, gadha peshab kar raha hai, (Look, donkey is urinating) does not
prevent men, their legs spread apart for the stream to flow through
without wetting their feet, from standing there and urinating. The smell
of urine permeates the air so much that we are accustomed to it. 

GT Road is also known for overturned trucks, hit and run accidents and
unattended victims on or along the its sides. If one sees a victim, it is
considered best to ignore and not report it to the police. The fear is
that whoever reports the accident, the police holds them as the prime
suspect unless proven otherwise. Ambulances do not exist. Those who do
make it to the sporadic government hospitals received no or delayed
treatment as they had become police case. FIR ( First Incident Report) has
to be filed before any examination or treatment can be initiated. Many
victims do not survive the wait. Almost always no one files medical
malpractice lawsuits. In fact, we heard that terminology when we came
abroad in the 70s. 

Roads in Model Town are wide and crowd free. Going through the galiyaan
(narrow streets) of the Shehar is another story. The narrow streets are
crowded, life in motion: People walk zig zag trying to find empty spaces,
rubbing shoulders, splashing red fluid, like a pichkari, from their mouth
indiscriminately on either side, while chewing the Paan; bicycles and
rickshaws weaving their way through the swarms of moving or gossiping
stationery bodies; people haggling loudly with the shopkeepers of the
various shops that line the streets. 

During the summer season most people wear white cotton clothes, dotted by
black burka covered Muslim women whose number keep increasing as we go
deeper into the old city. Their number rises significantly as we approach
the famous Kaldndar Chowk, the site of Kalandar's Masjid. 

Here the population is mostly Muslim and it was here that we had snatched
our first home in Panipat in 1947. 

Election times brings out rickshaws and tempos, with large posters
attached on the three sides showing names and symbols of the party.
Portable loud speakers blare loudly, urging people to vote for their
respective parties. Many people are illiterate. At the time of casting
their votes, they press their thumb on a blue ink pad and imprint it on
the ballot paper. 

One party always chooses symbol of a hand, to become easily recognizable.
There are many National and local parties with their individual insignias.
Most popular is the Congress Party. Their leaders wear white, boat shaped
caps. Their sign is GandhiJi's Spinning Wheel, Charkha. These symbols are
meant to cash in on the recently won Freedom from the British Raj of
almost 200 years. 

The Rickshaws are also used to have large boards depicting the currently
playing movies in the two cinema halls we have along the GT Road. Large
posters of these movies are also pasted on the prominently visible walls
and free standing display bill boards. 

Going to the cinema halls for watching films is a major event, although
rare due to financial constraints. Admission ticket varies from 4 to
8 Annas, equivalent to a quarter or half a rupee. There are three show
times: 3 to 6, 6 to 9 and 9 to 12 PM. Every few months we go to 6 to
9 shows. The hall is always packed. 

Sometimes tickets were sold in black market. There is no provision of
buying tickets in advance. They are purchased from a partially closed
window where a hand is inserted through a space at the bottom of the metal
guardrail. Money is exchanged for a paper ticket. Some people buy tickets
in bulk. Before long, the window is shut, a sign board depicting House
Full is placed next to it. The line is still long and people were eager to
see the film. The bulk ticket purchaser then, in hushed voice, goes around
and tries to sell the ticket by saying "thus ka ek, thus ka ek or bees ka
ek (one ticket for 10 or 20 Annnas) depending on the demand. 

It is an illegal act. But the cinema owners, black marketeers and even the
policemen, with their danda swirling in their hands, are part of the
scheme. 

Movies are about 3 hours long with an interval. Vendors came into the hall
selling tea or Moongphali (roasted peanuts in the shells). Some people go
out for the urinals or tea and cigarettes. Smoking is allowed inside the
hall but not during the showing of the film. National Anthem is played at
the beginning. Everyone has to stand and be perfectly still and silent. We
are relishing our freedom from the British. At the end of the show the
floor is covered with shells of moongphali. 

Dev Anand is our family's idol and hero. Puffed hair style and raised
collars at the back are copied by the boys. At times, even after being
told not to go, we still went to see the film. Once I stole 6 Annas, from
the space under Mataji's Singer sewing machine, to see the movie Jagriti.
Invariably we get caught and reprimanded by Pitaji. Kanchan tries to save
us as much as she can. 

The early morning music from Vividh Bharati still resonates with the fond
memories. Listening to the running commentary of the five days Cricket
test matches brought life to a stand still. A commonly used sentence
during these days is "Bau Ji, score kee ho gya hai?" ( Mr., what is the
score?). The radios and transistors are turned on at homes, offices and
shops. We listen to it on the fixed radio and later on a portable light
brown leather covered Transistor brought by our brother, Prem in 1957.
Binaca Geet Mala is something the whole family listenes to on Wednesdays
from 8 to 9 PM on our black Murphy Radio. Games are finished, food is
consumed before 8 and no other activity is scheduled between 8 and 9. The
voice of Amin Sayaani is ingrained in our minds. The show is broadcast
from Radio Ceylon. Bhaaiyo aur Behano, aaj ka dasveen padaan ka geet
hai....and finally, with a celebratory music, Pehli padaan (first
position) is announced. 

Such sounds fill the air in the noisy but now quiet household at 2 Model
Town, Panipat. 

I grew up surrounded by these sights, smells and sounds of Panipat from
1947 to 1959, when pursuit of higher education took me to Government
college, Rupar, Punjab. 

The early traumatic sights, smells and sounds from a fractured country had
been silenced by the glorious childhood. Poverty and misery had been
trumped by tenacity to survive, protective reflexes and limitless love.
They had been filled by the safety of a large loving home, sights of
abundant greenery, the fragrance of flowers, and sounds of laughter and
music. 
