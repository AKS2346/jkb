\chapter{Festivals of Our Childhood}

Life is nothing but a continuum of experiences... routine mundane
activities, sorrows, joys, cries, laughter, mournings and celebrations. 

Growing up in the 1950s in Panipat, India, gave us a plateful of all of
the above. 

Thanks to the wisdom and foresight of MataJi and PitaJi, as the boat of
our lives was going through the crests and ebbs of river of life, they
held us securely tight during those tumultuous time. They hid from us that
mere 3 years before moving to our home in 2 Model Town, we were all
refugees, had lost every material object including land and house, had no
money or job. They hid the tears and let us fully taste the celebrations
of life. 

Life seemed full of festivities and festivals. One festivals followed the
other, each one filled with wonder, laughter and joy. Even though it all
happened about 70 years ago, it seems like we the children are running
around, carefree, totally immersed in enjoying every festival. 

Dusehra followed by Diwali, filled with numerous activities, is celebrated
with great pomp and show. We are well conversant with the stories of
Ramaayan. We read Ramaayan not as a ritual of a religion but as
a requirement before dinner is served. After washing hands, we have to
pick up the thick book of Ramaayan, place it on a beautifully carved
wooden, folding X shaped book holder and each one of us reads two pages.
The thick thread is carefully placed between the last read pages to carry
on the story next evening.  

Readings at home and recitation of stories of Ramaayan in the temples make
the characters of Ram, Sita, Lakshman and others come alive for us. 

We the children go to the temples less for religious reasons but more for
the free Prasaad of boondi laddoos or loose boondi, sweet baked rice and
halva. Many times we join the lines twice.  Pandit Ji gives us a knowing
smile. Still hungry for more, we stand outside where devotees share their
sweets after having stuck a small portion on the mouth of statue of
Hanuman Ji. 

The Ramaayan stories are played out live on a stage through the Ram Leela
in the form of a drama every night for 10 nights leading up to Dussehra. 

This is held in the pleasant nights of October. After dinner, with a group
of friends, we run to the Ram Leela grounds. We carry a piece of jute cut
out from a bori (large sack used to pack whole wheat grain) to sit on the
dusty ground. The stage is set up by amateur local artists comprising of
only men and boys. Even the roles of women are played by Saari clad males
with long hair wigs and thick make up. They try to change the pitch of
their voice on the stage. They are all working people or full time
students with very little time for rehearsals or remembering all their
lines. Sometimes the voice of prompter is as loud as the artist on the
stage. We have a big laugh when we see the female characters behind the
stage lighting up beedis  and cigarettes, talking in regular voice.  

Ram Leela depicts the whole story of Ramaayan starting with Dashrath
seeing a grey hair In the mirror, to asking Ram to take the throne, forced
fourteen years exile to the forest, abduction of Sita by Raavan,
organization of monkey army, victory over 10 headed Ravan and final Lu
return to Ayodhya with whole city and country lit up to commemorate the
victory of good over evil. The whole story is played on the stage in ten
nightly episodes.

 The actors have colorful dresses, hand made decorative mukuts (crowns),
 bows and arrows. Amid the air filled with dust, which smells sweet, we
 become deeply engrossed in the story. We cry, laugh, get mad as the
 events seem real to our naive minds. The show generally starts around
 9 pm and goes on till midnight. 

Merrily, we walk back without any adult chaperones, climb into our beds
that had been laid outside in the front courtyard before leaving. In those
cool nights, a thick cotton filled razaai (quilt) and a pillow was all we
need to doze off in a matter of minutes. 

Next morning, sleep-deprived, we reluctantly wake up, get ready, have
a stuffed prantha with milk for breakfast and walk half an hour to Sanatan
Dharam high school, reliving the story of night before and looking forward
to the upcoming episode that night. 

During these days, at home, we get semi hard cardboard pieces, cut them to
the shape of crowns (mukuts), belts and containers for arrows. Straight
branches from the trees work as arrows whose one end is crowned with
pointed card board pieces. Bows are made of flexible branches from the two
neem trees on our land, one near the right gate and one at the junction of
house number 2 and 3. Shiny colorful papers are cut to appropriate shapes
and sizes and glued onto the card board pieces. A thick thread is passed
through small holes near the ends to tie the mukut around head or belt
around the waiste. Mustaches and wigs are purchased from the market.
MataJi sews colorful attires for the different roles and we have Ram Leela
at home. 

There are enough siblings to cover most of the roles. During summer
vacations Narang family children stay with us for about three months.
Among other activities we recreate Ram Leela in anticipation of the one to
follow in a few months. 

The tradition was revived once during our FGT (Family Get Together) at
Rishi Kesh. I still get a smile when I visualize Virinder with his mouth
puffed out to simulate monkey god Hanuman Ji, with a pressure cooker on
his shoulder depicted as the gada. 

Dusehra is a major celebration to depict annihilation of evil by the
virtuous ones, victory of Ram and Lakshman over the 10 headed Ravan and
his son, Meghnath and sleepy brother  Kumbhkaran. Almost 50 feet tall,
three effigies, filled with fireworks are created by numerous volunteers.
They are tethered to the ground with ropes and anchors. On the evening of
Dusehra thousands of people gather around the effigies. It is a noisy,
festive atmosphere. 

Vendors are selling all types of material to celebrate Diwali. Sweet
Jalebis are a favorite treat for the children. 

As the sun goes down, two men dressed as Ram and Lakshman pull out the
arrows and shoot them into the effigies. It is perfectly timed with the
organizers lighting the fuse to ignite the encased fireworks. Booming
crackers shoot out. One by one, the last one being Raavan, the effigies
catch fire and the wooden framework comes crumbling down. The crowd roars
in cheers and claps. As the last piece touches the ground we disperse and
head home, planning the next twenty days. 

Diwali is celebrated twenty days after Dussehra. Diwali celebrates return
of Ram, Sita and Lakshman back to Ayodhya after fourteen years of exile.
It also celebrates victory of righteous Ram over the evil Ravan, victory
of good over bad. These are special days, full of activities, laughters
and fun. 

The whole house is cleaned.  There are not many material things to move
around. A complete white wash (safedian) is done on the outside and inside
walls. Labor is hired to do the real work. White chalky powder (choona) is
bought in medium sized sacks. It is poured into steel buckets, water is
added taking precaution not to get the material splash into the eyes. We
are warned that it could burn the eyes.  We help the labor in mixing this
material with wooden rods. Small bubbles erupt as the mixing proceeded.
Koochees (jute brushes) are soaked with the wet choona and the laborers
apply this to the whole house. At the end of the day, their faces are
speckled with white choona. All the yellow streaks on the walls from the
seepage of water leaked through the roof during monsoons and the damage
done to the lower parts by the young hands are repaired and painted. The
boundary wall is also painted with choona, although it is pale in color.
The characteristic smell of freshly painted choona persists for a few days
in the house and forever in the mind. 

All the children get a set of new clothes and sometimes new shoes. Some
events become indelible in the mind, however insignificant they may seem
at the moment. In the Shehar (Old city) it is not appropriate to call them
roads as they are barely six feet wide with a drain on each side, carrying
rain water mixed with sewage, at times stuck and overflowing. We call them
Galiyan (small streets). Along one such street, in the center of the
Shehar, a Bata Shoe shop is located. One day PitaJi takes me there, sits
by my side as I try brand new Bata shoes!  He patiently lets me try many
till I settle for the ones that fit and I like. He never let me see his
expression whether he can afford them. And I had brand new shoes for the
first time! Normally we are used to wearing hand-me-down clothes and
shoes. But Diwali is special and special I am made to feel that day. 

It is not the amount of money spent on them that children remember, it is
the amount of quantity and quality time spent with them that they cherish
and save as priceless treasures that live on in their minds and heart long
after all the material goods and givers have vanished without a trace. 

With uncontrollable excitement we wait for new clothes. The tailor in
house number 40 is one happy man during Diwali season. Reluctantly, after
losing pleas and arguments, we have to patiently wait to wear them on the
day of Diwali. 

Krishan, nick named Harhtaalu because he was born during a country-wide
strike to oust the Britishers, meticulously cares for his possessions. He
folds his clothes, properly irons and stacks them in his personal locked
metal suit case.  The rest, in varying degrees, save them in different
shapes. I don’t particularly care what I wear or whether they were ironed
or not. There are more important things to do, such as playing and making
mischief. 

Sneakily, we get into Krishan's suit case and rumple up his clothes. This
makes him real mad, resulting in fist and stick fights. Kanchan finally
intervenes, punishes and locks up the mischief monger in the store room.
It is not a bad punishment because that is where all the sweet mithaai is
stored and locked. She doesn’t know or pretends not to know that there is
a secret way to remove the top shelf, slide thin arms and small hands to
get to the mithaai. It is a sweet punishment !

Finally the big Day arrives. All of us hugs one another wishing "Diwali
mubaarak ho.” Even strangers do not pass by without exchanging the words
"Diwali mubaarak, Diwali ki badhaai ho.” Family and friends exchange
wrapped boxes of sweets, mostly laddoos or kala kand. Bosa Ram makes more
money during Diwali season than the rest of the year. 

An integral part of Diwali, also called Deepavali (Deep means fits,
a lamp), the Festival of Lights, is to light up hundreds of diyas.  Many
days before the big night, we purchase hundreds of reddish brown clay
diyas of varying sizes from the vendors sitting on the floor along the
toads. Buckets are filled with water, diyas washed and soaked. They are
then sun dried. Large cotton balls are purchased. Patiently we pull out
pieces of cotton and draw them into shapes of about three inches long
wicks. Stacks of wicks are made taking care that they do not intertwine. 

 The day before and the morning of the big night, the house is busy as
 a beehive. A wooden ladder is placed against the kitchen, the lowest
 section of the house. Children carry the diyas and lay them side by side,
 about a foot apart along the perimeter of the railing-less roof edges and
 on the boundary walls of the house. Next step is to carry mustard oil
 and, without spilling, fill the diyas. Oil -soaked cotton wicks are place
 in the oil with one end sticking out at the pointed edge of the diya.

We are bubbling with excitement and can not wait for the sun to go down.
Candles are lit and all the children get busy lighting the oil dripping
wicks with the candle flame. Sometimes hot melted wax drips on the hands
making us jump with pain. Lighting of the diyas is happening all over the
town, and many parts of the country. 

A dark night is suddenly lit up with diyas, their flames flickering with
gentle breeze, making it a breathtaking sight. We go around with buckets
of oil for refills, wishing for the the light to go on and on. 

Fireworks (Pataake) have already been purchased according to each home's
budget. This is one time when PitaJi does not worry where the next rupee
will come from! Bags full of strings of red pataake, anaars, sparklers,
havaais, chakrees, are brought out on the street. We light them carefully
putting candle flames to the fuse and run back. Havaais are placed in
a glass soda bottle and the fuse is lit. They have a mind of their own
about which direction they will fly. Running away from these is absolute
critical. 

The air is filled with various levels of lights and sounds depending on
the size of the pataaka. Sky is lit with havaais shooting up all around
us, streets are lit with anaars, chakrees, jalebis. A round black flat
disc, once lit, starts growing in the shape of snake. This produces lot of
smoke, more than the other pataakes. Smell and sight of smoke fills the
air.  

A yelping running dog meant someone has tied a string of small red
pataakas on its tail and has lit the fuse. We dare not pick on a big dog
for fear of getting bit. Small dogs are our targets. Mercifully only a few
dogs became the victims of the innocuous fun for us but quite a scary
experience for them. Sometimes we keep pataakas under a small can leaving
the fuse out to be lit. Then we run back to see the can shoot up in the
air. No one knows about the dangers of eye injuries and luckily no one we
knew of got hurt. PitaJi carefully supervises and keep the little monsters
under check. Noise from the fireworks and children goes on till past
midnight. 

The stock of pataakas are finally finished, oil buckets gets empty,
glowing wicks get charred black, throats get hoarse from shouting and
inhaling smoke, even children's unlimited energy is really not without
limits. 

In between these activities, food has been eaten on the run, exchanged
boxes of sweets have been ripped open, sweets disappear before they saw
the lights. 

With full stomachs, tired eye and bodies we finally get into the beds that
have been laid out in the front vehrha (courtyard) by Virinder, MataJi and
PitaJi and whoever is around to listen and help. 

Tradition is to leave the doors open and some lights lit all night. It is
believed that Goddess of wealth, Lakshami, descends on the earth and goes
from home to home, blessing them with prosperity. Everybody wants her to
come into their house. It would be a catastrophe if she comes to the house
but finds the doors closed and no room lit, indicating there was no one
home to receive the bounty and she moves on to the next house. As a result
money will not come to careless one in the upcoming new year. 

Many communities and businesses commemorate Diwali as the beginning of new
year. Invariably the greeting cards say--Happy Diwali and a prosperous New
Year. 

Gambling by playing teen patti, a card game akin to poker played with
three cards and several variations, is traditionally a part of Diwali
festivities. People play this the night before Diwali. Level of stakes
vary depending on the financial status. Our stakes are in Paisas and
Aanaas. Even though we play, it is not a significant part of Diwali
celebrations in our home. 

End of one festival is sad in one way but also happy in another. We have
so many festivals to celebrate that the end of one mean anticipating and
starting preparation for the next—Lohrhi, Holi, Rakhi...
