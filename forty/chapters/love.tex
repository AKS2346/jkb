\chapter{Love}

(Nana asked me for my Memories of Panipat in the 1980s. Nearly every sentence is a question mark as much as a period.)

A one-story block-like home, peeled, faded paint. Entrance, small covered open area before entering home. Semi-detached with another low, matching structure that was someone else’s home. A family I don’t ever remember meeting. Walk through a formal room that no one seemed to use straight through to the dining/kitchen area. That and one other room is where everyone spent nearly all of our time. In that room too was a bed along the wall where Dadi slept. Just near the kitchen and the bathroom.

When we would arrive our luggage would be put in the large formal bedroom and we would take over that whole space. Wide berth for a wide American family. That room felt quite dark. Two wide beds shoved up against each other in the middle of the room. Steel almaris along the borders of the wall. All locked for safe keeping of bed linens, blankets, towels, clothes and whatever else lay inside.

In the middle room between the bedroom and the dining area was the room that was filled with all of us. Sleeping happening in all manner of space at night, singing and talks with what felt like 35 people crammed in sitting, lying at all levels, on the floor, on razais, on “sofas.” Lots of sounds and discussions happening simultaneously. Getting swept up in the Zindabhads! going around the room, spotlighting each person and celebrating them. (Lift Every Voice and Sing)

When this room wasn’t filled to the brim, I would spend time there quietly by myself and look at the glassed-in, formal book case and display case. There on the lowest shelf, you’d find framed, black and white photographs of every married couple. Suraj and Urmil, Prem and Shashi, Virinder and Santosh, Krishan and Chander, and so on. They didn’t look like the chachas and chacahis I saw every day in rumpled salvaar kameezes, hair unkempt, relaxed, aging. There, in the black and whites probably taken on honeymoons, they looked young, polished, slim and trim, “perfect” at the beginning of their married lives. Not to say that the fun-loving, fuller chachas and chachis in person were any less, in fact the full fun-level in the house was off the charts, but I do remember noticing the sharp contrast between those photographs and the reality. I also noticed that someone, probably Dadi, had taken great care and effort to frame those photographs and collect them one shelf. There weren’t any many other photos around so those were special. That room was rarely quiet and empty, so having time there by myself felt special. Perhaps there was a large, framed photograph of Dada high on a wall. Garlanded? A special puja we’d do for him. Beneficent, smiling, a twinkle in his eye. Father or grandfather of his huge lot.

The kitchen was a small box-like space, seemed enclosed. With a door? That’s where Dadi and her servant spent much of her time. Squatting on the floor, cooking, laboring with love to create dishes that pleased all those 30-odd people in her family. She didn’t seem to mind and her cooking was delicious and everyone told her so. All the love she poured into the fresh, hot parathas and the cold kali daal from the day before elicited loud raves and equal measure of love in return. But I do remember thinking that the place where she spent so much of her day pretty much on the floor didn't get much sunlight or any view.

In fact, right opposite the kitchen, quite removed by space was the outdoor latrine with an old-fashioned squatting Indian toilet that I suppose we all used. Now you’ll find “new research” that this method is better and more natural for the body. No kidding. We all shared it. We Americans probably didn’t like it, but no one really complained.

Here, I do remember noting the sharp contrast between Nani and Nana’s modern, more Western-style home – very clean, tidy, organized, minimal, quiet, structured, mannered, everything in its place, English or German-influenced – and this home. Nothing seemingly in its place, loud, boisterous, everyone sharing one latrine and one, small outdoor sink stuck on an outdoor wall, 17 different things happening at any one time. 17 different conversations or children’s games that one could join. All happening all at once.

I also remember thinking that Dadi slept on this small bed shoved up against the wall taking up a sliver of a space in a pretty large home. Not the formal bedroom, not the formal sitting room out front that sat empty unless hosting formal company.

She would try to sit with me and have a conversation. We would piece together words that made up an exchange of sorts. School, friends, food, teachers. We weren’t really exchanging thoughts or feelings, but we were sharing space and time sitting together. Her and me, one-on-one.

Outside the two wooden, screened doors that slammed all day long with people entering and exiting was “the backyard,” a cemented or stoned floor lined with trees and lush plants. The amrood tree was my favorite. I couldn’t believe that you could get the world’s best fruit hanging for free by just plucking one and eating it. That was an idea pretty close to heaven. Not all were ripe but I, like Dad, loved cacha ones as much as the ripe ones.

Here, maybe four large outdoor charpoys were set up in diagonals and no formation at all, just strewn about, moved and re-moved to people’s needs. People sitting and chatting, often chachis shelling peas from fresh pea plants that had been bought that morning or shucking corn. Again, lots of conversation happening in Punjabi about this person or that, this event, or that update. Sometimes I’d join in to listen.

The cousins were a mass organism of ages from near-adult all the way down to bachaas. Lots of running and chasing and games and laughing. We three from America slipped right in. At any one given time maybe 17 cousins in all. Playing outside and then inside in a circle for Murder. The range in age always struck me because the older ones – Nishi, Aarti, Mandy, Sandy, Ashit – were in a different league than the youngest like Rohit and Umang. Different conversations, different worlds. But all of us together.

There would be climbing the ladder with a chachas help up to the roof. Scary to climb up and scary too to climb down backwards. New view from up there of the neighborhood with its homes in a row, open naalis running along the sides collecting sewage and refuse. Dusty play area where local kids would be playing cricket or running around opposite the house. Did we interact and play with the local kids? Only a little.

When I’d play by myself I remember loving to make a big soup in some sort of caldron with a mixing stick. In it I’d put this leaf or that I cut up by my fingers. I was concocting some sort of stew and mixing it together over a pretend fire. I was playing house. Something I liked to do in Weirton, too, over the waterfall.

Another favorite spot was the swing. It sat in the doorway of another building structure that sat off to itself that I can only think housed storage. I don’t think anyone ever went in there. But the swing was there and it got used, a lot. All of us must have shared it, waited patiently for our turn. I didn’t remember any fights over the swing or anything else. No fights over food or “he did this to me” or “she said that to me.” I don’t remember bike riding out in the lane in front of the house or in the enclosed courtyard behind the house. But maybe there was a bike or two.

Just lots of fun, wild playing. We would probably drop in for five days and nights at a time. We would bring five huge hard-covered plastic-shell suitcases on two small, rickety and difficult-to-maneuver wheels that carried gifts that Mom would collect over the year. Christmas Day in the main sitting room where lipsticks and jeans and small objects from a land so far away and so different would be handed oit. A land none of them had ever visited but everyone on earth knew. We would go back with those same suitcases empty. All the toilet paper and cans of Chef Boyardee used up.

We’d spend hours on end going there and back on flights that had no TVs, no entertainment, except maybe different music channels coming in through the armrest and small headphones. All you could do was chat, sleep, or read. It was boredom personified. Maybe as I got older I’d have my own Sony Walkman, a sleek Japanese-designed personal music cassette player. So cool, so “leave me alone I’m a teenager.”

Did we love it? Yes. Did we enjoy the long journeys? Probably less so. Did we understand the financial cost of what going back during winter holidays meant to our parents? Probably not but it was probably exorbitant. Was there something there that drew us back every year? The people. Magnets like Prem, Urmil, Dadi, Virinder. Love. Did it cause conflicts between our parents? Yes. Fights over “your family got this” and “my family only got that” recordkeeping. Did we understand anything about how this home – Modeltown Number Dho – was found during the partition with a large Muslim family fleeing and large Hindu family taking its place? Not at all. Those stories would come much later. Would I have rather spent time with friends over the holidays? Probably. But that’s what we did as a family every year and I’m so glad for it. That’s the Luthra family in a nutshell – crazy, happy, chaos. Love.

