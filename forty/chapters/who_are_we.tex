Families are dynamic cycles of additions and deletions. We enter the lineage as products of the ancient genes modified by environments. 
Knowing the prior generations can help us to see the shoulders we are standing on. We appreciate the struggles, hard work, sacrifices and the love they gave to make what we are. We may even predict our physical and emotional make up, our predilection to some diseases and other traits.

When we see or hear the names of our ancestors who are one or two generations ahead of us, we know their names, faces and the persons behind them; their physical appearance, education, vocation, family, dreams, hobbies, achievements, failures, struggles, time spent together, life and many times its end. The third generation may be experienced as names and patchy life conveyed to us by others, but mostly a cloud covered shadow. Go one more, even their names become just that or in many cases they draw blanks. 

The persons whose names appear in our genealogy surely were born, made their parents happy or sad depending on the sex of the child and the period of historical timeline, had grown, had dreams, joys, heartaches, personal, religious, and political roadblocks, achievements, marriage, children, laughter, worries, sickness. They, like most people, would have had a passing thought of their inevitable death. Their progeny sadly had flowed the ashes down the Ganga river and their whole life suddenly got shrunk to a pot of ashes mixed with ever flowing stream and then just a picture on the wall for one generation and then a name in a book. 

Namita asked me how did Mata Ji and Pita Ji meet, who arranged their marriage etc. I did not know the answer to this and many other physical and psychological aspects of their lives. I did not ask and they did not tell. 

Some day somebody will ask our great grand children, how did you end up in USA, why did we chose to come to Weirton, how did we meet. What were our achievements and problems, survival in new country, professional issues, children etc. each one of us has similar questions but no documented answers. It should not happen again when some of us become just a meaningless name. 

In Kundan Lal Luthra Khandan (Lineage) we know the names for seventeen generation and time line of some of their lives. Thanks are due to the tradition of cremation, our faith in the holy Ganga, immersing the ashes of the departed ones in Hardwar, and above all the Pandas/Purohit (Priest/record keepers) with their red cloth bound ledgers, which they have maintained for centuries. Father transfers the red books to his sons who carry on the business, which continues till today. They make a living through donations and the family members of the deceased gain benefit of the entrees of names of current and previous generation thus creating a treasured genealogy record. In 1990 our family Purohits were Pandit Kashi Ram, Lachi Ram and Ram Kumar Shastry. The home telephone numbers was 01334-251803 and cell number was 9897232496. They had records of Sahiwal Luthras. 

Several family members of our expanding generations are lucky to know our current ancestors. This is a result of the bonding through growing up together, Family-Get-Togethers (FGT), attending weddings, other family functions, emails and now What’s app. 

In an attempt to keep a record of our current members, those who came before and will follow us, we need to start documenting who we are, where do we come from, not knowing where we are headed. 

A continuous, annually updated record will be of immense importance to all of us currently flowing in the river of life, soon to merge in the ocean of time and the ones who are just joining as rivulets. 

There is always a beginning to the story. In absence of one for our family, I will start with what I know. This is based on the contributions of all of us who have lived contemporaneously over the years, through family, conversations, lives observed, letters, emails, what’s app and records in Hardwar. In this age of fast improving technology, current and future generations can continue it, improve it, digitize it, add photographs, and preserve it. 

We do not want to see Mataji and Pitaji just become names in the red book in Hardwar. Their life will be remembered as walking, talking, feeling, loving persons they were and still are in our hearts and minds. 

In nature, seasons change. Only a few weeks ago what were the most beautiful colors, some bright red, some pale and some still green. They all take the toll of time. Gracefull, with beautifully planned, and executed actions of nature they fall. Oh, what was so wonderful few weeks ago on the trees, it is sad to see them gracefully disengage themselves from the branches, twirl around in the air and gradually settle down on the ground, decorating the earth with their colors for a few days, and then blown away by the wind. They disintegrate slowly, finally becoming food for the flora that will survive the winter, the new ones that are itching to grow taller, and the seeds that are currently awaiting for the spring to come so they can gleefully open their husks and live life as if this will be the first time anyone has lived. Too soon, and it always seems too soon, time flies. 

The snowstorm comes, the colorful leaves surrender, the trees become bare. The colors decorate the landscape. The tree trunks, the backbone, earthy brown, rugged and strong, stand steady, withstanding the severe windstorm. If the tree is standing alone, the mighty wind knocks it down. If the trees are in a group, they break the force of the wind, sway back-and forth. Finally the harsh wind subsides. They all stand erect, tall, and together. It is not one tree, or the other that is responsible for the survival of the group or this particular family of the trees. It is all of them, the small, the mighty ones with the deep roots, the thin stemmed ones which have the flexibility of bending, and not breaking –it takes all types for the family to survive. They do need the anchor around which all others contribute their reinforcements. They still need the backbone, the heart and the Will to pull through rough times.

Our mother and father, Mataji and Pitaji were the giant trees. They have been blown away gracefully by nature. But no one can take away the colors that they displayed, the colors they left on the ground, the colors they passed on to the living trees, the saplings they helped feed and grow and the seeds, even the ones that are yet to become seeds have imbibed their colors through genes. We can attest to the fact that their colors are still glowing in our hearts, homes, our lives. Their memories are making those colors vivid to even those who did not get a chance to see them, experience them. Those who love and are loved never die, they always live in the hearts and memories of those who received the love. 

We are relishing the love-filled experiences of all who have been reminiscing and sharing their memories like sprinkling of rosewater. These memories are their blessing, being showered upon all of us. Carefully, the backbone and the heart, Pitaji and Mataji, lovingly cultivated the saplings and enabled the seeds to flourish for generations to come. 

The Partition of the country was the massive storm which came in their and our lives. The giant backbone and heart, with help from flexible saplings, made the family survive and thrive. The family lives on. 

I belong to this middle class conventional family having seven brothers and three sisters. The second born Mahinder and the 7th born Sudesh died around age 6 years and one year respectively from complications of Small Pox in British India. The surviving ones went through the upheaval of the partition of India in 1947. My parents, Mata Ji (Vidya Wati Virmani) and Pita Ji (Kundan Lal Luthra) stabilized the boat of family undulating in the waves of violence-filled ocean of life. 

After the partition, my father, a man known as PitaJi by us, Lala Kundan Lal Luthra or Luthra Sab by many outsiders, was quiet by nature. So observed everyone around him, but hardly anyone knew why. There were almost insurmountable problems, obstacles, responsibilities causing immense constant internal chatter in his mind. He wanted to spill it out, share with others, cleanse his vessel so he could also become lighter, talkative and alive. The dilemma was that those  around him were people who were burdened with similar internal chatter and had no place to add more to it. They neither talked, nor listened but lived one day at a time. The rest had no idea about all the tumultuous times the newly displaced refugees, their families and the country had gone through. Everyone was busy in the survival mode, looking for the next shelter, next meal, next rupee; simultaneously protecting themselves from the enemies and tenuous friends who would not mind maiming or killing for their own survival. 

We, as children, did not hear his voice most of the time because of the above reason and because he was mostly away from home raising scarce money for survival of family. This included himself, his father, his wife, her parents and eight children. The other elders also were quiet in our presence to protect us from the horrific events created by the violent bloody partition. 

Years passed by and the young man grew prematurely old physically but fully alert mentally. The forced nature of being quiet stayed despite the vanishing noise of the chatter. 

British were hoping that the strife would delay their departure from the most precious jewel that had fallen in their lap by default and later expanded and controlled by their cunning tricks. They had perfected the technique of divide and conquer. But it was mostly a result of the short sighted, selfish politicians, princes and kings of numerous States. The country ruled by the British and 529 princely States, bound by the oceans and Himalayas collectively was called India. Divide and conquer, though an old trick, was somehow ignored or not seen by the short sighted leaders. British played all tricks to instigate division along religious lines as well. They were very happy that Indians were fighting with one another on the basis of religion, caste and acquisition of land, rather than fight against a very small minority of the real oppressor and common enemy. DAV (Dayanand Ango Vedic) College came up in 1888 in Lahore. Islam College and Khalsa Colleges opened in 1892. 

Weakened by the Second World War and also facing the mass movement of Swaraj (Self Rule), the British now had no choice but to move out. Prime Minister Attlee announced the decision to quit India in the House of Commons on March 15, 1946. Three cabinet ministers met with Indian National Congress and Muslim League leaders to find a mechanism to subdivide India without creating a partition. Under the three tiered proposal, princely states would remain undivided, one group comprised of largely Muslim majority would include all of Punjab, Northwest Frontier, all of Bengal and Assam. Third group will be rest of the land mass being the Central Government overseeing itself and the first two groups as well. 

Jinnah was ready to go along if Union had nominal role and groups had freedom to secede later if they wanted. Congress was also willing to keep the country intact. They wanted more central powers and inclusion of Northwest Frontier and Assam as part of the Center. After several permutations and combinations, demands and rejections between Congress and Muslim League, 2nd group-Larger Pakistan would exist as part of undivided India. On May 16, 1946 a plan was proposed where the groups would exist within one country but they could reconsider their status, including one to secede after ten years. 

On June 6, 1946 Jinnah, leader of the League, agreed to this proposal and seeing it as an initial step toward formation of independent Pakistan. He saw in it the plan of compulsory grouping and right of secession. British offered assurances to Congress by including confusing language in the plan. Congress also accepted the proposal. Toward the end of 1946, an interim Government was formed which included Nehru, Patel representing Congress and Liaqat Ali representing League. Jinnah opted to sit out. He knew but kept a secret about him having advanced lung cancer. 
When this plan was announced, other groups including Sikhs, Rashtriya Swayamsevak Sangh (RSS), Muslim League National Guards (MLNG) and others did not trust or like the provisions dealing with Punjab. They started arming themselves as tools for self defense. 
Nehru remarked that in-spite of the wording of the May 16 plan, the Central Constituent Assembly will finally decide the issue of provinces and final components of the groups. Jinnah mistrusted Nehru and called for Direct Plan. This caused further mistrust and violence, initially in Bengal and Bihar, later it spread to Punjab. Mistrust and violence increased. Nihang Sikhs sect attracted many Hindus. RSS and MLNG became more active. On 20th February 1947 His Majesty declared that British would leave India no later than June 1948. 

Punjab Governor Khizr preferred unity between Muslims, Hindus and Sikhs. But Khizr and his Hindu, Sikh allies were not acceptable to the League. Muslims opposed rule by Hindus/Sikhs. Governor of Punjab, Khizr resigned on March 2, 1947. It was celebrated as a Victory Day by the League. Master Tara Singh was chosen to lead Sikhs and Hindus. On March 3 he announced observance of Anti-Pakistan day to be held on March 11, 1947. These events led to violence in Amritsar, Lahore, Rawalpindi and other places. During these violent days about 500,000 Hindus and Sikhs moved to safer havens in Eastern Punjab or beyond before August 1947. Pita Ji and Mata Ji were some of those early planners. 

This is when, in his wisdom, advice from his friend Mohinder Singh and in partnership with Keshav Das Thakkar, our Pita Ji, Kundan Lal purchased a house in Sabathu Cantonment and safely parked his precious cargo here in May 1947. Maat Pita Ki Jai.

Proposals for and against partition of Punjab were being discussed. Jinnah offered a separate Sikh state within Pakistan. Gandhi suggested to let Jinnah and Muslim League take leadership of the free undivided India. Gandhi also chose Nehru as President of Congress against the wishes of multiple regional Congress committees who favored Patel. 

The turmoil followed a pencil-drawn line by an ill prepared British lawyer, Sir Cyril Radcliffe. He was under pressure of meeting a deadline. Last Viceroy of India, Lord Mountbatten, representatives of Hindus, Sikhs, and Muslims agreed on the demarcation lines of the partition. It was drawn by the poorly informed and veto-carrying Sir Radcliffe. 

The announcement of the partition of India was made on June 3, 1947. Commencement date for Independent India and newly carved West and East Pakistan was set for August 15, 1947. This date was about one year ahead of the more practical and planned date originally discussed. 
Royal assent is the method by which a monarch formally approves an act of the legislature, under act 1935. The final Royal assent was passed in London on July 18, 1947. This left less than a month for millions of people to digest the news, the ensuing massive upheaval, understand its implications, find solutions hurriedly as they were surrounded by looting, rapes, religious conversions, fires, swords, guns and blood bath. 

Situation was fluid. People were not sure which country they would be assigned to. Lahore was being considered to be part of India but finally was granted to Pakistan. Since Ferozpur was considered being allotted to Pakistan, king of Bikaner threatened to join Pakistan. Both regions eventually joined India. Gurdaspur belonged to Pakistan for two day. It was the land connecting India to Kashmir. Nehru intervened and had it placed in the India column. Irony was that the partition line was officially disclosed 2 days after the Independence Day of 15 August, 1947. This created immense confusion and anger. It became the infamous Radcliffe line.

The date was moved up from 1948 to 1947 because the riots and killings were rapidly increasing. The British did not want to have more blood on their hands. They already had more than enough blood to drown England as a result of their atrocities against Indians over the last two hundred years. 

Affected people did not have enough time to prepare for the traumatic tumultuous forced displacement, maiming and killings. 13 to 14 million persons crossed the hurriedly created line to and from newly created Pakistan. About 1.5 millions were not so lucky. They died by senseless killings. Some train full of massacred, mangled bodies crossed the borders thus generating more revenge. Starvation, dehydration, infectious diseases, fathers killing daughters to prevent rapes or abductions, and women jumping into wells added to the death toll. 

Our family escaped the above because Pita Ji had wisely bought the house in Sabathu ahead of the upheaval and brought all of us to the cool shades of hills. Away from the heat of sun and infectious hatred resulting in incessant violence in Khanewal. 

However, settling in the new land without money, job, shelter and social support was difficult, but certainly better than getting killed in Pakistan. 

Silence had made Kundan, once known to have angry nature in his youth, appear even keeled in good and bad times. No bad time could be worse than the one experienced by the family in 1947 and no good time could be taken for granted as had been shown by forced displacement from Khanewal to Sabathu. In one way the vacation was bad because it did not allow him to bring all the necessary documents nor the precious material objects the family had acquired. He was elated and comforted that his most prized possessions of eight children including the one still in the safety of a warm womb, his wife and her parents, his own father, his sister-Sumitra Narang, her husband, their children, were all safe with him. 

Kundan, along with millions, had been anxiously following the ever changing political and religious events. Conflicting reports were pouring in daily. Various proposals were put forth between the British, Congress, Sikh leaders  and the Muslim League. Riots were happening all around, most of them transient in nature. There were sporadic property damage and killings; few people fled from the risky areas for safer ones. He read about a family in Rawalpindi where the head of the house rounded up 25 girls and women and beheaded them with his own sword. They would rather be dead than get raped or converted by Muslim goondas. 

Deceptively the storm was followed by calm. For a few days there would be no such news. Everything appeared normal. Then the volcano burst and red lava flowed. 

Mohammad Ali Jinnah, as early as November 1946, had started advocating for the religious minorities to begin moving to the safer areas populated by their respective religious majority. Other Muslim and some Sikh leaders were also giving similar advice. However, Mohan Das Karam Chand Gandhi, normally called Mahatma Gandhi was vehemently opposed to the partition in any shape. Therefore he was not recommending such moves. Jawahar Lal Nehru and Congress members also were generally suggesting for people to stay put. 

In these tumultous times one had to decide about the most major dilemma all families were facing. Stay put and risk getting raped, converted, burnt and killed or move and risk losing all the material goods, which were mostly immovable properties in the form of house and land. And start all over again. 

Using his own judgement and thanks to the advice given by his friend, Mohinder Singh, Kundan acquired the house in Sabathu in March 1947. The family was safe, together and cool in the summer vacation home, away from the burning Khanewal. The address was 21 Mount Uniacke, Sapatu (Sabathu). He and his lawyer friend, Mr. Keshav Nath Thakkar, who lived on the same street called Lawyers Street ( Vakilan di Gali) in Khanewal, purchased the house in March 1947 for a total price of Rupees 13,000. The seller was an English army general who knew that some unspeakable thing called partition of India was a real possibility. Being methodical and a thinker who could project future possibilities into the present, Pita Ji’s instinct had convinced him to purchase the property in the hills barely 2 hours walk from Simla, the summer capital of British Empire's jewel possession-India. This move saved the family from becoming part of the millions who had to cross the border under inflammatory and risky environment of getting hurt, killed, raped or separated. 

Much later Sabathu became part of Himachal Pradesh. Ownership document of the property were crucial for Krishna to get Engineering job in Himachal Pradesh. Pita Ji used to say that it was a sheer stroke of luck which benefitted Krishan. 

In May, 1947 the family arrived for the summer vacation away from Khanewal. There was still a dim glimmer of hope that the bubble of partition might settle amicably and the family would go back before schools opened in September. 

Suraj had finished the undergraduate college in May, 1947 and Prem had finished Matriculation. He got the results while he was in Sabathu. 

The advance party of the elder sons was dispatched to Sabathu. Suraj, Prem, and their cousins Om and Swaraj arrived in early May to make sure that Dhani Ram, the caretaker, had prepared the place for the soon-to-descend large family. Supplies such as food, beds, quilts etc. had to be readily available when the complete extended family arrived. The early scouts did what was needed within a matter of days. The quilts were stuffed with cotton at a nearby shop. They purchased food and necessary supplies and stored them. Their entertainment was to go to the post office to pick up mail, go to village Kakkar Hatti to play cricket and swim in the river. They had more energy to spend than the allotted work to consume it. 

Dhani Ram showed them the head of a 20 miles trail that weaved up to Simla. The youngsters would be out of his way as he gave the final touch ups to the house and the warm young blood would find ways to burn off the over abundant energy. 

Off they went, carefree, not knowing where they would stay, what they would eat. In youth you do things first and think of consequences later. The thrill of adventure was enough to overcome any problems if they arose, they thought.

With zest and laughter they embarked on their journey. Soon the skies turned grey and sun was setting behind the majestic  mountain ranges of Simla. Hungry and thirsty, they settled for the first halwaai (sweet maker) shop. Having eaten to their hearts content, they were asked where did they plan to stay. No idea—was the response. Halwaai took pity on the four adventurers and let them sleep on the floor and he locked the door from outside. Stomachs were full yet the sight of sweets late into the night was too tempting to resist. Being proper, Suraj instructed that no one touch the sweets. The mischievous Prem felt that since no one was watching, halwaai would not notice a few missing pieces of barfi, kalakand and laddoos. The tussle went on till they could not keep their tired eyes open. All except Prem. Quietly, he ate a few and saved some for the free breakfast the next morning. 

For this mischief he did get fair share of scolding from Suraj as they ate the goodies in the shade of a tree next day. They washed them down with cool water from a nearby stream brimming with clean bluish water. As it warmed up, they took a bath in the same stream.  They found a cheap hostel to stay for the four days of adventure. 

The gang was not allowed on the Ridge or the Mall road. Those areas were meant exclusively for the ruling white British sahibs and mem sahibs. The Club house had an engraved stone which read—Dogs and Indians Not Allowed. 

The colored Indians were allowed to do business and walk on the lower bazar and other roads on either side of the Mall road. There was a tunnel under the Mall road for the unsightly Indians and the animals lest their sight spoil an English person's perfect day or a romantic evening stroll. A real or perceived look at a mem sahib by a colored Indian was a punishable crime. 

The foursome was not allowed to visit elegant Gaitey theater in the central part of the The Mall. It was an ornate theater meant to seat 130 persons plus three box settings at the rear. The central box was reserved for the Viceroy and his family. That was not unusual because since 1868 Simla had been the summer Capital of India and the Viceroy spent summers in Simla. After independence the theater opened for Indians. It was in this theater that Dolly Luthra (Sachdev)
was the lead character in a Play called Paisa Bolta Hai. She won the best actress award in the All India inter-medical college drama competition in 1965. 

To escape the oppressive heat, the white skin, government officials, the working staff, and essential papers moved to Simla from the main Capital, Calcutta; Kolkata being a tongue twister for the foreigners. Shimla was too, they changed it to Simla. They came to the hills in March and moved back to Calcutta in November. Poor dark Indian coolies carried all the official documents back and forth on their backs and animals to 5000 feet above sea level. Many did not survive the ordeal. 

One of the first railway train, called Howrah Express was started between Calcutta and Delhi in 1866, Kalka segment was added in 1891.

In 1913 Kalka was connected to Simla by a narrow gauge train through 107 tunnels. Again, dispensable Indians created the railway line. Numerous lives were lost in this endeavor. This was called toy train, a name that still sticks today. It was the same year when the winter capital was moved from Calcutta to Delhi. 

This train did not connect Sabathu to Simla. One had to take a bus, taxi or trek as our foursome had done. 

Suraj was particularly lucky, as if this whole trip was pre-ordained for his benefit. Just prior to the summer vacations he had taken an entrance exam for joining the Railways training college at Jamalpur. Four pioneers were strolling and ogling at the occasional gori mem who for some reason might visit the lower bazaar. Suraj spotted an office of Education department. As if pushed by an external force and a magnetic pull from inside, they walked in and were met by a portly, Indian man wearing a turre wali pagdhi, the kind our Dadaji used to wear. Warmly, he asked the purpose of the visit by the youngsters. Suraj told him about the examination he had taken and wondered if he had passed. The gentleman was a kind man. He asked them to sit and started sifting through a pile of papers. His face could not hide a faint smile. In a loving tone he said "I am not supposed to divulge the results. Do you subscribe to Tribune?"
Suraj and others chimed in, "We have barely enough money to buy food. We can’t afford luxurious item as subscription to a newspaper?" 
The man advised Suraj to start getting the Tribune newspaper for the next few days. He also updated the permanent address on file, which had been Khanewal, Multan. Who knew in May 1947 that there is no such thing as permanent address. 21 Mount Uniscke, Sabathu, Punjab address was given because this is where the family would reside till at least the end of summer vacation; if they headed home to Khanewal. The updating of address was also a God-send. Without the Tribune, Suraj would have never found out that he had passed the entrance exam for the prestigious Jamalpur Engineering college. Without the change of address he would have never received the compulsory medical check up documents required prior to admission. He was one of the 17 applicants from the whole country who got selected to this prestigious school. Without the divine or chance intervention, India  would have missed out on a brilliant engineer, who in due course, would become general manager of Northern  Railways. 

Without his job who knows what Suraj might have done and the family would have suffered even worse times financially as they would have missed out on every additional scarce, precious Paisa and hand me down clothes that Suraj sent.  The budding doctor, Gindi would have missed out on the new stethoscope he received as a result of Suraj's official trip to Germany in 1965. 

Return journey to Sabathu was easier than the steep climb. It was downhill physically but uplifting for Suraj’s future. 

Rest of the family joined in batches. All arrived except MataJi's parents who planned to stay back in their permanent home in Sargodha. They hoped that the volcanic eruption was transient. It must have been very difficult to leave behind two houses and all the memories. But they were in for a rude shock when on August 14, 1947 Pakistan celebrated its birth and the India got divided. 

While our family was safe, animal instinct and mob mentality turned friends into savages in both sides of Punjab. 

Based on religion, the country of Pakistan was carved out by a wavy line that cut through Sindh, Gujrat, Rajasthan and the most populated Punjab, (punj meaning 5 and aab meaning water), a land through which five major rivers traversed. This affected the western India and Bengal suffered the same fate in the east.  Pakistan, being a land for Muslims, left Hindus, Sikhs, Christians and others with a choice between leaving their permanent address , become refugees till they got replanted in secular India or get converted to Islam or killed by staying in their rightfully owned permanent home. 

Many Muslims in India chose to move to their sought after land and some were forced to leave by the influx of new home and revenge seeking refugees. What was meant to be a peaceful transition, turned into a bloody mayhem.

Over 1.5 millions died and 13 to 14 million people migrated between the two countries—India and Pakistan. The latter was comprised of West and East Pakistan with vast India separating them. 

The majority of displacement happened on foot, bullock carts, trucks, camel backs, horses, donkeys, mules, elephants and took several days in the oppressive and sweltering summer heat of North India. Some of the well-to-do families, those with political or military connections or simply lucky ones crossed the border by military trucks, cars or planes.  Many traveled by buses and trains, some of which transported passengers to safety while others were not so lucky. Some trains were stopped randomly by angry mobsters on both sides. With slogans of Allah U Akbar, Har Har Mahadev, Jo Bole So Nihal, occupants were killed with swords, guns, stones and children with sheer hands. The trains were then sent to their destination filled with mangled bloodied dead bodies inside the compartments and some on the roof tops. Such sights in turn fueled and generated more hatred resulting in similar atrocities inflicted by the other side on the innocent participants of the fire created by circumstances beyond their control.

A few politicians whose ego and lust for power knew no bounds and did not foresee or care for the ordinary people living a peaceful life. Both sides claimed the killings and crimes against men, women and children as self defense or retaliation against atrocities committed against their families on the other side. 

Fortunately our family escaped the violent crossing. We had minimal belongings but were safe, together. Maat Pita Ki Jai. 

 My mother was Vidya Wati Luthra (Virmani). She was born in Samundri, Punjab, now is Pakistan. Vidya means knowledge. She was not formally educated beyond 5th grade. There was no middle school in Samundri and girls were not allowed to go to school outside their hometown. Inspite of this handicap Vidya was an encyclopedia of worldly knowledge on every subject, particularly human psychology. She had seen it all, lived it all, dealt with all kinds of people and circumstances. She kept meticulous mental notes. Carefully she passed on this knowledge to those who listened and many times she put her thoughts and practical solutions in the form of well written, to-the-point letters always filled with a pinch of humor. She knew only Hindi script. She used Punjabi as spoken language but could understand Hindi, at times speak if needed. Later Arati tried to teach her few English words. 
Vidya Wati, the young pampered child was not allowed to touch her soft feet on the hot sands of Samundari and later Peshawar and Sargodha, had her own personal ghorhi (mare) provided by her well-to- 
do parents called Beji—Kesar Bai and Lala Ji—Mathra Das Virmani. Mathra Das was born in 1870s in Khewra, near Salt miles around Jhelum. Mathra Das died in the old house in Panipat in 1948. 

His brother, Bhagwan Das was father of Chuni Lal who was Vidya’s first cousin brother. Since Vidya did not have a real brother, she started tying Rakhi to Chuni Lal. That’s how Chuni Lal became our loving Mama Ji. He and his children used to visit us in Panipat and we visited them in Delhi often. Bhagwan Das was lost during the Partition. Somehow he managed to escape the wrath of Muslims and made a surprise arrival at Chuni Lal’s home, much to everyone’s delight. 

Vidya was born on January 1, 1912 in Samundari, near Karachi. She was a late child. She told us that as a teenager she used to hold actor Raj Kapoor in her lap. Raj Kapoor was born on December 4, 1924 in Peshawar. Vidya was close to 13 years old. Apparently Mathra Das was transferred from Samundari to Peshawar between 1912 to 1924. He was a Quango in the revenue department. The head of province was Nazim (Governor). Kardar was in charge of collecting revenues in several pragnas. Quango was in charge of revenue collections in each pragna. He travelled from village to village on a horse to collect revenues from the Patwaris who in turn had it collected from the farmers. There were several villages under the supervision of Qanungo. He kept meticulous records. Some farmers sent vegetables and fruit to his home. They hoped that this gesture might reduce their taxes. From Peshawar the family shifted to and retired in a newly developed and planned town of Sargodha. We don’t have any photograph of Mathra Das. We do have one of his brother Bhagwan Das, father of Chuni Lal, our Mama Ji. From that we can get some idea how our Nana looked and how handsome he must have been. 

 
Bhagwan Das, our nana Ji’s brother

Sargodha, City of eagles was established by British as a canal colony in 1903. In order to populate canal colonies, British Raj was offering grants of land to those who would build a house there. 
Sar means lake and Godha means Sadhu. A lake was in the middle of old town where a Sadhu lived in isolation, leading to it being named Sargodha. My maternal grandparent, Nana Ji and Nani Ji had a house in the old town. Later new developments evolved at which time they built a second house on plot number 6. 
This is where Nana Ji retired. It is about 200 miles northwest of Khanewal. Vidya got married to Kundan Lal Luthra here on January 22, 1927. 

After marriage she moved to Thatti. My father, Kundan Lal, with his family lived there. 
His father, Lala Gokal Chand owned land for farming, for development and two brick kilns on the outskirts of the town. He was helped by his two sons. It was Thatti where Karam Chand died along with dreams of Kundan Lal becoming a doctor. 

Initially Lala Ji worked for British government. Later he became land surveyor, a farmer and developer. New developments were cropping up in fertile part of Punjab to accommodate large number of migrants coming in from eastern Punjab.

Lala Gokal Chand was married twice. They had one daughter from the first marriage—Tulsa. His first wife died young. Then he got married to Raj Karni, daughter of Radha Krishan. She was very beautiful and looked like a porcelain doll. She also died at a young age on July 3, 1926. From the second marriage they had two sons—Karam Chand and Kundan Lal and two daughters—Sumitra and Sheila. Karam Chand died of an infectious disease on 30th August 1925. 

Sumitra married Dr. Ram Lal Narang who later settled in Ludhiana. He had a medical clinic in their house on GT Road. Their children were Ved, Bhupinder, Usha, Ravinder and Anil. We used to spend summer vacations together, either in Panipat or in Ludhiana. My nick name is Gindi. Anil Narang, a year younger to me was called Chota ( Younger) Gindi. 

Sheila married Anand Sagar Khera. Their children were Anjali, Nimmi, Vinay and Madhu. They settled in Panipat and owned Bharat Carpet company. 

Lala Ji, Gokal Chand Luthra was born in 1875. Our Luthra clan belonged to Sahiwal, sometimes gets called Saiwal. This region lies between Sutlej and Ravi Rivers and located between Lahore and Multan. Lala Ji’s mother was very pretty. Suraj used to help her, get desi medicines (Arak) for her and was her favorite grand child. She died in 1943, the same year when Suraj’s matriculation results were declared. Lala Ji had greenish eyes whereas his mother’s were sky blue. He was son of Chaudhri Ram who died on September 17, 1919. His wife, Soalakhan died in 1906, the same year Kundan Lal was born on June 6, 1906. Chaudhry Ram was son of Ram Singh. Ram Singh is the only one in our genealogy with the last name of Singh. It was not compulsory, but in many Hindu families the first son became a Sikh. This tradition was created to have sufficient Singhs (Lions)to protect Hindus and Sikhs from the atrocities of Mughals, especially Aurangzeb. Ram Singh must have read Guru Granth Sahib, Sikh’s holy Book and the last Guru. Gokal Chand read Granth Sahib everyday. When family came to Sabathu, Granth Sahib was left behind. Once someone from India had a chance to go back to Khanewal. He asked Pita Ji if he wanted something from the house. He asked for only his father’s beloved Granth Sahib. This was placed in Lala Ji’s room at 2 Model Town. Back door was always open and anyone could come and read the Granth Sahib. 

British officers, in their plan to develop Khanewal, had given a contract to Lala Ji for building roads around it. He must have done the job well. As a result, Lala Ji was awarded 10 acres of land in Khanewal. He acquired extra land in Mian Channu as well where wheat was grown by tenants. 

Our family moved to the developing Khanewal. 
It was a small town. New railway line was built to connect Multan with Lahore. The railway track passed through Khanewal. Khanewal was called Khanewalah at that time. Being a small town, the railway department decided to build a flag station here. Flag stations are smaller stations along a railway line where the train stops only if someone waves a flag. Later Khanewal was connected by train to Layalpur, and Karachi. It was later designated Khanewal Junction. 

Lala Ji, and Pita Ji did farming on the allotted land. Main produce were potatoes and cantaloupes. Pita Ji used to go to Patna in Bihar to get the seeds. Their monthly income was about 1,000 rupees a month. They were very well-to-do in those days. The currency was Rupee, Athani (half rupee), chavanni ( quarter rupee), duwani (one eighth rupee), Anna (one sixteenth rupee), taka (one third two rupee), paisa (one hundredth rupee), duels(half paisa), pai (one third paisa). For reference an imported cloth hanging pin cost one paisa in 1930s. 

They had a house in the old town and later built a house in Block # 8 on the outskirts of Khanewal. The house was 60 foot deep and 80 feet wide. The main entrance was from Badhi Gali (Large Street). Entrance led to an office with a bathroom to its right. Behind it was a room to store hay and one was Lala Ji’s room. There was an almost 50 feet wide court yard. Along the badhi gali, from right to left, there was a kitchen, storage room-trunk room to store quilts (razaais), and three rooms. Along the left wall also there were three rooms. It was common for men folks to eat food on the roof top. 
On the outside of the right wall there was a rope attached to a basket on one end. A pulley at the edge was used to have the other end tied to a post on roof top. Prepared food was sent up in the basket and empty plates sent down. 


Life was moving along smoothly. Prosperity brought more dreams. Pita Ji and some friends planned to develop a township on the outskirts of Khanewal. He bought six plots of land, one for each of his sons, one was yet to enter the world. The town was going to be named Prem Nagar. Alas, it never came to fruition due to Partition. 

Suraj was born at Nana Nani’s home in Sargodha on January 1, 1928. Mahinder was born in 1929 and Prem on June 1, 1931 in Thatti. 

Between Kunan Lal Luthra and Vidya Wati (Virmani) had a total of ten children. Sudesh was born in 1939. Mohinder around age of six years, and Sudesh around one year succumbed to small pox. Miraculously, others survived the poverty, small pox, malaria, cholera, other  infectious diseases along with random and planned acts of violence. Kanta, Kanchan, Virinder, Krishan and Juginder were born in Khanewal. MataJi was pregnant with Ashok who was born in Sabathu on August 28, 1947, thirteen days after partition. He is the only one of the ten who was born in free India. 

Beji and Lala Ji were two of the millions of people who made the journey across the border after the partition. They were rich and influential enough to fly across the burning border. Camps were created at several locations between the Punjab border and the Capital, Delhi. One of the main camps was in Ambala. 

Government machinery was amazing and meticulous, given the inexperience to handle this unique problem and magnitude of the tsunami of displaced migrants. Tents were raised, registers were created to keep track of the migrants for official purposes as well as to reconnect the missing family members. 

Beji and Lala Ji were finally convinced that there was no turning back to the permanent address. Therefore she wore all the jewelary on her body, in several layers and tied them with a cord around her neck, he stuffed the cash in his pockets and small bags and flew to India in a plane. 

She had brought a metal trunk filled with Crown bearing coins to distribute to the poor on the occasion of the future wedding of their first grandson, Suraj. The tradition was to throw coins on the Sehra-covered groom as the baraat moved to the bride's home. The poor and not so poor neighbors, young children and some not so young, lined up on both sides of the street as the  baraat slowly and haltingly proceed in to ed. The mare carrying the groom was made to stop, the band played loud movie tunes, family members and friends danced ahead of the groom riding the horse. During some such stops, Beji had made plans to shower their first grand son with the precious coins. 

Since their only daughter had left the house, the empty nesters Beji and Lala Ji asked Vidya and Kundan to give them their first son. As might have been the custom, the new parents agreed and Suraj became unofficially adopted by his Nani and Nana as a son. Suraj had moved to their home in Sargodha when he was two years old, was raised by them and used to call Mata Ji as Vidya. He was usually loaded with money, thanks to the generosity of Nani and Nana. In one of the family meetings, Vidya gave an opinion. Suraj interjected “Vidya, there are only men here. You don’t belong here.”

After partition, at the age of 35 Vidya was suddenly thrown into the role of a cook, washer woman, cleaning lady, a mother, a wife, daughter, daughter-in-law, protector, disciplinarian, responsible for everything because the man of the house was rarely home. He was struggling to procure whatever he could manage to shelter and feed his wife, father, in-laws, and the hungry bunch of six children. Oldest, Suraj had left home to join Jamalpur Railways College and next one, Prem had gone to Ludhiana to complete Bachelor of Arts (B A). 

Simultaneously he was busy reclaiming property and monetary help in exchange for the property abandoned in what was now Pakistan. The district was Karnal, 21 miles north of Panipat. Indians had left behind 6.7 million acres of land compared to Muslims abandoning 4.7 million acres in India. Proportionately less land was allotted to Hindus and Sikhs compared to what they were forced to abandon. Through Pita Ji’s efforts, he was able to received some land about 20 miles north of Panipat and two plots of land in Delhi, one of the in Lahore Pat Nagar. These two plots were sold to cover the expenses of Kanta and Kanchan’s marriages. 


Between Kunan Lal Luthra and Vidya Wati (Virmani) they had a total of ten children. Sudesh, around age of one year and Mahinder around six years, had succumbed to small pox. Miraculously, others survived the poverty, small pox, malaria, cholera, other  infectious diseases along with random and planned acts of violence.

Vidya, literally means knowledge. She was not formally educated beyond 5th grade, There was no middle school in Samundri, Peshawar or Sargodha, the girls were not allowed to study outside their hometown. Inspite of this handicap Vidya was an encyclopedia of worldly knowledge on every subject, particularly human psychology. She had seen it all, lived it all, dealt with all kinds of people and circumstances. She kept meticulous mental notes. Carefully she passed on this knowledge to those who listened and many times she put her thoughts and practical solutions in the form of well written, to the point letters always filled with a pinch of humor.

Beji loaded with jewelry  and Lalji stuffed with whatever money he could carry along with their precious  trunk flew in a plane to India and somehow made it to the camp in Ambala. They had sold their two houses in distress sales and brought the proceeds with them before leaving their part of India what had now become Pakistan. 

Pitaji's sister, Sumitra was married to Dr. Ram Lal Narang. He was very fond of and protective of his sister and her family. Their whole family had accompanied Vidya/Kundan to Sabathu for the summer vacation. Except for Dr. Narang, who stayed back in Pakistan to take care of patients and medical practice, and mistakenly hoped for peace to return. 

Although naive in hindsight, most believed that the talk of partition was simply a posturing by the politicians to gain points and increased power in the united Indian Government. Gandhi was firmly against partition because he could not bear to visualize a knife going through his beloved India. That knife would go through millions of innocent peace loving people. 

Dr. Narang's eldest son, Ved was en-route to Sabathu via Ambala to join the family. On arrival he told Suraj that he might have had a glimpse of Beji and Lala Ji in the Ambala  camp. Suraj could not believe that he might find his Nani Nana and adopted parents. They immediately turned around and traveled by taxi and then a train to Ambala. The camp was studded with thousands of quickly made tents, some formal, some just layers of clothes hung on bamboo posts to demarcate one's precious possession of space on the land of scarcity. They ran to first visible desk. Holding their breath, they inquired from the clerks managing the registers but found no entries by Mathra Das Virmani and Kesar Bai Virmani. They frantically started manual search from one tent to the next. Lo and behold, they could not believe their eyes. There they were, sitting together, holding hands under a tree on top of their prized trunk. They were looking vacantly into space, wondering what would they do next, how would they connect with loved ones who were hopefully safe in Sabathu.

Ved and Suraj ran to meet Beji and Lalaji. Even though they were tired to the depth of their bones, their eyes lit up and they jumped with joy to meet, greet and hug one another. Tears of joy mingled with traumatic experiences of getting thrown out of their homes. Now at last the whole family was together and safe. They took the next available bus to Sabathu. 

Life may start with a vacation but does not always end with it. It has to go on, adjusting and adapting to planned and unexpected circumstances. The one half of the house in Sabathu was too small for the large immediate and extended family. There were no jobs, no schools or other amenities necessary to raise a family. Overnight life for millions including ours was transformed. 
        
                 Kundan was 41 years old at the time of partition. Nothing had prepared him to manage the magnitude and the number of problems that needed assessed, analyzed and solved. All that he knew in the way of a profession was to be a farmer and brick maker. His father, Lala Gokal Chand Luthra was land surveyor and a contractor. When British government decided to develop Khanewal, District Multan, he was given the contract to build a road around the town. He was a perfectionist and honest man. Being happy with his work, the government allotted him 10 acres of land. 

This land was cultivated with the help of the two sons, Karam Chand and his younger brother Kundan Lal along with hired labor. He also had land in Mian Channu, a village 30 miles north east of Khanewal, where they grew wheat. In Khanewal all types of vegetables, primarily potatoes and cantaloupe were grown and sold to make a decent living. Business was good. Even though the family was not loaded but had a very comfortable living standard. Kanchan once told me that the income was about 1000 rupees a month. A huge sum in those days considering that the House 2 Number Model Town was bought for 14,000 rupees. 

Kundan Lal Luthra was born on June 6, 1906 in Thatti. It is located 61 miles east of Karachi, about 120 miles where his future wife was born. Looking at the genealogy records in Hardwar, our Arora clan is called Luthde Sahiwal ke (लूथड़े साहिवाल के). 
Sahiwal is a located between Rivers Sutlej and Ravi between Lahore and Multan. 

There were two main clans of Luthras, one from Thatti, later called Thatta and the other from Sahiwal. Apparently there was migration of families within the region. Sahiwal was named Montgomery by British Lieutenant General of Punjab in 1865. During Ayub Khan’s regime the name was changed back to Sahiwal in 1967. As children, we used to hear the word Montgomery many times, but none of members of my generation sat down with Mata Ji or Pita Ji to ask anything about this or their early lives. I am trying to reconstruct all this from bits and pieces heard from them, our family and supplemented by internet. 
Khanewal bordered Sahiwal on east and Jhang on west. Our language had several words of Jhang Punjabi. 

Kundan Lal went to school in Shahpur, on the bank of River Jhelum. It is about 23 miles southwest of Sargodha. 

As a child he was a lean boy and once was carried up in the air by one of the frequent dust storms. Luckily there was no bodily harm when he landed about 10 feet away. 

After matriculation he joined DAV (Dayanand Anglo-Vedic) College in Lahore in 1921 at the age of 15. He planned to become a doctor and therefore joined the 2 years course of Pre-Medical. He planned to join King Edward Medical College, Lahore in 1923. 

For no fault of someone, events outside can change lives of innocent bystanders--collateral damage. Following Jallianwala Bagh massacre the quest for freedom of India from the British Raj was gaining momentum. Mahatma Gandhi’s call for Non-cooperation spread all over India. It was causing concerns to the British all over the country. 
(Reference page 299, Punjab by Raj Mohan Gandhi)However, Punjab was involved more for collaboration between Hindus and Muslims than non-cooperation against the Raj. A Unionist party was created by Punjab government in 1923. Main leaders were Fazl-i-Husain, a Raj leaning Minister in the government. He promoted Muslims interests in education and civil jobs. The other was Chotu Ram from Rohtak representing Hindu Jats of peasants from eastern Punjab. 
Fazl, in November 1921 helped increase the seats allocated to Muslims in Government College and Medical College to 40%. A ratio of 2:2:1 was created. 2 for Muslims, 2 for Hindus and 1 for Sikhs. This number far exceeded the 15.2% share of seats taken by Muslims in 1917-1918. 

This increase in the reserved seats for Muslims directly affected Kundan Lal who had sufficient marks to qualify for the admission. He, among several other Hindus, did not get seats which were given to Muslim students with lesser marks. 
Not to give up easily, Kundan joined Bachelor of Scinces, Medical to improve his chances of admission next year. 
All was progressing as planned. But on August 30, 1925 his elder brother, Karam Chand died of an infectious disease. Their father, age 50 at this time was unable to handle the Government job and the farming. Karam had planned to take over the farming. His sudden death was a big emotional blow to Kundan. He got K C tattooed on his right forearm, keeping his brother in sight through out his life. We never asked and he never told us anything about his brother. It was also a blow to his career. He was asked to abandon the idea of becoming a doctor. Instead he became a full time farmer. 

Ironically Pita Ji’s birth year was the same when an organization was formed which forty one years later would affect his life along with millions of others. All India Muslims League was formed in 1906 in Dacca whose influence and demands resulted in partition of India and creation of Pakistan. Even though Muslims, Hindus and Sikhs lived together, had common friends, although rarely visited one another’s homes. Many of them proved to be real who saw the person and not the covering of religion, yet many more did not trust each other. It was commonly said that Roti and Beti are not common; otherwise they had cordial or even friendly relation. Eating at Muslim homes was not commonly done by Hindus and Sikhs. If Muslims did eat at Hindu or Sikh’s home, there were separate plates and steel glasses for them. Inter racial marriages were absolute taboos. 

Kundan’s fair skin turned darker from the bright sun of the deserts of  Khanewal fields despite the khaki hard hat. The fair skin and good features were inherited from both his father and mother.  
After a tumultuous life fro almost becoming a doctor to almost becoming developer of Prem Nagar, he became the provider, educationist and protector of the large family at the age of 41 years, he finally retired at the age of 56 in 1962. 

Around 1967 while visiting Shashi and Prem he felt unwell. “I have digestion problem.” He had almost non-stop belching for one day. Dr. V. Chaddha diagnosed a cardiac problem. “This cannot be true. I worked all my life and am too strong to have a heart problem at age 61.”
Pita Ji insisted that the diagnosis was not correct. An ECG by Dr. Mookerjee confirmed it was a silent heart attack. Pita Ji wanted to rush back to play cards with his buddies, but now was bound to rest for four weeks and started taking two medicines. Finally Mata Ji and Pita Ji went back. Prem snd Shashi hired him with numerous packs of Cards. He gave most to his friends as gifts. He found more joy in giving than receiving. He was very good Sweep and Rang Mallan cards player and took it seriously, sometimes getting angry if partner made foolish mistake. 
Later in life he became vegetarian except for fish. 
He loved eating onions and was a fast eater. 

His father, Gokal Chand had  fair skin, light grey eyes which genes carried over to Prem, Kanta, Neeru, Rohin and Rashmi. His mother’s blue eyes had full expression in Priyanka. He had an erect posture making him look very elegant. He was very well dressed and fond of wearing white washed clothes. He used to wear a white turban with a couple of feet long end hanging over his back. 
 
Gokal Chand's mother was very beautiful, and also had blue eyes. She was very fond of their first great grand son, Suraj. She died on the same day when Suraj's matriculations results were declared in 1943. Gokal had one daughter, Tulsa from his first wife who unfortunately died at a young age. Kundan's mother was Gokal's  second wife, Raj Karni, a beautiful woman, and was daughter of Radha Krishan. She died on July 13, 1926. Death of her son less than a year before that must have broken the mother's heart. His siblings were Karam, Sheila who got married to Anand Sagar Khera and lived in Panipat and Sumitra who was married to Dr. Ram Lal Narang and got settled in Ludhiana. 

In 1958 Sumitra had gone to Delhi for conveying condolences to a family member. One day she was trying to get into a bus, the bus moved too soon causing her to fall. She sustained head injury and went into coma. She never came out of it. She was admitted to Ganga Ram hospital, one of many institutes erected by this philanthropist. Mataji also was in Delhi. Nishi was two years old. Urmil, only 25 years old at that time, left Nishi at home and spent the night with Sumitra and was the only person with her. Around midnight Sumitra passed away. A young family of four sons and a daughter was left behind. They were similar in age as Kundan's children. 

Gokal was a practicing Hindu but also read Guru Granth Sahib everyday. After partition when every material object was left behind, a Muslim friend was going to make a visit to Khanewal. He asked Kundan if there was anything Kundan wanted and brought back from there. Kundan asked for only one thing and that was his father's treasured Guru Granth Sahib. The friend did follow through with the request. Guru Granth Sahib, the 11th and the final non human Guru was placed in the smaller room on the right side of the house as seen from the street. That side had its own gate and the room had a door facing the street. It was well known in Model Town that the doors were always open for anyone to walk in and read Granth Sahib any time of the day. The devotees did not need to inform or leave any donations. During that time frame in the early 50s all the construction was focused on housing. The houses of worship such as Gurdwaras and Temples came much later after the basic living spaces had been completed. There were several Temples and Gurdwaras made in Model Town but mosques were limited to the old city where many muslims continued to live and some returned from Pakistan after the initial firestorm had subsided.  

In his later years Gokal Chand was very fond of playing cards and teaching his grand children popular card games called Sweep and Dusehri. He kept a cow whose fresh milk was a treat, raw straight as a stream from the cow to his mouth or boiled with lot of malaai. He taught his grand children how to milk the cow. They would squat down, hold the washed and clean bucket between the thighs and squeeze the milk into it. Sometimes one needed to tie the hind legs for fear of getting kicked. One had to watch getting hit by the tail also.  Rumors were that he talked about the benefits of drinking cow's urine though no one had physically seen him do that.

He always wanted freshly made rotis and vegetables and when approaching the kitchen he used to make a coughing sound to alert the daughter in law to cover her face. Even though it was not universally practiced by Hindus in the changing times, Gokal was still unable to change the Parda system in his house. Vidya generally wore a white or beige sari. On hearing toe pretend cough of sound of the walking stick, she promptly covered her head and face with its end hanging a bit longer on the side from which her father in law was approaching. She did not follow this system any where or with any one else. No other female in the house followed the Parda system. 

 During December 1954 winter break from college in Ludhiana, Virinder was in Panipat. Prem and Suraj had also come. Lala Ji loved playing sweep with them with full enthusiasm. A winning card was thrown down with force. “Bachchu hun ki karen ga!” ( Hey lad, what can you do now!) He was always last to leave the table. One day he left the table saying he was not feeling well and lied down. Children didn’t think much of it and returned to their respective places. Pita Ji sensed something was not right with his father. He offered to sleep on the floor in the small room used by Lala Ji. On January 8, 1955 Lala Ji read his Gutka, a small Sikh scripture book and went to sleep. Lala Ji around mid night went to bathroom and came back to bed. I heard the loudest shriek at 2 Model Town Panipat on January 8, 1955. Kundan had gotten up to check on his father. The cold body told the son that his father had died in his sleep at the age of 80. The funeral services were done at home. In fact Lala Gokal Chand's life was celebrated. A halwaai (sweet maker) was hired to make food for the visiting relatives but also anyone from the street were welcome to join in the sumptuous feast that was served. Such was the custom when an elder lived to ripe old age. Average life expectancy in India from birth in 1950 was about 35 years. Surely Lala Ji’s 80 years of full life was an event to celebrate. A clay water pitcher was broken, to symbolize end of this physical life. Covered with a white sheet, the body was laid out on a cot. With family and friends chanting " Ram Nam satya hai" the all male funeral procession carried him across the railway line to the only shamshan ghat (Cremation Ground)in Panipat. After several days the ashes were taken to Hardwar. 

 


A unique system is followed in Hardwar where Hindus bring the ashes of the departed relatives to be immersed in the holy Ganga, sometimes referred to as Ganges by English speaking persons. After prayers performed by a priest, the ashes are place in  a floating lamp and placed on the holy waters. Most clans have their assigned 'Pandas' or priests. The Pandas maintain, preserve the books and pass them onto next generation as the old keepers pass on. The person bringing the ashes may make an entry in the clan's red covered book about the name of the person whose ashes were brought and details updates about the rest of the family. 

 
 
Juginder Luthra in July 2016 with our family Panda in July 2016. He, Dolly and her sister, Ambika brought ashes of Dr. Sushil Kumar Sachdev. 


One such entry was made on September 17, 1919 by Lala Chaman Lal, one of the four sons, for Sh. (short for Shri, akin to Mr.) Chaudhry Ram’s ashes. L. Chaman Lal returned in 1921 for his wife and again on August 30, 1925 for Karam Chand, Kundan's elder brother. In return, Lala Kundan Lal visited on September 30, 1932 for Lala Chaman Lal.  Some entries have been made during non death related visits. For example an entry was made by Kundan Lal... Came here with family to take bath in the holy Ganga. This was dated 13.8.40. It was signed— Kundan Lal s/o Gokal Chand s/o Chaudhry Ram s/o Ram Singh. 
This shows his hand writing at the age of 37 years. He combined his first and middle name. 
 

Another one was made by him saying---I came here today the 28th Aug. 1944 and took bath of the holy Ganges. It was signed Kundan Lal Luthra. One earlier entry stated...Gokal Chand visited Sukh Sanan in 1897.

From these entries one may surmise why Gokal Chand would read Granth Sahib. Gokal's  father's name was Chaudhri Ram who had died on around first week of September and his ashes were immersed on 13th September, 1919.

To combat the atrocities by the Mughal emperors startin with Jahangir and especially the last one-- Aurangzeb on Hindus and Sikhs, on April 13, 1699 the 10th Sikh Guru named Guru Gobind Singh symbolized Khalsa as a group of warriors to fight against the oppressors. They were identified by the 5 Ks...Kach, Karha, Kes, Kirpan and Kangha. They vowed to fight to the best of their abilities and protect the Hindus and Sikhs from muslims. The tradition was that the eldest son of every willing Hindu family would become a Sikh. That person's children generally but not always continued to follow the teachings started by the first Sikh Guru, Guru Nanak Dev who, by many records, was born on April 15, 1469. Since 1868 his birth is celebrated on the full moon day in November. It may be that day was chosen to celebrate the day of his enlightenment. 

Ten Gurus followed. The 5th Guru, Arjan Dev, compiles the prior Guru’s teachings in a holy book called Guru Granth Sahib. The last mortal Guru Gobind Singh pronounced the holy scripture that carried the teachings of Guru Nanak plus the ones added to by the following Gurus as the final and eternal Guru. This holy book is Guru Granth Sahib. 

Male Sikhs were also identified by using Singh, meaning lion, and the female Sikhs used Kaur, meaning princess, as their middle or last  name as appropriate. The tradition continued even after the death of Aurangzeb in 1707 and the ensuing fall of the Mughal dynasty.  

Chaudhri Ram's parent's were Ram Singh and Soalakhn. Soalakhn died in 1906, the same year when their great grandson, Kundan was born.  Ram Singh must have been the eldest of the four sons of Moti Ram and must have become a Sikh as inferred by Singh after his first name. His other brothers were Duni Chand, Ludhinda Mal, and Ganga Ram, all Hindus as suggested by their names. Moti Ram and his brothers, Ganesh Das and Sewa Ram were children of Surta Ram, again all Hindu names. Ram Singh must have been a devout Sikh and would have started the tradition of keeping and reading Granth Sahib that carried on through to Gokal Chand. None of Ram Singh’s children carried the last name of Singh.  Through that very red  book the genealogy of 12 generations up to Kundan's children was documented as seen in Appendix 1. Names of the males and sometimes their wive's or mother's name was entered. The tradition of visits to Hardwar continues till today and the family's status is updated. Currently the names of female members are also entered. 

At the age of 41, with no job, no formal professional  education and burden of the family of current, previous and the next generation, it must have taken a man of steel with the support of a strong and wise wife, who was merely 35 years old, to keep it all together. Life as had been known to them for generations vanished with a stroke a pen in Delhi. There was no time to grieve the losses. It was time to charter a future filled with uncertainties at every level. They had no clue where to start and the list to do things was never ending. 

First thing first. Arrange for the perpetually pregnant Mata Ji to have a safe delivery in the new place. Thirteen days after the Independence day the last of Vidya and Kundan's children, Ashok Kumar, was born at home in Sabathu on August 28, 1947. For lack of baby clothes Ashok, later fondly called Shoki wore Juginder’s (Gindi) shirt. This was obviously too big for the baby boy but at he was covered. There was no time or resources to have any more children. Eight is enough, was the by product of the partition. 

Gokal's brother, doctor Gopi Chand was also displaced from Pakistan. He was on his way to Delhi through Panipat. There he learnt about and checked out a large, rambling old house without electricity, recently vacated by a Muslim family. He immediately occupied it and sent a message to Kundan "You have a large family to shelter. This house will be perfect for you. But you need to hurry up before someone stronger than me throws me out and take its possession." As soon as Kundan received the message he, along with the grown up children rushed to Panipat and took over the house in early 1948. The rest of the family followed soon thereafter. The house was deep into the city, beyond Qalandar Chowk on a  narrow street with open sewers in the middle.  No one minded or even noticed the sights and smells. They were thankful that they had roof on their heads, a roof that had recently sheltered a Muslim family who would have become a refugee, migrating and looking for a roof vacated by a Hindu or a Sikh in newly born Pakistan. They still needed to guard the house against anyone else coming and throwing them out. Animal mentality of survival at any cost was prevalent. Each one wanted to hang on to the neighbor and friends’ shoulder for support yet they did not know who to trust. This was our home from 1948 to 1950. The house had no electricity. One room had no sunlight as well and was dark. It must have lot of mold. 

Even though Mata Ji never had any indication of asthma, but in this house she had first and the last one. It was almost fatal. She started wheezing, breathing kept getting harder and harder. Nails were turning blue. Nani Ji started beating her chest and crying loudly. “ Hey Vidya enni jaldi chorh ke ja rai win.” Luckily Virinder was home. He was only 12 years old but knew the urgency of the situation. He ran to the market asking everyone if there was a doctor nearby. Hod was kind and someone called Dr. Hans Raj was there and said “ Yes I am one.” Virinder quickly told him about the emergency. Dr. Hans Raj immediately came to the house, gave an injection to Mata Ji and she started recovering. An almost fatal event was averted. She stopped going to the dark room and never had the episode happen again. 


After handing over the house to Kundan and family,  Dr. Gopi Chand with his family ended up settling in Abohar where he established a reputable medical practice. 

The family started making lists of material possessions that they had left behind. That included the  house in Block # 8 in Khanewal. It was a large house measuring 80 feet wide and 60 feet deep. It was approached by a large street ( Barhi gali) for the use of family and friends and a small street (Choti gali) used for laborers to enter. The main entrance was on the right side of the house leading to an office, a bathroom with Lala Ji's room to the left of the foyer. A toorhi room to store hay was the adjoining room. Nearby steps lead to the roof. A rope was tied at one end on the roof top and a tokri (container made of cane) was attached at the other end. This end used to be lowered to have someone in the kitchen fill the basket and send roti, vegetables, rice etc. up to the roof where the family would enjoy a delicious variety of food. The food had recipes from the Mugals with significant Afgani touch.  A very large court yard in the center was flanked at the back by covered winter kitchen. In the summers cooking was done on the outside chulha. Storage room with a large trunk stored the cotton filled quilt--razaais, was next to the kitchen. Additional 6 rooms lined up along the back and the left walls. 

In addition to the house, claim was made for the 10 acres of land on the outskirts of Khanewal. Lala Ji and Pita Ji had used this land for agriculture. They had made plans to develop a residential colony on it. The land had been surveyed and approval was pending. They planned to call it Prem Nagar in honor of their second son who was appropriately, as coming years would unfold, named Prem Prakash. Prem means love and Prakash means light. Born on June 1, 1931. Prem continued to love life and spread light of love to all those around him including friends and family. Even a stranger passing by or co-traveler would receive more love from Prem than they would receive elsewhere in months. The Prem Nagar colony would have been wonderful place for people to live in but the circumstances left it on the drawing board and now on the list of claims to be made for appropriation of equivalent land or money in the new India. Lala Ji  had another piece of land in a neighboring village, Mian Channu. 

These three items made the bulk  of abandoned and lost properties. Additional items including beds, chairs, tables, utensils, clothes, money in the bank and everything else in the house down to the small spoons with approximate value were enlisted. 

The lists were then submitted to the Claims office in Karnal, the district in which the small town of Panipat was one of the tehsils. The court system was inundated with millions of such and other claims. The claims had to be verified and matched for  their validity against the court documents in Pakistan, affidavits and personal judgment by the judges. Just by the sheer numbers and the general chaos in the subcontinent this was a Herculean project. Kundan also corresponded  with whoever he could in Pakistan to get any documents he could obtain from Khanewal. A date would be set for him to show up in the court. Court systems are notoriously known for delays but those times were particularly worse than they would be under peaceful conditions. Kundan would show up on the appointed date and invariably the date would be postponed and the process would repeat. One had to go every time because he never knew when the case would be actually heard and decision taken. Der aaye drust aaye,(came late but reached correctly) is a common phrase used in India. 

The Hindus and Sikhs refugees from what became West Punjab abandoned 6.729 million acres. Muslim refugees from East Punjab, Delhi and UP abandoned 4.75 acres. Accordingly formulas were created to give a lower percentage of the property compared to what was left behind in Pakistan.  

Kundan was compensated some tillable land about 20 miles south of Panipat, a distance he would cover on the bicycle. Two parcels of land were allotted in Delhi. One was on the ring road in Laj Pat Nagar, Delhi. This land was in close vicinity to where the current Ram Sharnam is located. The house that Kundan purchased in the newly developing colony of Model Town in the vast open land west of the railway line was also later paid partly from the claims reimbursement. 

But the land and house does not feed the family. Money was needed for basic amenities for day to day living. Fortunately Vidya's father, Mathra  Das had sold his two houses in Sargodha and brought the cash with him. He loaned the badly needed money to Kundan to run the household. 

In the meantime, Kundan got an employment with Ram Das, great grandson of Moti Ram. Moti Ram had four sons-- Ram Singh from whom came Kundan's lineage, Ganga Ram, Jinda Ram and Duni Chand. Ram Das was son of Duni Chand, a compatriot of Chaudhry Ram on Kundan's side. Therefore Ram Das was Kundan's grand uncle. 
Journey to Palwal (Now in Haryana), 50 miles south of Delhi just past Ghaziabad was difficult. This was the station where Mahatma Gandhi was stopped and returned to Gujral by the British on his way to Punjab to join non cooperation movement following the massacre at Jallianwala Bagh. Gandhi’s statue was installed at the station in 2013. 

Kundan’s journey took all day. Our house in the city was at the far east end and train station was past the western boundary of the city. He walked from home to Panipat train station. The local train took about 4 hours to Delhi. After few hours of wait he changed the train which took three hours from Delhi to Palwal and finally walk to his residence. Journey was a torture and was not cheap for someone who had no money. This ordeal resulted in visit to Panipat every two weeks. 
Ram Das had a brick kiln in Palwal where Kundan got his first job in free India. He had experience of brick making I Thatti where Lala Gokal Chand had two brick kilns to sell bricks to fast developing Punjab. Even though Kundan had experienced making bricks in Pakistan, this was not his primary vocation as he was farmer by training and at heart still carrying the lost dream of becoming a doctor. 

After a few years he decided to move to Panipat. Hundreds of thousands refugees had decided to make Panipat their home. The vast open land to the west of railway lines was open fields for miles and miles. In 1949 the government had decided to create housing in this area and this was called Model Town. All this new construction needed building material including bricks. Kundan created a partnership between himself, his sister Sheila's father in law--Tulsi Das Khera, a dynamic man who lived into his nineties and kept working almost up the end of his innings, and Khushia Ram, a distant relative from Karnal whose one son Roshan Lal, commonly teased by getting called Roshan Daan by Kundan's children and a butt of jokes by Prem, lived in Panipat. The KKK depicted in the center of the rectangular brick represented Kundan, Khera (Tulsi Das) and Khushia Ram. 
Kundan was not business savvy and rumor was that he got cheated in financial transactions. He also hated the business side of the work with so many inspectors to deal with and giving off bribe. He would always sweat at the appearance of an inspector at the door. With plants he felt in control but with the inspectors he felt at their mercy and was not very good in the methodology or the system to bribe others. In later years he would instill this fear of inspectors in us and would rather see us as professionals. 

The brick kiln called Bhattha in Hindi was located on the northern side of the train track connecting Jind with Panipat junction railway station. On the opposite side of the track towards Model Town  was the Bharat Carpet factory owned by Anand Sagar Khera, Kundan’s brother-in-law. Before partition time Kundan was wealthy enough to have loaned possibly 15000 Rupees, which was a huge sum those days. No one ever heard from him or Vidya about the settlement of the accounts. Except perhaps once when in the absence of Kundan, Vidya needed a Chawanni, quarter of one rupee, to get her broken shoe repaired. She walked bare feet to ask for that but her request was declined. This was her only shoe. The cobbler, who used to sit along the road outside house number 16, fixed it, agreeing to get paid later. They did not let this loan issue ever spill into the relations between the families. So much so that children did not know this issue while they were growing up. In fact they never complained about anything nor did they ever bring up the losses they incurred or difficulties they were going through as a result of the partition. They were immersed 100 percent into dealing with the present and future of the family.

Just as the plots were being sold, Kundan put in a bid to purchase House Number 2. Someone else had just put in a bid to buy the first house to be built but Kundan was eager to move his family out of the small, dingy house to a large house with a 1000 square yard lot. The down payment of the house which was going to cost a total Rupees 14,000 was just 500. He handed over the check of Rs.500 to the clerk knowing fully well that this much amount was not in the bank. He cajoled the clerk to hold the check for 2 days. In the meantime he arranged to borrow the needed amount and check was good to go. According to Kanchan,Mataji pulled his sleeve and said “For God’s sake don’t bid for a house.” She worried about how they would pay for the house when they barely had money to buy daily necessities. But Kundan's desire to get back to or even close to their living standard of Khanewal was too strong to resist the purchase. He left without saying anything. Where there is will there is a way, was proven by the events that followed. Small payments were made regularly toward the loan. Sometimes payments could not be paid and were defaulted. Later the rules were changed. Eventually the bulk of payment got paid by the government and adjusted against the claims Kundan had submitted to the court. 

According to VIRINDER auspicious day of 1st Sawan, corresponding to 17th July was chosen to move from the old house in Panipat Shehar to 2 Model Town in 1950. With Sural in Jamalpur and Prem in Ludhiana, shifting of the entire household goods of the large family was done by a few hand pulled thelas, escorted on foot to cover distance of 3 miles. This was done mostly by Pita Ji and 13 years short skinny Virinder, Lala Ji plus one ricksahaw deployed for Nani Ji, called Beji by us. They carried one very heavy Takhtposh, on which Pita Ji took first nap in 1950 and breathed his last in December 1976. Also carried were assorted charpais with heavy wooden legs. All but one rotted over the years except one which was used by Mata Ji for 40 years till she passed away on February 7, 1990. 
In addition they also carried wooden almirahs, few jaalees, steel stools, few steel trunks, 2 sets of kitchen supplies. Beji used the left side kitchen—Mothers do not eat from the house of married daughter was the custom at that time. 
The move started early morning and by noon everyone was settled in a four room home. Hurricane and mustard oil lamps were soon replaced by low powered lightbulbs. Electric wiring was done by a SardarJi who lived in Number 11, father of Kanchan’s friend—Surjit. 


Again according to a 13 years old boy, Mata Ji was a robot programmed to get up early and start the routine. She was doing numerous tasks under hard times of practically no money, so many hungry mouths to feed, educate many at different ages, discipline them with tough love. All practically single handed as Pita Ji was mostly away making money to keep family going. Taking care of Lala Ji was a chore in itself. He required breakfast, lunch and dinner to his exact specifications and at specific times. She needed to take physical and emotional demands of her mother and practically visiting husband. In the 13 years old mind he hardly sees her talking. After getting ready, she shuttled light footed between kitchen and store taking stock of curd, light up chulha, chop vegetables, knead dough, receive and heat milk. She attended to beggars, vegetable vendors, Jamadarni, take stock of clothes of school goers, wake up sleepy heads, feed them well give tiffins for lunch in the school. In between breakfast was served to Lala Ji. Once children had gone to school, Lala Ji walked to the city to socialize, she sat down for few minutes and eat the left overs. This was followed by, with help from Ganga Devi, a long session of dirty clothes. Children’s games were such that we made clothes dirty every day. Yet 13 years eyes and ears never saw anguish or hear grumbling. There was a set routine of washing clothes. A wide pan full of water, locally made soap and dirty clothes was heated at low heat over a separate chulha. As clothes were getting hot and partly dirt free, she went to kitchen and start getting ready for lunch. Clothes were taken out, and beaten with what looked like smaller version
cricket bat. Alternating arms were used to avoid fatigue of one till the obstinate dirt came off. Virinder thinks that we should have received the beating instead of the clothes. Boll Bum Bum used to go from home to home every Thursday. He had a bamboo staff which had many ghungoos attached to it. He thumped it’s one end thuk thuk. Mata Ji always gave him some food. Worried and tired of Gindi’s mischiefs, Mata Ji and Pita Ji had approached several Pandits and astrologers. MataJi seeked help from Boll Bum Bum, a Shiv Ji devotee. “Do not was clothes on Thursdays” was his advice. She never washed clothes on Thursdays till her death. 

After washing the clothes she took bath, changed clothes and started for lunch routine—adding wood to already smoldering chulha. First portion was served to Lala Ji at exact time everyday. 
Lala Ji approached near the kitchen and sat down on a Pidi, a low lying square stool. He had announced his arrival by sound of cough. Mata Ji pulled down one end of sari to cover side of face. She made special perfectly round rotis served with two vegetables. Lala Ji did not have teeth and took longer time to chew. The gap between first and second roti was perfectly timed to serve second roti piping hot. He drank lassi, which Mata Ji had managed to make during her busy morning, after the meal. 
Not a word was spoken between the two, yet their communication was perfect. 
Then all of us got the delicious food. It was mostly by 2 PM in summers and 3-30 in winter, after our return from school. Then she ate the left overs. Many times with no vegetables. She did eat, even if only Rotis with salt, onions or aachar. She knew the value of energy. 

Finally she had a short nap. Then the sewing machine came out for repair work on the clothes, even making alterations. Pouches for fruits and grapes was done according to the season. 

Then the third food shift started. Youngsters got snacks before rushing out to play. The fourth food shift meant preparation for dinner. Lala Ji’s routine was the same minus lassi. This time the eating place was charpai. Children came in running, excited with dusty clothes and happy faces. They all had to wash faces and hands. After reading one or two pages of Ramayan or Mahabharata, they lined up to eat Rotis with vegetables and home-made achaar. Then she sat alone on a pidi with Rotis and left over, if any, vegetables. 
Her day ended by putting small amounts of curd in boiled milk to make yogurt for next day, sorting out dry clothes, repair or buttons if needed. Clothes were not ironed for anyone except Lala Ji who always wore washed and ironed by the washerman. Sometimes children lit up coal to heat the iron and press our own clothes, especially Krishan. 
Children received milk and bread. Then she made sure that children had rinsed their mouths. 
Finally the tired body went to much needed sleep. 

Virinder reminisced later “Looking around at many desolates like us at that time of history, I felt lucky for having a person like Mata Ji around. She gave a very happy and childhood full of joy.” He continued with misty eyes “I wonder if it was at some cost to her personally—she was lonely in taking care of her own emotions. Or may be seeing us all happy, she made up for own lonely childhood.” 
Luckily with time things settled down, her true happy emotions found an outlet and her inner child bloomed again. She taught us by example and actions to respect all, particularly the elders. She made sure that we stayed physically and mentally active, gave us freedom yet keeping her eyes open that we don’t go astray. She never reported our indiscipline to Pita Ji, even if we at times had hurt her feelings. No mortal is perfect, during journey of life one is liable to make omissions and commissions affecting others’ lives. Mata Ji was quicker than most to realize this to make amends to the maximum possible extent. 

Prem also shared some of his memories about Mata Ji. Uppermost traits were cooking, household work, worldly wisdom and personal relations. She couldn’t do too much but she loved traveling. 
She loved cooking and serving. Her varieties of Pranthes are remembered by all who had a chance to eat. She made special foods according to the likening of the family member or guest. Saag and Pranthas for Suraj, Karele, baingan bhadtha for Prem. Some dishes like Chawal vadi, arbi with gravy, biryani where mutton is cooked in the rice etc. are on extinction list. 
Prem reminisced about the most memorable lunch at Krishna Oberon in Hyderabad during 1989 FGT. Shoki and Gindi were thinking of planning to honor an elder. Suraj was being considered. Namita says “ Are you kidding! We must honor Mata Ji!” Mata Ji sat on a chair set in front of all others. Each one presented a rose flower to her. A hug was followed by her blessings to each one individually. She left the world peacefully in her own home, where she was the queen, in her own kitchen where she had her throne of a pidi and doing what loved the most, cooking her home grown baingan holding between her fingers her most prized masaale. She passed away before a frantically running Sushma could even get a doctor to arrive. Her fear of a prolonged sickness entailing burden of others never came true and she died a saintly peaceful death on February 7, 1990. She had survived widowhood for sixteen years bravely but the untimely death of her 45 years old son, Krishan made her say “Seems that Krishan’s death has left a pus filled ulcer in my heart and it keeps oozing drop by drop.” 
Prem fondly reminisced about Mata Ji’s other traits. She laughed heartily relishing the happiness and enjoyment of the moment. A prolonged laughing session triggered overflow of her tears. She wiped them with the end of her saari, which generally was white. If made to laugh too much, she jokingly scolded “bus kar bhadiya.” It meant stop it you bad boy. This trait was passed onto Prem who always kept a handkerchief in his pocket. His easy laughter and seeing humor in everything make the hanky appear often. 

While Pita Ji relished playing cards with his friends, it upset Mata Ji. She had to make tea for them every time. In her retirement she had no desire to cook more than essential. 

The farmer in Kundan flourished as was evident in the number and amount of fruits and vegetable that were grown in Number 2 Model Town. An owner of a nursery sold a large mango tree for planting in the front yard. Kundan felt that the tree was too large to take roots in a new home and it would be a waste of money. It had to be placed on a cot and carried by several people to its final destination. The nursery owner asked to be paid after a year and that also only if the tree survived. Lot of discussions were held about where to plant it. Finally it was placed near the front entrance. Ordered by Lala Ji, it was Virinder’s duty to hold onto the tree when it became very windy. Doubts of the tree taking root vanished soon as new off shoots started appearing. Pita Ji would lie down on a cot under the shade of this tree. Not only did it survive, it flourished under his tender care. It continued to provide delicious Dussehri, yellow mangos as long the house was in the family. All the children and grand children have deep fond memories of this tree. There were 6 guava trees, one gave pink amrood while others had white pulp, one Jamun tree which had to be at least 60 feet tall, Shehtoot ( Mulberry) tree, two pomegranate trees with their beautiful flowers and sweet anaar, a lattice covered with sweet grapes vines. He would feed the plant with goat blood he procured from a butcher. Small cloth pouches were stitched by Vidya and placed over the bunches of grapes by the children to protect them from birds. Banana trees, papayas, an offspring of the original mango tree, a lemon tree provided all the fruits that the family would need. The fruits were also distributed to neighbors till they started complaining--"Oh no, more amrood!! from Mataji". That is when she started to ask the children to fill containers of guavas and take them across the railways line to be sold to a vegetable shopkeeper on railway road. In exchange he would give the kids the vegetables that we did not grow at home. Potatoes, egg plants, bhindi, mirchi, matar, were all grown at home. Two Neem plants provided the tooth brushes to the whole family. The appropriate length of a brach was broken off, leaves stripped, one end chewed to make bristles and there you were, a ready to use tooth brush. Women used bark of walnut tree to clean their mouths. Shakuntla Behan Ji used to do the same. 

 It was the children's job to shoo away the parrots which wanted their first bites of the guava. If the fruit was not ripe, they would go onto the next one. What a nuisance it was. The kids had fun with sling shots and banging of steel drums. Between the parrots and neighboring children climbing the walls to pluck the fruit, there was still more than plenty for all.  

In addition to the irregularly, irregular non-dependable water supply from the city, Kundan got two hand pumps installed. One in the center of the sprawling yard and other near the main entrance of the house, next to the outdoor kitchen. It looked like this one—
 
Small trenches were made to get the water to the plants everywhere. The hand pumps were also used to get steaming water in winter months for bathing because to underground temperature was constant 50 degree F. The weaklings would still mix some boiled water to make it more bearable. 

The toilet systems in the house also kept changing. Initially the two portions of the house had the latrines in the center, open on the top and facing the front yard. There were two doors , one for each large hole in a concretes two foot deep concrete slab. There was a wall between the two pot holes. Jamadarni came every day to haul away the refuse. Shooing away flies and many times seeing white wiggling worms in the feces was common. 

Later this area got converted to a large store room. Two windows replaced the doors. 
They were verandas with their arched openings. Later they got covered providing extra rooms. Its solid wood door with jali doors and springs behind them got installed. There used to be a hand pump next to the mango tree. During winter months initial water out of it was freezing cold, but if we kept pumping the water soon we could see steam rising out of the bucket. Everyone jumped up with joy on seeing the steam and collected the water in the buckets. That was the time to jump under the freshwater. We took turns to bathe each other. That is if we ever did take a bath. Next to the hand pump, there were three papaya fruit trees, and two banana fruit trees. The hand pump later moved to the middle of the courtyard and later shifted to the area between the modern toilet and gate number one. The gate somewhere along had changed from wooden door to steel door. Sometimes along the stretch of time running water system was installed. Water supply was intermittent, of course. We used to hate it when in the middle of cleaning teeth with ‘datum’ the water would stop. We liked keekar datum. We were told to use neem datun. We did not do that because it was so bitter but they knew better. Rest of the world later found out the benefits of neem. A faucet with the white sink got installed to the right of the entrance door number one. A small mirror was hung above it. It’s run off water ran into a small drain which would go around the house and drain into the main drain around the house outside the wall. Mataji used to wash clothes with help from neighbor lady, Ganga Devi. Her husband was supposed to be a Pandit but was well known to us kids that he loved to eat meat, drink, and eat onions. But he did this secretly. Desi soap, a square piece which was thicker, non-scented and cruder than bathing soap, was used to wash the clothes. The clothes were then beaten with a small cricket bat like piece of wood, sota. This was sometime used to spank the clothes and sometimes us, mostly me for doing things that others in the family blame me for. I am told that I used to break doors and water pitchers, throw food, and in general, I was born just to give trouble to Mataji. Anyway, that mini cricket bat beat the clothes which were then squeezed and re-squeezed and then hung on a long steel wire that stretched from near the mango tree to shatoot tree. During one of these washings and draining the soap, I threw the one rupee coin into the drain and let it run down the stream to the main drain outside the house. When Mataji found it out all hell broke loose. Everything was halted and everyone of us sifted through you know what, in the main naali as it flowed outside the house. This was filled with putrid smelling sludge. City jamadars took out the sludge with a U shaped metal scoop attached to a long bamboo stick. He moved the sludge along or threw it on the side of the nali. Of course there were no plastic gloves. We sifted through this mess for hours but we’re not successful in locating our last treasure of one rupee. You can all guess what happened – mini cricket bat on my butt made a century. Wonder if we ate any vegetables that night. Money was really tight, but love was plenty. Who cared for money when love trumped all sins. 

A new store room was built across from the mango tree next to house number 1 boundary line. It had an open veranda in front of the door in which two metal rings were installed. A strong rope and a wooden seat made a perfect Jhoola (swing). We used to swing high up in the air on both sides.. We jumped far out on the grassy patch, the winner would be who lands farthest. No one ever got hurt, flexible bodies, soft grounds helped. 
Later a Pooja room was added, which reduced the size of the swing which we and grand children enjoyed. 

The latrines got moved to the far left corner of the lot as seen from the main entrance door, behind the store room. The custom of a pretend cough was used when going to toilets. There was no roof, no doors, just a half wall to protect the viewing of the insider by the others playing in the yard or the next user of the facility. The newcomer planning to use the toilet would make a coughing sound and if it was already occupied that person would cough back. The next in line would wait outside with a container full of water in their hand. When Dolly came to the house after marriage in 1968, she found it strange and nauseous. 

Later the sit on the feet laterite with a flush system was made in the left corner, flushing the stuff into open sewer just outside the outer wall. There was no need to do yoga classes, if you could sit in the laterine. 
The older one got used for collecting wood for fuel. 
There was a big bee hive here every year. 
Mataji was allergic to milk, although she could eat yogurt without a problem. 
Surprisingly honey bees did not bite Mataji. She simply swayed her arms as if she is shooing away house flies. We as children, used to hold our breath till she got away from the hive. 
Dolly also was surprised to see a line forming to brush teeth at the only outdoor sink near the front entrance. To the rest of us it was a norm, not having seen anything different. 

Jamadars and jamadarnis had the unenviable job of taking away the refuse in shallow wide metal containers. These people were the ones who were labeled as the untouchables or Dalits in India. Their level in the society was considered so low that they did not fit into anyone of the four caste or Varna system of the Hindu society. Hindus had four casts: Brahmins, Kshatriyas, Vaishyas and Shudras. In the Luthra household, the jamadars or untouchable were not allowed in the kitchen but they were treated with respect. In addition to the agreed salaries they were given food or drinks as a token of appreciation and sharing. However, they had separate utensils which were exclusively used for them.  The jamadard and jamadarnis would wash their utensils and stack them separately with no chance of getting mixed up with the ones used by the family members. 

When most of Panipat Model Town got built up, Kundan decided to look at new pastures for business. His distant cousin Ram Labhaya lived in Ropar. That must have been the reason to start a btick kiln there. Ram Labhaya’s oldest son was Gurbaksh Singh Luthra. He wore turban and raised Sikh family. His younger brother was Shyam Sunder Luthra, a mathematics teacher at Government college. He was the reason for first Krishan and then I joined this college. Shyam was instrumental in switching me from Pre-Engineering to Pre- Medical. After a short stint in Ropar, Punjab, he started kilns in Chandigarh. In 1952 Chandigarh was the first planned city getting ready to be built under the guidance of La Carbuisier, an eminent architect of  of France. In 1956 with partner, Ishwar Das, Kundan started a Bhattha near sector 23 in Chandigarh. He along lived in a two room rented house in Sector 15, Chandigarh. His daughter, Kanchan stayed in that house from 1956-1957, later shifting to hostel as she was studying to complete her course of Bacheolar of Education, (B.Ed.) to pursue a career in teaching. By this time he had mellowed down from his nature of quick anger. It is here he told Kanchan about his inability to get admission into Medical college due to quota system. Fate had different scheme but after her husband, Bhushan Arora, passed away, this training came handy as she taught English to stay busy and connected.

 ‘It's  a small world after all’ is just not a saying. Kundan's next to youngest son, Juginder's future wife, Sucheta (Dolly) was growing up in Sector 22. Her father, Sushil Kumar Sachdev was one of the first 10 people to inhabit Chandigarh. Initially they were living in tents till they got a house near Kiran cinema in sector 22. In 1964 they built 182 Sector 11, their permanent home. He was an overseer planning and executing the building of the most beautiful city in the country. They had moved to Chandigarh, a city named to honor a Devi in whose name a temple called Chandi Mandir is located in a nearby village. 

Children were growing up, going to schools and colleges--a dream of Kundan and Vidya that every child must receive education so they can stand on their own feet. 

Kanta was in class 5 years and Kanchan in class 3 when we migrated yo Panipat. Being two years apart they were very close. Once Kanta failed Mathematics examination. Both the sisters cried a lot. They were good students and teachers loved them. 

Life was being lived fully, children were provided  enough that they never felt deprived or poor. Suraj and Prem's contributions and later Virinder's in addition to  Kundan's earnings made sure that every child got education and enough food, clothing and shelter. When youngsters gave trouble, they were told “Behave, don’t go out—pathans Will pick you up, cut you and fry you on a hot pan.” Listening to this the young lads followed every order. According to Kanchan’s memories, Gindi and Krishna used to fight a lot. Sometimes it seemed they would scratch other’s eyes out, fought with axe and 3 inch round wood. Mata Ji did not intervene and let them continue their fights and she stayed busy with work. Kanchan asked her why she didn’t stop them. 
Mata Ji said “I feel happy when they fight, at least they have each other. They fight one moment and next moment they are one. Being the only child I saw other families all together but I had to be careful not annoying others in which case they might stop playing with me. So I am glad that they have each other to fight without worrying about the consequences.” The fighting drama created funny moments when Krishna was armed and Gindi was not and hid in the big room. Or when the fight created lot of scattered furniture. Sudden appearance of parents caused culprits to run away and little Shoki, just an observer got slapped. In Khanewal, Kanta and Kanchan were assigned to take care of Krishan and Gindi. When they went out, they had to take the two young ones with them. Mata Ji used to stitch nice clothes for the children and sisters happily dressed them up. 

In our first home in the city, there was frequent departure/arrival of Pita Ji, Suraj and Prem. It was customary to see them off to the street level. Sometimes Kanta would stay back, alone crying in a corner, something she rarely did. On asking she said “I feel sad when Bhapa leaves, even when he goes back to Ludhiana.”

As Kundan got busy with trying to raise money, his wife, Vidya became the pillar of strength for the safety of the family and running the household almost single handed.  Kundan used to say “Your mother saved this home. Without her the family would have been scattered and destroyed.” 

Vidya was the only child of Mathra Das and Kesar Bai Virmani. Mathra Das was born in 1870s in Khewra a town in Jhelum district, known for salt mines from which 98% pure rock salt was made from pinkish rocks. Consumers purchased it in the form of small rocks and crushed them in hamam dasta.
Mathra Das moved around, possibly related to his job. Vidya was born in Samundri, near port town of Karachi on January 1, 1912, sharing the same birth day with their first son, Suraj Prakash and a future great grand daughter, Disha, daughter of Ashit and Jyoti. 

Famous movie and stage actor Prithvi Raj Kapoor lived in Peshawar, about 320 miles northwest of Samundri. Vidya recalled having carried Raj Kapoor in her godi. Raj Kapoor was born on December 4, 1924. Vidya was about 13 years old. Obviously Mathra Das had been transferred to Peshawar by now. Vidya owned a Ghodi, mare so she didn’t have to walk off the heated roads and sand. 

Sargodha area was nothing but a forest. In an attempt to develop it, free land was offered to any one who agreed to build a house there. That might have been a reason he chose to retire there. At retirement he was a rich man maintaining a pension of Rupees 30 a month. He had two houses in Sargodha. The one they lived in was on plot  # 6. It's ground floor was rented and they lived on the upper floor. The second house was fully rented. Mathra Das was Arya Samaji, a tradition of doing Havans and followings of Vedas. Belief in Arya Samaj continued in Panipat. We used to visits to Arya Samaj center in Panipat. It was located between house number 17, which Kanchan and Bhushan had purchased and later disposed requiring many visits to the courthouse by Kundan, and the mangoe garden. It was a white building primarily composed of a hall without any Murtis of gods. We used to have Havans at home also and surely at all marriage times. 

I don’t have a picture of our Nana Ji. He might have some resemblance to his bother, Bhagwan Das. He was father of Chuni Lal and grandfather of Uma Malik, Usha Sethi, Indu Virmani and Bharat Virmani. 
 
Mathra Das's brother, Bhagwan Das had a son named Chuni Lal Virmani. Even though Chuni Lal and Vidyawati were cousins, their relation got strengthened as real brother and sister. She used to tie Rakhi on his wrist every year. 

 
Chuni Lal Virmani, our Mama Ji

He had become a favorite Virmani Mama Ji whose visits were always anticipated with great enthusiasm as he would bring gifts for the children and play with them. He taught us yoga and even the sophisticated game of Bridge. Virinder, a teenager at that time simply got to sit by his side and wondered if he would also become a bridge player. His car and membership to Chelmsford Club in Delhi used to impress the youngsters. 


Mathra Das died in the summer of 1948 in the occupied house in Panipat city. Kesar Bai moved with the rest of the family to 2 Model Town where she died in August 1954, just two months before the long anticipated wedding. She never got to throw the coins she had been saving for years to throw over the Sehra covered groom, Suraj. She had gone for a walk. On her return she complained of chest pain, collapsed and passed away at home. 

Vidya was married with Kundan Lal Luthra on January 21, 1927 at the age of 15 in Sargodha. Kundan was so handsome that when he visited Sargodha, women used to discretely peek and say “He is so fair and so handsome.” 

A succession of 10 children were born between 1928 to 1947. 

Life at 2 Model Town, Panipat was a very busy one. Even though Kundan was engaged in outside jobs of settling  claims, going to the land and several other errands, he was always involved with children's education and marriages. The two elder sons Suraj had joined Jamalpur Railways Engineering college and Prem joined college in Ludhiana. Initially he had chosen pre-medical classes. Some doctor there told him that it would require lot of hard work and many years of training. Prem changed his subjects and decided to complete Bachelor of Arts. He accepted a job with Alpha Radios and Novelties in Madras. 

Vidya was responsible for running the household occupied by her father in law, Gokal Chand, her mother, Kesar Bai; two daughter--Kanta and Kanchan; four sons--Virinder, Krishan, Juginder (Gindi) and Ashok (Shoki). Vidyawati slept very little. She was always found to be working. Various activities included chopping vegetables and cooking, mending and washing clothes, cleaning the inside and outside of the house, taking care of children, one of who would definitely fit the definition of being called a brat. He would make an already difficult life even harder although he didn't have the wisdom to know what he was doing. He would break crockery, throw away the precious dough, break clay water containers. Vidyawati or Kundan did not use any physical beating, except once. 

There is a custom in India in which the numbers 1 is added to any monetary or laddos given at an auspicious occasion. Sweet 'ladoos" are commonly exchanged between friends and family. As per custom there would be 11, 21 or 31 etc. One such tray had been gifted to the family. All the expensive eatables were stored in a cupboard that had one panel made of cane so that one could see the contents inside without opening the door. It was always kept locked with the key secure with Vidya. The family could not afford to eat these expensive sweets daily. These precious and rare goodies were taken out on special occasions. One day  some guests arrived. As was customary, a table with big cane chairs called "mooras" were set up for serving tea to the guests. Kundan got the key to the cupboard, handed it to Gindi and 
asked him to get the "laddoo" tray. Holding the tray, all alone in the store, the laddoos were too delicious for the drooling kid to resist and he quietly ate one, cleaned up his mouth and obediently placed the tray on the table. Kundan looked at the tray and asked the brat " Did you eat one laddoo?" The quick answer was "No". Are you sure that you did not eat one and when the same word, no, was blurted out, a hard slap was given by Kundan to Gindi's left cheek. Later on Gindi learnt about the tradition of  11, 21 etc. and the reason how Kundan knew that he had eaten one laddoo. 

With age and with exposure to the sun when one drives on the right side of the road in USA, the left cheek get darker compared to the right. That darker shade on the left cheek reminds Gindi of his loving and caring father, Kundan Lal and the lesson with one slap he received. 

In 1962 Ashok was the last one to leave the house to join F.Sc. Non-medical course in Sonepat. It was only 30 miles from home. He used to come home frequently. The bubbly house gradually started lowering the sounds and the actions. Mata Ji and Pita Ji heaved a sigh of relief of mission fulfilled but sound of silence was deafening. Saving grace was that children, grand children, relatives and local friends visited often.

In the winter break, Virinder was home for an extended leave. Pita Ji was playing cards with his friends on the right side of front porch. Mata Ji made tea for them with one hand on her chest. She complained of chest pain. Pita Ji dropped his cards, made Mata Ji lie down on the floor. Virinder quickly rode the bicycle to Dr. Lal Chand’s clinic on the railway road. He found that Dr. Lal Chand had left Panipat after he had given a wrong injection to somebody resulting in death. Bad reputation followed in the small town making him abandon the town. 
Virinder rushed to Dr. Dhamija’s clinic. He promptly came over and gave injection of Pethidine. Mata Ji slept all day. A later ECG showed an old heart attack. Virinder helped save his mother’s life twice. 
The news of Mata Ji’s sickness spread fast. Many visitors started coming. Virinder had a gate duty. He did not allow anyone to come and disturb Mata Ji. 
With complete test, gradually the pain subsided. 
Later on she got check up in Delhi. An ECG showed an old heart attack. She was given medicines for coronary artery disease. Virinder helped save his mother’s life twice.
Life was moving along at a slower pace. 

Pita Ji was not fond of traveling whereas Mata Ji loved going places, meeting people. He did like going to Calcutta. It was here that he had a silent heart attack. 

He did like to go vegetable and other shopping at Bosa Ram choke. One morning he did make a usual trip. He felt fatigued. Enough that he skipped Satsang. 

Suraj was the first born child of our family. He was born on January 1, 1928 in Sargodha. He studied in .——School. He passed Matriculation from—- He holds record of maximum marks—704. Prem received 703. He was one of 17 students selected to join Railways Training College in Jamalpur. He got married to Urmil Malik in 1953. His jobs took them to Railway colony Delhi, Lucknow, Jodhpurs and back to Railways colony in Delhi. They built a house at B 1 Anand Vihar and retired life was spent here. 
He loved playing Bridge and was extremely good at it. On July 3, 2017 he had a major stroke. Cerebral bleed was the cause of stroke. He passed away on July 4, 2017. 

Urmil was born …
She studied …
She was a teacher…
She loved to play Badminton
She started showing early signs of Parkinson’s in 1987. This continued to progress. She was bed ridden in the 15 years of life. They did try to split time between Takshila and Delhi. They finally lived in Delhi only. 

Rajdhani express train was started between Delhi and Calcutta. The rail tickets used be a rectangular thin card board. Suraj and his family’s photograph adorned tickets of the first batch. 

Prem Prakash Luthra, generally known as Prem Luthra was born in Thatti in summer of 1931. School admission requires a date, therefore Pita Ji randomly picked First day of June, 1931. Stories go that he was so fair, like a shiny pearl. Therefore got Nick name of Moti. Our beloved Chuni Lal Mama Ji, Mata Ji’s cousin, fondly gave this Nick name to Prem. Prem jokingly said it was because we had a street dog called Moti, resulting in us calling Prem as Moti. 
Prem had approved arranged marriage to Shashi Kalra who lived in 100 Model Town. Marriage ceremony was in Panipat on 8th October 1960. 

Prem studied in three schools in Khanewal: Arya, DAV, and Government schools. As written by PREM—“The time would be October/November 1947. We were at Sabathu. My result was to be declared the next day. I sat down with Suraj (possibly Pitaji too) and made a guess of my marks. Guess was 705. Suraj had obtained 704 and perhaps it was a record for Khanewal Schools. The next morning’s Tribune newspaper gave me 703! I was very happy-a very good result and keeping me at my rightful place, next to Suraj. The marks included 39 out of 40. For the practical science paper, partial thanks to a sumptuous Mataji-cooked mutton-rice-phirni dinner for the examiner at our place the previous evening.”

He had no interest in joining a college. He just wanted to loaf around and play Bridge. Pita Ji firmly refused to accept that. Reluctantly Prem joined Government college Ludhiana in 1947, with classes stating in March 1948. 
According to KANCHAN he used to travel by Bombay Express which had a new style of rounded front shape. There were no homes between our house and railway line. Prem left the house when signal came down announcing the arrival of the train. Shoki was engrossed making two parallel tracks in mud. He held two sticks, one vertical and one bent. He talked to himself “Chhing don ho gaya—Bombay ichh pichh aandi payi ae.” 

Initially he wanted to be a doctor and joined Pre-Medical. A smile , word of encouragement or discouragement or action can change the path in an instant. A doctor told him that you have to work very hard to become a doctor. This changed Prem’s mind, course of his life and he switched from biology to mathematics. In the college, he excelled in Badminton, carrom board, table tennis, chess and above all Bridge. 
After graduation, he joined Alpha Radios & Novelties as business and job in Calcutta from 1951- 1953. He worked in Madras from 1953 to 1958 and then back to Calcutta. He has one surviving son, Ashit. Shashi and Prem lost one son soon after birth due to Rh factor problem and could not have any more. 
Problem of intermittent back ache stayed as a nuisance with Prem through out his life. He wore brace during long distance traveling. First episode happened in Panipat in 1962. Best known treatment for this in Panipat by a specialist who was a breech delivered person! Prem laughed at this strange qualification. Pita Ji took him to one such quack in a rickshaw. The guy blew air from his mouth at the site of pain. Pain cleared up in two or three days. usubsided in two or three days. Ashit married Jyoti in 1992. He was the glue of the family, always giving, always loving. He contributed generously yo the family fund so the youngsters could have a rich full childhood and education. 
His main hobbies were reading books, writing and storing letters. He was good in playing Badminton, an excellent Bridge player and loved photography. God was king for taste buds and he made full use of it by trying numerous foods. He loved chilled beer. 
He loved playing pranks all his life. 
An excellent listener, conversationist and bringing out the best whoever he came in contact with. 

Prem felt he had poor memory of past events. But after exchanges of numerous letters, many hidden memories bubbled up. One evening he reminisced of past after a glass of chilled beer. “ Pita Ji practiced Neki Kar darya mein daal—Do good deed and let it flow down the river. Pita Ji had written about 10 quotations which he prevailed in his life. 
Pita Ji literally blushed if someone praised him. He was, so to speak, allergic to compliments. He was honest, good hearted. He had least desire for material things. In early 1970s he was upset and unhappy when he received a gift of new shirt. He was content with his own old shirts and hand-me-downs from his sons. His happiness was strongly linked with his children and grandchildren. He detested waste to the extent of finishing non-tasty food rejected by others or burnt milk, over ripe fruit etc. 
Couple of long sips later, the flow continued. 
“Pita Ji’s emotions reflected on his face including anger and displeasure, although they were short lived. Once he got really upset when he found that Shashi had 20 pairs of footwear!”

“We had planned to visit Panipat and informed Mata Ji and Pita Ji of the dates. But the company arranged an all paid trip to Goa for us on the same dates. I wrote to Pita Ji that there is a change in the plan. We got reply right away. He wrote—‘It would have been alright. But now we were waiting and looking forward to your visit. It is not fair.’ 
We cancelled our free trip to Hoa and went to Panipat as planned. No one knew that Pita Ji was going to die soon thereafter in December 1976. 
We left Panipat on December 25, 1976 by afternoon train. Pita Ji did not reach home from wherever he had gone to see us off. He met us at the platform minutes before the train arrived. He handed us a packet ‘Your mother wanted me to get this and ask you to give these almonds to Ashit.’ We saw him lovingly and longingly waving to us from the platform. Was it the rushed hurried trip to the bazaar, that made us feel he looked tired. We flew to Calcutta on the 27th. A letter arrived on 29th that Pita Ji had peacefully left the world on 27th evening—a saintly death. He had mala in his hand and Mata Ji was pressing his legs. 


In December 2013, he had bout of cough. Tests revealed Multiple Myeloma of esophagus which created an irreparable fistula into his trachea. Pneumonia caused his death at 8.55 AM on February 18, 2014. Prem lives in every heart he touched and encourages all of us. 
Prem rarely complained about anything. Money was never a measure of success in his eyes. He often counted his blessings: 
Resolution of Shashi’s osteoclastoma, Disha’s congenital dislocation of hip, Tanuj’s squint, happy married life, bilateral love and respect to family on both sides, good health, Jyoti’s success, and above all—not being slave to luxuries and no greed for money. He loved Family Get Togethers ( FGTs), 2 Model Town and family mail group. He summed up his life—I came, I conquered, I failed, I went. 
 


The older of the two sisters, Kanta had an arranged marriage with Laxman Das Khera on        1953. She was 18 years old. Laxman was generally a quiet man who lived in a joint family in Anand Parbat, Delhi. He used to work as air traffic controller at Delhi airport. Later in partnership with aman who turned up not honest, Jija Ji had a shop in Chanakyapuri. Kanta taught in the Modern school, Darya Ganj and had two children, Neeru who became a physician married Dr. Anil Arora and moved to Florida, USA and Sanjay (Babla) who became an Engineer, got married to Madhu and settled in Canada after living in Malaysia for a while. 

Suraj, having finished Engineering, finally got married to Urmil on October 2, 1954. The eavesdropping children heard the word Darling for the first time and were quite amused by it. The marriage custom during that time was arranged marriages. Sometimes the parents would know of each other's family and sometimes there was a go in between relative or a Brahmin. The parents would meet, talk and agree to the alliance of their sons and daughters. Suraj and Urmil had this alliance agreed upon by their respective parents 7 years earlier. In fact, other than seeing each other's  photos, they never physically met till the day of the marriage. Fortunately there was no dowry system in our family. 

KANCHAN


VIRINDER 
He was born on May 12, 1937 at home in Khanewal. He was named Bhushan, pet name was Bhushan. There were two other Bhushan in the extended family. Later he decided to get his name changed to Virinder. 

Virinder remembers traveling roads near Bilaspur and Sundernagar when Gammon had a project in that area. 
“Driving back towards Kirtpur Sahib I looked for him on the road turning to Shimla on which Krishan made his last fateful journey on the morning of 12th December 1987. 


Juginder 
When was Juginder Luthra born? It has always been a mystery. There were no mandatory birth registrations in Khanewal. According to some guesses, it was either somewhere between Kartik and Poh of lunar calendar, position of the moon and how cold it was. Consensus was about 23rd of November 1943. 
The official birthday is what is entered on the matriculation (10 grade certificate.) The story we heard was that Pita Ji took me to Sanatan Dharm high school. When asked by the teacher “What’s the child’s date of birth? Not knowing the date Pita Ji said write down April 13, 1944. It is a Baisakhi day celebrated to mark cutting of crops. From then on my birthday on all official documents is April 13, 1944. 
For sure it was in Khanewal. Deliveries were done at home. I was 3 years old at the time of partition in 1947. I have no memory of Sabathu except once whe Kanta save me from attack by a monkey family. I again have no memory of the old house in the city of Panipat. I do have vivid memories growing up in 2,Model Town, Panipat. At that time Panipat was in the State of Punjab. Haryana was carved out from Punjab on linguistic basis. Panipat was part of majority Hindi speaking Punjab and therefore part of Haryana. Haryana was comprised mostly of native Jats. Mata Ji would sometimes say “We have always been refugees.” The glorious childhood memories have been written in the second half of the book. First to fourth grade schooling was in a make-shift house with lot of Kant at the south end of Model Town. Most classes were outdoor, under trees. Fifth to tenth was in S D High School by the side of G.T. Road near the wool mill. 
Pita Ji nephew, Shyam Sunder Luthra was a mathematics teacher in Rupar, Punjab. Pita Ji’s thought was that Shyam Sunder would keep an eye on the youngsters and be there in case they needed any help. My older brother Krishan was already in the second year of F.Sc. ( Faculty of Sciences)-non-medical. This mostly meant that the student was headed to join Engineering college. 
In 1959 I also joined the same two years course. Pita Ji had different ideas. During first 3 months out of two years course I was taking non-medical courses. While I was away on a 10 days, school sponsored India trip, Pita Ji with help from Shyam Sunder switched me to medical courses, deleting mathematics and physics, replacing them with botany and zoology. 
I have written about this in Dreams Don’t Die. 
Receiving over 60% marks was almost essential to get admission into a medical college. In addition to studying, I took active part in table tennis, carton board, cricket and boating. River Sutlej was within walking distance of our college. The hostels were across from the college. Later I learnt that Rupar was a historical town. Strangely Babar, the first Mughal passed through Rupar and first war of Panipat his way. 
I got selected in Medical college Amritsar and joined in 1961 at the age of 17. Again in addition to studies, extra curricular activities included cricket, table tennis, carrom board. New activity was acting in dramas. Sucheta Sachdev tried to join medical college in 1964 but she was not quite 17 years on official certificate. This was the minimum age for Medical college but not for Dental college. She joined the Dental college. We saw each other on December 6, 1964. A drama called ‘Paisa Bolta Hai’ ( Money talks) was being prepared. Drama director, Dr. Surinder Chibber, auditioned Sucheta as the lead actor and I was assigned a side roll. Her name in the Play was Tara. After one rehearsal I told her “Tara Ji, aap bolti bahut achcha gain.” Little did we know at that time that this line would turn into life long married life with its ups and downs. 


We used to have torian vines between # 2 and 3. When vines got dry they had fine tunnels of open veins through the stem. We would break the stem, light one end , inhale and there was natural cigaret. This was followed a couple of times with bidis. They tasted and smelled terrible. This lasted only a few times. Cigarettes, later on, were bad enough, if only for a few years, one too many. 
The gray steel gate followed a one piece wooden door and later replaced by the double door. Much else is written separately. 
I do remember buying winter clothes donated by rich Americans for poor countries. A man sold them from a wheeled cart. We wore hand-me-down clothes from Suraj, Prem and Virinder. Virinder bought one light blue jali wali shirt with his first salary. I saw it, loved it, he gave it to me without even wearing it once. Shoki shared it later in Chandigarh. 
I have saved Mataji’s ECG tracings. Doctors used to write her name sometimes as Mrs. Vidya Wati but mostly as Mrs. Mataji. 

Ashok (Shoki) 
I remember smoking self made bidi/ cigarettes under the pipal tree. Laxman Jeeja Ji chanced to see us and before I could reach home he had already informed Pitaji. I remember getting verbal banging. 

I remember the painted gray gate, the house painted yellow. Cannot forget the mandir next to store room. That was my favorite because of jhula being there. 
Pitaji rarely went to the school except for getting children admitted. 

SHOKI recalls that his Master Hans Raj came home to request Pitaji to come to school to see Ashok get four prizes—singing in inter-school competition, debate, cricket and one more. 

When I think of my childhood at two model town, out of my seven brothers and sisters, the memories belong to Gindi Christian Kanchan from Virinder Suraj, concha, roughly in that order, which is based on the time spent by them during those years of mine.

Suraj Bhapaa, I remember as the officer. He was a big shot as far as we were concerned. I cannot forget when once we accompanied him to Panipat railway station. We normally would have bought a platform ticket to enter the station like any other ordinary mortals. But this time no platform ticket was needed and as we entered the station along with him, Bhapa flashed his badge and we all got a smart salute by the head chap. Normally he won’t even look at us. Saroj Bhapa has been a father figure to all of us throughout his life.

Prem was the fun brother, there was always the sound of laughter and more laughter whenever he was around. He was always up to something, bringing gifts, going over to Anand Sager Khera house, initiating card sessions, and so on. He was always involved with every aspect of the household and equally concerned for every member. I think he had decided at some point of time that he would not consider any marriage proposal till Kanchan got married. Kanchan got married on 14th December 1959 and it is only after that he agreed to go and meet Shashi at 100 Model Town even though Mataji and Pitaji were after him for the same relation for some time. They got married on 8thOctober 1960.

I always felt that Virinder came into this world to look after everyone. When I was young, I used to feel that it was somewhat unfair that he took so much upon himself, but over the years realized that he was the blessed one. I don’t remember much of him as a young boy, except that whenever I had a problem with my geometry or even math in general, he was the one I would go to, occasionally sometimes Prem. I also remember he is the one who taught me not to scratch my chappals on the road while walking. Did I have a choice though, being have his height?. He was also the one who was directly responsible for me getting admission into Punjab engineering college, Chandigarh. Otherwise I would’ve studied in Patiala. Virinder paid his salary as my fee for engineering college, Chandigarh.

I remember Krishan a lot. As I have already mentioned earlier, he was the royalty. Always smartly dressed, having his exclusive friends circle, always taking care of my needs, which would mostly be either a taka or an Anna. Chavanni or athanni were real events. But once in a while, I got even that from him.

I also remember Gindi a lot. He was mostly busy playing games with his mahalla friends. He had them by the hoards. I think he was the most universally popular guy amongst us, very talented, very good looking. No wonder Dolly fell for him, though she always thanks me for approving her for him when I met them in the public gardens at Amritsar. I am sure if I had said no he would have disowned me.

I remember Kanchan a lot. She was there for quite a while. She was the first one in the family to have traveled abroad. She went to Colombo with her college mates. In addition to the store room, Kanchan brings a lot of fond memories including protecting me from Pitaji when I got caught seeing a movie when I should not have been. She studied in Chandigarh when Pitaji he had his brick kilns business there. 
I don’t remember much about Kanta because she got married before my memory cells started functioning. I do remember my visits to her at Anand Parbat where she lived on the first floor and the house of her in-laws. I also remember the time Neeru had swallowed a marble and had to be rushed to the hospital. This happened at 2 Model Town, and it was such a intense moment.

SHOKI remembers the last day she stayed with Savita and Shoki in Hyderabad. 
“on February 4, 1990 she was laughing with my friends. Surinder-Kamlesh and Ashok-Sudha had come home to meet Mataji. She had very friendly relationship with both. Jokes were being exchanged galore and Mataji was laughing and laughing with tears of laughter in abundance. It was really a memorable sight for Ma and Savita. 

Fifth morning was the departure day. Her train was at 6:30 AM. We reached the station well in time and we settled Mataji and Sushma in their compartment. She says 
‘Mera vada khan nu dil kar riya hai. So I went to the railway restaurant and brought her vada saambar. Savita tells me while you were away Mataji was saying that she does not want to go. I said Mataji, (in Punjabi) get off the train, what difference would it make. Let’s go home. She replied “No, I have promised Kanchan, Bhushan will come to the train station. Let me go.”Rest I think is well known how she pleaded with Bhushan at the train station to let her go home, settle all the baggage and she will come to them very soon. She reached Panipat on 6th February. That night she had dinner with the neighbors where she was in full spirits and was laughing a lot. Next day she went to her garden, picked some brinjals, had a bath, wore a colorful dress. She was cooking the vegetables and had haldi in her hand when she had a cardiac arrest and passed away. I was told when Shakuntala Behan Ji visited our home that evening, she looked up in the sky and exclaimed: Wah Mataji wah.” She also said this is the way any one would wish to die. 

Mataji was at Hyderabad mainly because of the family get together. When Mataji was leaving for the station, she came to Bhavana, Priyanka‘s room, woke them up, hugged them and said goodbye. When we came back from the station, we found both Bhavna and Priyanka writing letters to Mataji. I posted those letters the same day, and they were received in Model Town after my Mataji had already passed away.
 

SAVITA
When Mataji and Pitaji visited us for the first time at Hyderabad Mataji told me “ tum ghar ka kuch kaam nahin karna, main San sambhal loongi, tumhara kaam hai surf Pitaji ke saath taash khelna. Agar un ke saath taash nahin kheloge tho vo Panipat wapis jaane ki rut laga lein ge. I used to play cards with Pitaji. Mataji would take care of all our meals including evening tea, even though she herself did not drink tea. 

Those days Iused to do cooking by reading cookery books. One day I made Upma along with the evening to you. Mataji and Pitaji both enjoyed it and Pitaji noted the recipe in a notebook. A few days later one of Pitaji’s friend, one Mr. Seth who were our neighbors called Pitaji for tea and served Upma. Pitaji felt Upma made by me was much better. When he came home he told me I have called Mr. Shah tomorrow and I want you to make upma. I found it very sweet.

Another cooking experience – I made a sweet dish-banana crumble. First Ashok ate, apparently did not like it. Mataji did not care much for sweets and tactfully avoided it. Pitaji was watching this. He ate the dish and said “bahut achchi bani hai”, I want to have it tomorrow also. I still have his affectionate face in front of my eyes, he didn’t want that I should feel hurt, because main ne itne shauk se banaai thi. 

When Bhavna was about to be born, Mataji and Pitaji visited us. Mataji as usual, took charge of kitchen and I and Pitaji used to play rummy. This time Mataji told him “dopahar ko us ne rest karna hai. Char vaje uth ke khelegi.” Pitaji used to look at his watch and at 4 PM sharp would tell Mataji “chal hun utha de us nu.”When I was in the hospital my neighbor Lilly told me not to worry about Pitaji at all. I will play cards with him every day and she did.

When I got the news that my father was really unwell, it was early morning about eight. I was packing up to go to Indore. Ashok was coming back from the factory after arranging tickets. Mataji made two pranthas and kept feeding me while I was doing my work, following me wherever I went. She had realized that my father will not survive, and that I will not be able to eat for the rest of the day. Thinking of this gesture, I am getting tears in my eyes while writing this. 
After I lost, my father, this was on first January 1986, I used to remain sad. Mataji realized my feelings and supported me in such a way that it is difficult to explain. 
Yaadein itni hain ki book part 2 bhi ban sakta hai. 





URMIL as reported by Arati 

In those days women never went with the barat-so mataji had not gone to Ludhiana for Suraj’s marriage to Urmil. Pitaji had gone accompanied by various brothers-who else? The wedding done, Suraj and Urmil came to Panipat where they were received at the station by Mataji and Pitaji. Suraj got down, greeted his folks, Urmil floored and proceeded to do a bow down pranam to Mataji. MJ blessed her and then called out to Suraj “ Tu noo kithe chchad aaya hai?” This was because Urmil was not bedecked and bejeweled as brides were supposed to be. 

SURAJ as reported by Arati

Having been selected to join the prestigious Jamalpur Railway training college, Pitaji proudly took took him to get an imported gabardine suit made at Multan—Khanewal wasn’t good enough for this!! The suit was proudly worn for the interview which got him into Jamalpur and then worn on his first journey to Jamalpur, with cycle, hockey sticks and tons of food. 
At Jamalpur he decided to deposit them at the cloak room and head straight and report for duty. In his almost new gabardine suit and a red tie he headed for his new home. 
Reaching the workshop he introduced himself as new ‘officer’. All the other boys were in blue, filthy overhauls, doing hands-on work at the shop floor. ‘Officer’ Suraj Luthra ( remember in gabardine suit and red tie) was given a hammer, a piece of iron and summarily instructed to beat the piece of iron to shape. Needless to say Suraj beat his thumb much more than be heat iron piece. His pride did get a beating too. But not as much as when half his mustache was shaved with tooth paste at lunch time. Gabardine suit, red tie and half a mustache —coming down on earth ain’t that easy. 

PREM
We had very few towels and sheets. We used old beaten ones. Properly folded clean were saved for guests. 
Arati had taught few English words to Mata Ji. With time the vocabulary and pronunciation got better. Except she always called Party as Palty. 

Post card letter to Prem dated 29.5.49
Dear Prem
Received your letter after a silence of nearly a month and that only because you needed money which shall be sent through P.N. Bank tomorrow the Monday. Suraj brought a watch for you & could not send it because of your silence & uncertainty of your departure from Sabathu. 
Did you and Narinder apply for the military commission…? 
PREM said that after receiving this letter, the numerous letters he wrote to Pitaji could not compensate for the absence of letters prior to the one received by him in May 1949. 
Prem’s random memories:
We all had good card sense—genes
He learnt Sweep and Rang Mallan from Lala Ji and party. 
Availability of loose change in the trunks, suitcases in Khanewal was always there. The loot, disclosed after buying Tosha, the best sweet ever, from Parsaram or drink of lemonade or banana sold in glass bottles with a marble stopper. Each used to cost one Dhela, equal to less than half paisa. 

PREM got his first attack of backache in Panipat in 1962, at the age of 31. While putting on, the pants slipped. He bent to pull it, could not straighten due to acute pain. It was called’ chum’ ( slipped disc or pulled muscle.) The traditional doctor for this ailment at time used to be a person who was born upside down (breech delivery.) Pitaji took Prem to such a ‘doctor’ who would either jolt the back, rub it with a broom or blow air from mouth. Pain subsided in 2-3 days. His backache became a periodic headache for him through out his life. 

PREM wrote “Whenever I think of my Mataji, upper most, I relate her to the following:
1.	Cooking
2.	Household work
3.	Worldly wisdom 
4.	Personal relationships.
There were many more traits in Mataji. 

1.	Cooking: Mataji was a fantastic cook, producing both quality and quantity. One cannot know if she had talent or she reached that level of quality and variety by producing day in and day out helping the quality. Mataji was perfect for both quality and quantity. Her pranthas were famous far and wide. She loved to make them and serve them to a large large variety of guests—youngsters, elders and everyone in between. So many must remember her simply for her trait of being a warm and wonderful living cook and a generous host. 
She always took special care of for the guests even when they were her own children or grandchildren. She knew their tests and likings. For example she made saag and pranthas on Suraj’s visit. Bhindi, karele, baingan Bharathi were my demands. I liked so many things. 
She accepted and performed the role beautifully. She taught cooking many dishes to many people. 
1.	Household work was a handful and it kept her. busy from morning to night. Virinder has beautifully described his one day of her life. 
2.	Worldly wisdom and personal relationships. She had good relations with neighbors. In Calcutta she shopped for her friends in Panipat. “In case your Pitaji dies before me, these are the friends who will help me.” 
3.	She had individual relations with all children and gran children. She consulted Suraj, Virinder, Krishan and above all, Shakuntla Ma before taking any major decision. She faced widowhood boldly but loss of Krishan left a hole in her heart. She relished laughing and enjoyed the moment. Tears of joy flowed easily, transmitted to Prem as well. 

SHASHI
I met Mataji when I was in my first year in SD college Ambala. I had to come home and my sister Shanta was visiting from Delhi. Shanta Behan Ji knew my Mataji from Khanewal and she wanted to meet Mata Ji. So she took me along. We went to 2 Model Town. There we came to know that Mataji was in nine model down with Shakuntala behan Ji who later to be reverently called Shakuntla Ma or jus Ma. Shanta had to return to Delhi so we went to nine Model Town. Mata Ji was busy organizing. Swami ji had come and was meeting devotees who wanted to take Diksja. Mataji must have met Shanta and me. Next I realize that I am in a room in the presence of a dynamic personality, who is sitting on the opposite end of the room with his back to the wall. We all sit in a line facing him, and as luck would have it, I was the first in line in the line and quite close to Swami Ji. Before I knew who was this person and what was happening I had become his disciple. I am indebted to Mataji for guiding me into that room. Today I met two great personalities in my life on the same day. Who could imagine that one day one will be my mother-in-law and the other my guru. Both of have been and still are my guiding force. One is showing me the path of every day life and the other taking me close to the ultimate goal.

I got engaged on 16th December 1959, two days after Kanchan‘s wedding. When the mango season set in Mataji called me to 2 model down. When I got there he pointed to a mango on the tree and told me that it has been kept for me to pluck and eat. It was an easy task for me and probably Mataji had come to know by then that I could climb trees like a monkey. In no time I was up on the tree and about to pluck the mango when Mataji heard me shouting “meri chunni, meri chunni.”Before climbing I had taken off my dupatta and I was about to pluck the mango I saw Pitaji looking up at me. I don’t remember from where he appeared so suddenly. In those days daughter-in-law was supposed to have their heads covered or wear a dupatta in front of sasur Ji! Very sweetly and in an unruffled manner I heard my sasur Ji say “tere Pita Ji nein tainu badhi vaari sau number de bageeche vich bina chunni de vaikhya hai. Thu chunni nu chorh aur umb thorh .” So, after that day I was a free bird. Chunni or no chunni, before, or after marriage, it did not matter. A proof of how broad minded, Mataji and Pitaji were.

For Mataji Shakuntla Ma was everything-her friend, her guide, her ‘Shehanshah.’ She always used to say “Shahnshah da jo hukam hosi.” Meaning she would always consult Behanji before any major decision had to be taken at home. Behan Ji was also a great fan of Mata Ji. Actually they were friends. Later on Behan Ji reached great spiritual heights, but like Sudama and Krishan these two ladies remained friends. Hence the doors of 9 Model Town were open for Mata Ji 24 hours a day. Mataji would send us children to Behngor small things like “saag de ke aao, tandoor ki garam garam roti dene jao.” Prem kalkatte thon aaya hai, mera dil karda pya hai aj satsang de baad panj mint meri kutiya tho honde jana. Ye ja ke Behan Ji ko bol kar aao.” This used to be a ritual for many years after our marriage. Then suddenly Mataji changed. On asking she would say “Hun Behan ji kol time nahin. Unhan ne bahutan nu vekhna karna honda. Hun pichche hatna hee theek hai.” She never ever had a grudge that she no more had access to her Shehnshah anymore. 

Mataji was a very practical person. On our wedding day we had to reach nine model down for Behan Ji’s blessings. The marriage ceremony was cut short so that we could reach there at the given time.

Pitaji was a quiet person, but full of love, affection and concern. He felt very happy when his children were around. I remember playing my first holi with him. When he approached when we approached him sitting on the Munji , he would pick up the envelop of gulal and laugh and laugh, applying gulal on us. 

He used to play a lot with Ashit. His cliché was “ O soor de putra.” Whether he was happy or angry with Ashit he would say his pet sentence “soor da putter, aa thenu vekhaan.” 

Mataji and Pita Ji visited Calcutta almost every year or two. They both enjoyed the winter months in Calcutta in house number 7 Bakery Road. Both enjoyed their stay very much. Pitaji went to the office with Prem and Mataji having made friends with the local ladies. She got a tandoor made, and would invite the ladies to cook rotis. The grandson Ashit got all the love. He was there “pota Ashit, aanhon ka tara.” . Shashi Prem were nowhere close to ‘Ladla mera chun.’ Till today Ashit loves them and respects them from the core of his heart.





DOLLY
My thanks go to my only Dewar Shoki who made me part of this beautiful, loving, considerate, compassionate Family. I would not be married to your brother if YOU had not given your O.K. 
My mother was a worrier as all mothers are. She would say “ How Will Dolly adjust to a large family. She is so quiet and shy and this family is so loud and noisy.” Well, Mummy wherever you are, you will be very happy to know that your daughter is blessed that she is in this FAMILY. I wish you had produced more children. 

The other thing which made me nervous was that since I was chosen by J.K. and it was not arranged marriage like others in the family, will I be accepted and loved as other daughter-in-laws. I am going to tell you this memory which is so fresh in my mind as if it happened yesterday only. After marriage we reached Panipat on 2nd October about 3 P.M. when I entered # 1 gate MJ and PJ were standing to receive us. I touched their feet and Pita Ji said to me “Kudhian kolo assi pairan nu huth nahin lagwaande.” For Namita, Anil, Rohini— the meaning is we don’t let our daughters touch our feet. 
It made me feel so good, I have no words to express the feeling of clove and security I felt at that time. 

The other thing where I felt so good was when the whole family would play Paploo and Aloo, Matar, Gobhi and since there were so many people and not enough names for vegetables we would come up with some funny names like Chachanda. Remember Virinder Bhapa. It is making me smile as I write this. 

This is the second installment of my memories of 2 Model Town. 3rd day of my entrance to the Luthra Family, 3rd October 1968. Early in the morning everybody is ready to go, you know where? Well when I realized the latrines are different than what I am used to go, I controlled. At least two days till I could not help. 
Lunch time that day is the most memorable day. Mata Ji was making Pranthas. She would serve hot, straight from the Tava. She gave me one and served it with cold Kali dal. First I thought may be she forgot to warm it up. There was a big culture shock also because back home our eating habits were different. Sitting on the dining table, eating with a knife and fork, without making any cluttering noise, since my father had been to England and Germany he wanted to practice western way of life. I used to feel I am living in a very disciplined way. Any way eating Garam Prantha with cold Dal is my favorite now. Salomi, Urmil Bhabi and Savita can vouch for it. Jo maza is mein hai woh dining table par kahan. 
The joy of eating is lot more than sitting on a dining table. Well, living in this country for last 27 years especially since children left home, myself and Gindi take our food plates in the Family room and eat our dinner watching either Indian movie or a serial.
Whenever I eat cold Dal, Saag, or some Sabzi I remember MJ. 

Third installment:
Atari mentioned that in USA one needs to have more than 24 hours in a day and four hands to keep up with everything. We must thank Raju and Rashmi for starting this forum to share memories. 

My mother, Sheilly Sachdev was a worrier as most mothers are. She would say “How Will Dolly adjust to a large family. She is so quiet and shy. And this family is so loud and noisy.” Well, Mummy wherever you are, you will be happy to know that your daughter is blessed that she is in THIS FAMILY. I wish you had produced more children! 


CHANDER

From where I should start the day I came in this house, it is so fresh in my heart that it looks like it is the talk of today. I made Halwa (actually Kanta Behanji helped me.) I did not know anything. Every person in the Satsang knew that Halwa was very good. 
•	I for the first time got my hair cut as Krishan wanted it that way, I used to cover my hair. For 2-3 days Mata Ji-Pita Ji we’re watching, then one day she said to Pita Ji in front of me—“Chander pehle ta sadee dhee si hun nu ban ke aai hai.” I just took my Pallavi off immediately. 
•	Krishan always wanted me to wear bright colours. Once Pita Ji sarees foe everyone from Jodhpurs. Mata Ji said “Chander your Pita Ji specially bought this for you and that was such a bright and beautiful saree. 
•	For sometime Mata Ji was fed up with Pita Ji’s friends, chess and cards. In the morning Card session and 2-4 chess and Mata Ji didn’t have time to talk to him. But Pita Ji understood her problem so fast that he stopped playing chess. He was very punctual in his routine as everyone w we ours know. At 3 O’clock in the afternoon he would be on his moorha having long mala. 
•	Once Pita Ji was telling Prem Bhapa at Panipat, I hope you remember Bhapa. It was summer days, he said “Savere ta twadi mama khwandi e 3-4 sabzi a par rati puchegi rhodium bhindi te thode tinde pai be kuch hor na banwana. Pyaz nal kha lawoge na.”
•	While going from Delhi once , my father came to Panipat to take me to Ludhiana for Rakhi but Mata Ji refused to send so children may stay for some more time, I could not say a word. But was feeling sad. In the evening Mata Ji made Dhuli Miongi Dal. At dinner time I was going to make chapatis and was to serve to everyone. I saw one chipkali (lizard) which fell from Dhookash in the dal. I felt so nice that I didn’t go that day that Dhookash was closed after that. 
•	Everyone in the family knows that Mata Ji had much interest in the kitchen and machine as Pita Ji in tash (cards). It is about 30 years before I am talking about. Pitaji on his bed and Mataji on machine. I was on Mataji’s bed. Pitaji playing patience, Mataji said “je tusi pehle gai te twanu tash dwangi” for which Pitaji said “je tusi pehle gai te twanu sui dhaga dwan ga” and that topic was so long enough we were laughing. 
•	When Pitaji went on 27th Dec. 76, Krishan and I were staying at Panipat as children had holidays for a month. I cannot forget 
NEERU

NISHI

SANJAY 

VIVEK
He is always called Vicky
Prem recalls young Vicky used to call isn’t it as ijhn’t. 
Mataji jokingly would get into fights with young Vicky and tease him and enjoyed getting beaten up by him. Prem quotes Vicky “We will take a crane and put Nani in Hooghly river.” 

VINEET
I don’t remember Pitaji as a stern person. One time while he was doing soo soo, I peeped in and was pushed by Pitaji. 
I remember Naniji a lot more, and loved her spirit, sense of fairness and aloo pranthas! I could also confide in her about almost anything. Nani Ji would either laugh or have some good advice, always to the point. And I do see a lot of Nani Ji in Ma and Kanta Masi b


ARATI
I don’t remember being part of the green window paint scratching outside the newly wed’s room incidence. I must have been been being good like I always am-or may never. 

Urmil as reported by Arati— In those days women never went with the baaraat– so Mataji had not gone to Ludhiana for Suraj’s marriage to Urmil. Pitaji had gone, accompanied by various brothers – who else? The wedding done, Suraj and Urmil came to Panipat where they were received at the station by MJ and 
P J. Suraj got down, greeted his folks, Urmil followed and proceeded to do a bow down pranam to MJ. MJ blessed her and then called out to Suraj. “Tu noo kithe chad aaya hai?” This was because Urmil was not bedecked and bejeweled, as brides were supposed to be.

SURAJ
Having been selected to join the prestigious Jamalpur Railway training College, Pitaji’s proudly took him to get an imported gabardine suit made it Multan…Khanewal was not good enough for this. This suit was proudly worn for the interview which got him into Jamalpur. He wore the suit on his first journey to Jamalpur, with cycle, hockey sticks and tons of food.
At Jamalpur he decided to deposit the luggage at the Cloak room and head straight to report for duty. In his almost new gabardine suit and a red tie, he headed for his new home.

Reaching the workshop he introduced himself as the new officer. All the other boys were in blue, filthy overhauls, doing hands – on work at the shop floor. Officer, Suraj Luthra, in gabardine suit and red tie, was given a hammer, a piece of iron and summarily instructed to head into the workshop and beat the piece of iron to shape. Needless to say Suraj beat his thumb much more than he beat the iron piece. His pride, duly dressed in gabardine suit and red tie did get a beating too but not as much as when half his mustache was shaved with toothpaste at lunchtime. Gabardine suit, red tie, and half a mustache… Coming down to earth, isn’t that easy.


ASHIT 
One of Ashit’s memories as described by him—
I was small and selfish I think. I would try my best not to let mommy stop and meet her parents at 100 model town and urged the ricksha wala to ride straight to 2 Model Town. Not that there was any dearth of love and caring on the part of Nani and Nana, but once on Panipat soil 2 model town beckoned too strongly. As we passed Bosa Ram chowk, I would look for three things…the pakoras, and the milk cake with makhees on Bosram’s counter, Suraj ki dukan on the right and Pawa’s shop on the left. There was a sense of great relief if all three were in place. Then as we approached Kala Singh’s dispensary I would frantically pray that mummy not mention a visit to him for a ‘desi totka’. Further ahead at the fork in the road I would always tell the ricksha wala “Bhaiya jis taraf se jaldi pahunchen ge us taraf chalo.” A difference of mere seconds no matter whether he went from the left or the right of the park. So can you really blame me for not wanting to stop at 100 model town? 

I would be off the rickshaw before it came to a halt. I couldn’t be bothered to handle the luggage or what happened to it. Dashing and opening gate # 1, pulling at the jaali wala Wala darwaza right in front, which would often be bolted unless Pita Ji was having a card session. Next would be a more inpatient tug at the kitchen door. Finding that shut, a final sprint to the main door.

Loud greetings, pairi ponas, tight hugs, and I will be off to savor the flavor of the season. I would scurry around the place, just like those chuhiyas. There was no fun without those little mice. We had to chase them with a broom, and we had to devise ways to trap them in the ‘chuhe dani to observe them at close quarters, and finally release them somewhere outside. Once we nabbed a big one, and took him and flung him far into a pond near the railway tracks.
Mataji used to say “ve sirsadiya do mint baith te ja.” But where was the time for that. The Jhoola had to be jhooloed, the trees had to be inspected, the vegetable patch on the far side of the house explored, the roof had to be climbed either from the ladder, or as we grow older from the bathroom door. 

Bosaram was waiting with free goodies, lotteries which yielded two leaky balloons had to be played. Paws was a confirmed chor, but we had to go to the shop too because he ran a different sort of lottery and more expensive—a dice with different colored faces, and you could choose your prize from the color you landed. None was worth getting back home, but nonetheless we were repeatedly drawn to it. 

There used to be sparrows inside the house particularly in the morning and this is how we used to try and catch them. When a few were inside, we would shut the three drawing room doors and also the roshandans if they were open. One person would prod them with a broomstick or stick and another would be sitting by the jaali wali windows. In their bed to escape, the sparrows would seek the jali ka window route. And once they were on the jali the person waiting would try and shut the window before the sparrow got out of there again. If a sparrow was trapped in that little space, we, mostly Vijay, would open the window, just enough to slide the arm in and try and catch it. I think he was quite good at this. The last time we tried catching a sparrow in this manner was when one got crushed while the window was being slammed shut! Then we shifted to catching birds outside the house. It was less gory, as well as there was a wider variety of catch on offer. A wicker basket was placed upside down and held open with a vertical stick, tied to a long string. The bait, wheat or rice was kept under the basket and we the string holders sat on munji some distance away and waited. When a bird came under the canopy we pulled the string and the basket dropped. More often than not after the bird had escaped! Small wonder because Luthras have never been good at pulling strings, at least not in a way that is detrimental or harmful to someone else. Babela and Mahinder were the accomplices. 

ASHIT in another moment of bubbling memories remembers that in those grand old days, the grand trunk (G T) Rd appeared no wider than an elephant’s trunk. You could almost stick your arm and shake hands with someone traveling in the bus overtaking yours. And mummy used to keep nagging every two minutes haath andar karo for fear that it might be carried back to Delhi by a bus traveling in the opposite direction. That’s how narrow GT Road was and that’s how crazily close speeding vehicles used to come to one another. Yes, the roads were rickety, so were the buses from their constant too, and fro travels between Delhi and places further north. It was safest to shut your eyes and let the driver worry about the oncoming traffic. There was and still is plenty of it. There was no divider between the two lanes. But how could I shut my eyes. What would happen if I was asleep when we passed by Kamal or Kishore or Naval and one more, the four movie theaters of Panipat.

It could be traumatic if such a thing happened and to avoid such a possibility, Mataji and Pitaji made sure that they knew the names of the movies, their respective theaters and show timings, backwards! If they knew we were arriving around showtime, Vijay, the only servant, I remember, would have to be kept free and ready or it would turn into a tantrum time. I wish I wasn’t even a movie buff, but there was always something irrepressible exciting about going for movies in Panipat. Maybe it was the noise in the theater, the Chana badaam vendors, whose access into theaters, wasn’t always restricted to the interval, the fun of once in a while being front benches and the seetis and gaalis, the shortcut over the railroad tracks to get there quickly, always fearful of the ‘house full’ sign, even when ultimately there would be only about 10 people and a few cats, rats and cockroach inside. The mobile, audio-visual ads. Those men in the rikshas barely visible (they were covered by those pictures picture hoardings) from left to right and above. It provided good protection from the almost noon time sun. That’s when they usually arrived and that was our morning alarm. We often used to follow them till wherever we were allowed to venture on our own. They were blaringly audible—Aaj se dekhiye, Kamal pe, Devanand ek naye role mein, Johnny Mera Naam. The announcement was followed by snippets of songs on loud speaker. That was quite a thrill.

The welcoming flavor created by Mataji and Pitaji varied with the season although things like the Jhoola, pranthas and pyaar were the unshakable constants. Actually, the harder, the Jhoola shook the more thrilling it got. The crashing of our bums against the jaali door of the mandir and then trying to pluck the mangoes on the lower branches or flowers from bougainvillea plant. The competition of who can jump the farthest, jumping with bad judgment when the Jhoola beginning it’s descent and landing on our backs; trying tricks, like jumping over in munji, (cot) jumping onto a munji, jumping to the side of the munji, trying to show off, by not holding the ropes and tumbling out and splitting open my head. I wonder how Mataji, Pitaji and other elders felt when they saw us do all of this. I would get multiple heart attacks if my kids were to do all that!

Coming back to the welcoming flavor, in the summer, mangoes used to be designated to be picked by various grand children. Aye Rajoo de nein, te aye bable de…So on and so forth. The ladder was in place, leaning against the trunk. We were allowed to climb the lower branches, but only under strict supervision of Pitaji and Mataji. For the higher branches, and for the less daring there was that long, wooden pole with the hook to yank off the mangoes from the higher branches. Almost all the reachable ones were covered with thin white cloth (Mataji’s old saris or old sheets) to protect them from the parrots. The other smaller mango trees were to be conquered as well, but this was the favorite. The bunches of grapes were covered too. I never enjoyed plucking the grapes because it did not involve action, but given the chance that’s the first thing I would now, because now I am able to understand what sort of care and effort went into protecting the delicate fruit. There were Amrood trees, apples, nimbus, jamuns, shahtoot…

When water was abundant, digging implements were ready for us to make ‘kuana’ (wells) under the big mango tree. Even here we would compete to dig the deepest well. The water was supplied mostly by a pipe connected to the nearby bathroom tap or the wash basin tap to the right of the main door where Pitaji used to shave. 


VIKI


SUNANDINI remembers being part of paint peeling incidence. 

During Mataji’s stay in Bombay with Neeru and Sunandini, she like to socialize to the extent that the two grand daughters have he a Nick name of Titli (Butterfly) 

ASHIT also remembers same incidence. 
At Shashi Prem wedding Banwari Kohli and his wife Chaman put about seven of those toys that squeak under the bedsheet. Mataji used to call her office as kitchen. 
He remembers that his fondest memory of Pitaji’s winter after shaves in Panipat. He had bright smile and sprinkling eyes when he cupped his cheeks with his freezing hands after his shave in front of the outside mirror between the main entrance and the bathroom. The fact that he did this often is a sure sign that I enjoyed it immensely. It is hard for you to imagine how much warmth flowed out of those freezing hands. Babla was the devil. He used to imitate Pitaji and his friends and have me in splits of laughter. The way Pitaji slurped his kharbooza and the way his hand would come down in an arc and slam a playing card in the playing arena in a victorious voice “ ai lao meri trikki. For some reason we were most amused by trikki, even though there were the dukkis, panjees and ikkas .

RITU

UMAMG

RAJAN
Not possessing a good memory, my memories of Pitaji are not very many. And the ones that are there are at Do Number Model Town,Panipat-an address that brings to mind a lot of differing memories, depending upon the time about which one thinks. But more about that later.

One thing that I do remember about Pitaji and his interaction with me is about, what else, playing cards. I still clearly see myself seated on the charpai under the famous mango tree with both Mataji and Pitaji seated along with me. I’m about 3 to 4 years of age and Pitaji is teaching me how to count with the cards. I believe that counting for me used to start with an A and not a 1.

All other memories about Pitaji are just brief glimpses of him playing cards, sitting on the Moora or divan in the dining room with his frien by ds who are wearing big pagris. I also remember sitting in the kitchen on the stool, along with Pitaji and Mataji serving freshly cooked chapatis.

Unfortunately, what I associate with Pitaji most vividly is the days surrounding his demise. it was the first time I recollect having been to a funeral of someone else. I remember Papa coming back from Office when we were in Kasauli in 1976 and saying saying something to Mummy that we children could not hear. Next scene is we all trying to sleep in the train which we got from Kalka around midnight, by when my mind told me that while we were going to Panipat, this time something was not right. Next day, going along with the family, a brief stop at a place near the railway line crossing just before reaching the place where they lit the pyre. This ‘place’ has continued to haunt me since that day. The other two visits have not helped matters either.

My memories of Mataji are many. Surprisingly, in most of them I find myself alone with her at Do Number [it can never be 2MT]. The most vivid ones going through my mind right now, I am trying to put down in words.

Having her ‘hath ka bana hua aaloo ka pranthas’ with lots of ‘makhkhan while I am sitting on the dining table outside the kitchen. Sitting with her in her favorite place – the store – where she is packing/ re-packing in torn pieces of newspaper the not so ripe mangoes with each mango having a family member’s name associated with it.

A favorite one which refreshes the heart is being with her early in the morning, plucking the small white flowers that have a wonderful smell – forget the name. These used to go to the Puja room, the most sacred place in the house in front of Pitaji’s photo.

Her laugh, though rare, [I mean the real one, and not just a grin a], was so cheerful and almost always halted by some comment in Punjabi to the person/the thing that had made her laugh out wide.

Smt. Vidyawati Luthra, 2-L Model Town, Panipat, Haryana, India

This is the address of Do Number, as I remember, having written on the few occasions that I posted a letter to Mataji. As is obvious from all that has been written about it, Do Number was an amazing place where so many of us had some of our fondest memories. The most cherished ones include climbing the mango and jamun trees, going to the roof, just to look around, watch the kites flying, getting the jhoola put up in the slot for it outside the Puja room, the wonderful, creaking sound of the front metal gate, which was low enough for children to open bringing you to the place where you felt you ‘belonged’ and many more. 

One particular period where I do remember lots of family members at Do Number was the time of the 1st/2nd get together. Among those whom I associate more than the others during those days include Umang, Rohit, Prem Chacha, Bhushan uncle, Virinder Chacha, Mataji and Papa. I am sure all the Chachas/Chachis, Ruchi, Mummy Within door number. Chinese cricket was also popular I think – I’m sure Umang will remember. There was also some international match going on and we had a transistor phone most of the time. The amount of Murli’s particularly and salad in general eating eaten during the daytime was phenomenal. It was hwonderful having dinner out in the open under the shadowed copied., and most of my generations were there as well.

We played lots of cricket that time, mostly mostly within the number. Chinese cricket was also popular, I think – I’m sure Umang will remember. There was also some international match going on and we had the transistor on most of the time. The amount of ‘moolis’ in particular and salad in general eaten during the daytime was phenomenal. It was wonderful having dinner out in the open under the ‘shehtoot ka ped with, Mataji in full swing at the tandoor.’ And sleeping on the ‘navaar ki manji’ under the stars was heavenly, despite all the odomos having to be rubbed all over the hands and feet. This was the time when Prem Chacha introduced in-between to the next generation, and my own first, I think, of playing cards with money. Lots of fun and laughter with frequent joke sessions. Just about the most memorable time I’ve ever had. Those older than me at the time and even now please correct me if my memory has been adding/mixing up some wishful parts just to bring it to the level of happiness that you are you associate with those few days. And I would love to hear more detailed account of this particular get together from anyone willing to put in the effort. Prem Chacha– did you happen to write about this in your letters before or immediately after this.

However, Do Number always brings with it my worst nightmare – going over and over the reaction of Ruchi and Rohit‘s faces when they entered the place on 13th October 1987, and Mummy broke the news to them. This was much worse than actually seeing Papa in the hospital the previous day. 

I had thought of not writing the above, but felt maybe sharing it with all will help getting over with it some day. 






RUCHI
I don’t remember much about Pitaji, just one time in Himachal, I guess he got a bucket full of mangoes. It was raining and we were out in the rain. Instead of asking us kids to go inside, he joined us. We ate mangoes in the rain. Mataji made all of us feel special in our own ways. I remember her visit in Kasauli, Una, especially her stories before going to bed. 
The visits to Panipat were really special. We were eager to reach 2 Model Town. There was a distinct feeling about that place. I loved her pranthas with mangoes. I somehow got very close to her after Papa. I spent 15-20 days with her in 1989, just me, Mataji and Sushma. It was before my second year examination. I enjoyed every day spent with her. 

On the last day when I was to go back, she asked me what all the sweets I like. I said milk cake and kaddu (squash) halva. She made kaddu halva just for me, though she had never made it before. It tasted so good, she definitely put her love in whatever she made, that was her secret ingredient. There is so much I didn’t know about Pitaji and Mataji and everyone else. Thanks for sharing. 
I don’t know how much Mataji did such a wonderful job of bringing up eight kids, I get tired with two.

ROHIT

NAMITA

ROHINI

RASHMI

BHAVNA
My recollections which are vivid are of Mataji in Panipat house. I remember many details of that home. The gate painted grey, the house painted yellow. Cannot forget the mandir area next to which was a utility room. That was my favorite area coz of the jhula being there. Then there was the house, of course, with the attached bathroom. I used to enjoy seeing the frothy water flowing out of the open drain and still remember the metal bucket and mug which was of transparent plastic, with Hamam soap and sarson ka tel lying around. The latrine was a little scary as it was away and I was always looking out for lizards. 
I remember one of Matji’s servant boy-Channi. He was flat nosed., loud spoken chap and only slightly elder to me and we were playmates. I remember he was the first person to expose us to a new breed of firecrackers ‘Bombs’ which my parents never used to buy on Diwali. All these years I had been terrified of bombs and laddis and watching them burst. This was the first time I got to light them while they were still in my hands and then throw them away and watch them burst. Channi used to get a blasting from Mataji as he was always dodging the house work and playing with us instead. Priyanka and I used to play around a lot in the clay mud in front of the mandir where mango tree was. 

PRIYANKA 

MataJi used to make hot pranthas with white butter. When she visited Hyderabad last, she was given my room. She used to sit there on a red chowki which she carried with her wherever she went in the house. Mataji was very protective of Sushma. Since Sushma and I were same age we both used to play as well as fight a lot…but Mataji often used to scold me for fighting with Sushma. Sushi’s and I used to sit down while Mataji sat on the red chowki and narrated stories to us. 
She also used to give me money so I could go to bakery nearby and buy sweets and chocolates.

I have preserved a few things of Mataji—
One shiny white hair, a few coins and a lacy bit out of het blouse. 

Didi and I used to plait her hair everyday. We even shampooed it once. The plait used to get thinner and thinner downwards and we used to ask her if there was any way she could make her hair thicker. She had a hearty laugh and then called us ‘Sirmunnis’.







MANDY

We had studied about battles of Panipat in school. In 1970-71 I remember walking about in the field in front of the house (where the water tank is) and looking and looking gor remnants of the battle-it was enough that I was in Panipat and in a field that looked to an 8 years old like a battle field, never mind that the last battle had happened a few hundred years ago. And being very disappointed that my search yielded nothing. 
Then there was Shoki mama’s marriage in 1972. Viki had high fever treated by Arati with cold sponges. 
Babela and I tried in vain to scratch the green paint off the windows paint to peep into Shoki Mama’s bedroom. 
Neeru didi-being the eldest among us was given talk of churning malai to male Makhan. Sunandi and I hated malai and felt like vomiting at the sight of it. Nishi didi told me that wasn’t nice since Neeru didi was up to her arms in it. 
Neeru didi and Nishi didi were glamorous older cousins. When dressed up many people commented that they looked like Rakhi and Zeenat Aman. I remember having a photo of the two of them together and showing off at school. 

After finishing school I went to Panipat with Babla and Kanta Masi. Chander mami was there with very fat, very fair Rohit. Babla and I scared him in the TV room with a Lion/tiger( a decoration piece). Chander mami would say-very mildly “ hai, kyon rulate ho is ko.” 
We all except MataJi to see a film. As soon as light went out in the hall, Chander mami started passing out pranthas with bhindi rolled in them. Picnic in the movie hall. 
From college I regret not visiting MataJi more. One time I picked up an overnight bag and showed up there. MataJi was sitting on a chowki by her bed and Mahinder was massaging her arms. As I walked in he said to her “ see, every time you get sad because someone leaves, someone else shows up.” Savita mami or Chander mami had left that morning. She told me she was feeling a bit low. In fact every time she felt low because someone left, someone else arrived. 
My fondest memory from college days is of Ma Sunandi, I and Kanta Masi being at 2 MT together and MataJi, Masi and Ma would sit around her bed (while Sunandi and I would watch TV in the living room) and they’d talk and talk and laugh and laugh. It was a warm and comforting feeling-listening to mum and daughters chat and laugh together…

I do remember ganne da rasas. In railway colony Nishi scolded us (Arati) a lot for the aspirin and coke. 

I also remembers Marlboro and being shown by Gindi mama how to inhale the awful Winstons. 


NAMITA

Love

(Memories of Panipat in the 1980s. Nearly every sentence is a question mark as much as a period.)

A block-like home, peeled, faded paint. Entrance, small covered open area before entering home. Semi-detached with another low, matching structure that was someone else’s home. A family I don’t ever remember meeting. Walk through a formal room that no one seemed to use straight through to the dining/kitchen area. That and one other room is where everyone spent nearly all of our time. In that room too was a bed along the wall where Dadi slept. Just near the kitchen and the bathroom.


When we would arrive our luggage was put in the large formal bedroom and we would take over that whole space. Wide berth for a wide American family. That room felt quite dark. Two wide beds shoved up against each other in the middle of the room. Steel almaris along the borders of the wall. All locked for safe keeping of bed linens, blankets, towels, clothes and whatever else lay inside.

In the middle room between the bedroom and the dining area was the room that was filled with all of us. Sleeping happening in all manner of space at night, singing and talks with what felt like 35 people crammed in sitting, lying at all levels, on the floor, on razais, on “sofas.” Lots of sounds and discussions happening simultaneously. Getting swept up in the Zindabhads! going around the room, spotlighting each person and celebrating them. (Lift Every Voice and Sing)

When this room wasn’t filled to the brim, I would spend time there quietly by myself and look at the glassed-in, formal book case and display case. There on the lowest shelf, you’d find framed, black and white photographs of every married couple. Suraj and Urmil, Prem and Shashi, Virinder and Santosh, Krishan and Chander, and so on. They didn’t look like the chachas and chacahis I saw every day in rumpled salvaar kameezes, hair unkempt, relaxed, aging. There, in the black and whites probably taken on honeymoons, they looked young, polished, slim and trim, “perfect” at the beginning of their married lives. Not to say that the fun-loving, fuller chachas and chachis in person were any less, in fact the full fun-level in the house was off the charts, but I do remember noticing the sharp contrast between those photographs and the reality. I also noticed that someone, probably Dadi, had taken great care and effort to frame those photographs and collect them on one shelf. There weren’t any many other photos around so those were special. That room was rarely quiet and empty, so having time there by myself felt special. Perhaps there was a large, framed photograph of Dada high on a wall. Garlanded? A special puja we’d do for him. Beneficent, smiling, a twinkle in his eye. Father or grandfather of his huge lot.

The kitchen was a small box-like space, seemed enclosed. With a door? That’s where Dadi and her servant spent much of her time. Squatting on the floor, cooking, laboring with love to create dishes that pleased all those 30-odd people in her family. She didn’t seem to mind and her cooking was delicious and everyone told her so. All the love she poured into the fresh, hot parathas and the cold kali daal from the day before elicited loud raves and equal measure of love in return. But I do remember thinking that the place where she spent so much of her day pretty much on the floor didn't get much sunlight or any view.

In fact, right opposite the kitchen, quite removed by space was the outdoor latrine with an old-fashioned squatting Indian toilet that I suppose we all used. Now you’ll find “new research” that this method is better and more natural for the body. No kidding. We all shared it. We Americans probably didn’t like it, but no one really complained.

Here, I do remember noting the sharp contrast between Nani and Nana’s modern, more Western-style home – very clean, tidy, organized, minimal, quiet, structured, mannered, everything in its place, English or German-influenced – and this home. Nothing seemingly in its place, loud, boisterous, everyone sharing one latrine and one, small outdoor sink stuck on an outdoor wall, 17 different things happening at any one time. 17 different conversations or children’s games that one could join. All happening all at once.

I also remember thinking that Dadi slept on this small bed shoved up against the wall taking up a sliver of a space in a pretty large home. Not the formal bedroom, not the formal sitting room out front that sat empty unless hosting formal company.

She would try to sit with me and have a conversation. We would piece together words that made up an exchange of sorts. School, friends, food, teachers. We weren’t really exchanging thoughts or feelings, but we were sharing space and time sitting together. Her and me, one-on-one.

Outside the two wooden, screened doors that slammed all day long with people entering and exiting was “the backyard,” a cemented or stoned floor lined with trees and lush plants. The amrood tree was my favorite. I couldn’t believe that you could get the world’s best fruit hanging for free by just plucking one and eating it. That was an idea pretty close to heaven. Not all were ripe but I, like Dad, loved kachcha ones as much as the ripe ones.

Here, maybe four large outdoor charpoys were set up in diagonals and no formation at all, just strewn about, moved and re-moved to people’s needs. People sitting and chatting, often chachis shelling peas from fresh pea plants that had been bought that morning or shucking corn. Again, lots of conversation happening in Punjabi about this person or that, this event, or that update. Sometimes I’d join in to listen.

The cousins were a mass organism of ages from near-adult all the way down to bachaas. Lots of running and chasing and games and laughing. We three from America slipped right in. At any one given time maybe 17 cousins in all. Playing outside and then inside in a circle for Murder. The range in age always struck me because the older ones – Nishi, Aarti, Mandy, Sandy, Ashit – were in a different league than the youngest like Rohit and Umang. Different conversations, different worlds. But all of us together.

There would be climbing the ladder with a chacha’s help up to the roof. Scary to climb up and scary too to climb down backwards. New view from up there of the neighborhood with its homes in a row, open naalis running along the sides collecting sewage and refuse. Dusty play area where local kids would be playing cricket or running around opposite the house. Did we interact and play with the local kids? Only a little.

When I’d play by myself I remember loving to make a big soup in some sort of caldron with a mixing stick. In it I’d put this leaf or that I cut up by my fingers. I was concocting some sort of stew and mixing it together over a pretend fire. I was playing house. Something I liked to do in Weirton, too, over the waterfall.

Another favorite spot was the swing. It sat in the doorway of another building structure that sat off to itself that I can only think housed storage. I don’t think anyone ever went in there. But the swing was there and it got used, a lot. All of us must have shared it, waited patiently for our turn. I didn’t remember any fights over the swing or anything else. No fights over food or “he did this to me” or “she said that to me.” I don’t remember bike riding out in the lane in front of the house or in the enclosed courtyard behind the house. But maybe there was a bike or two.

Just lots of fun, wild playing. We would probably drop in for five days and nights at a time. We would bring five huge hard-covered plastic-shell suitcases on two small, rickety and difficult-to-maneuver wheels that carried gifts that Mom would collect over the year. Christmas Day in the main sitting room where lipsticks and jeans and small objects from a land so far away and so different would be handed out. A land none of them had ever visited but everyone on earth knew. We would go back with those same suitcases empty. All the toilet paper and cans of Chef Boyardee used up.

We’d spend hours on end going there and back on flights that had no TVs, no entertainment, except maybe different music channels coming in through the armrest and small headphones. All you could do was chat, sleep, or read. It was boredom personified. Maybe as I got older I’d have my own Sony Walkman, a sleek Japanese-designed personal music cassette player. So cool, so “leave me alone I’m a teenager.”

Did we love it? Yes. Did we enjoy the long journeys? Probably less so. Did we understand the financial cost of what going back during winter holidays meant to our parents? Probably not but it was probably exorbitant. Was there something there that drew us back every year? The people. Magnets like Prem, Urmil, Dadi, Virinder. Love. Did we understand anything about how this home – Model Town Number Do – was found during the partition with a large Muslim family fleeing and large Hindu family taking its place? Not at all. Those stories would come much later. Would I have rather spent time with friends over the holidays? Probably. But that’s what we did as a family every year and I’m so glad for it. That’s the Luthra family in a nutshell – crazy, happy, chaos. Love.


Also didn’t mention the little boxed-in room where we would all bathe, with a balti, and turning on the geizer(?) in advance to make sure there was enough hot water for everyone. How did we have enough hot water for everyone?

And I don’t remember the men-folk ever doing anything of help, but there must have been all sorts of behind-the-scenes support for Dadi that was happening by all the chachas. I do remember card playing by the adults, beer or whiskey drinking. 

GAURI

Model town, Panipat, a small town in India, was where my grand uncle and all of his family lived in his childhood.

Their house was a beautiful white mansion, surrounded by red brick wall with vines. In front of the house with the road that leads up to a black shiny gate. Beyond it was the back of their house. The main entrance was in the back where there were mango tree and six guava trees.

The right side of the house belongs to my grand uncle and his brothers and sisters. Kanchan and Kanta were his sisters: Suraj, Prem, Ashok, Krishan, Virinder were his brothers. They shared one room and his sisters had one. My grand parents kept the left side of house. Great grandma would love to cook and garden, so they placed the kitchen next to the backyard. And to this day, those trees are still alive with juicy fruit and all the warm memories.

One memory that stands out is the wooden swing that hung from the ceiling in the veranda or patio.

Another memory that he still recalls is my great grandmother’s garden. There were grapevines shrubs, and in order to make the grapes ripe, my grand uncle, and all of his brothers and sisters would cover the grapes with white cloth pouches. This would help the black grapes ripen quickly, and it would keep the colorful, noisy parrots from snatching and eating the juicy fruit. Somehow they would still end up stealing the grapes, so everyone would take turns guarding and scaring off these hungry birds.

My grand uncle, and all of his siblings would also bring blood from the butcher and feed it to the vine’s roots. Before he used to think it made the grapes red but now he realizes that it was giving nitrogen to the roots.

My grand uncle’s memory involved a jamun tree. It is a tropical fruit only found in India. It grew on its own. My great grandma had always disliked uninvited guests but she really hated this particular plant because it would stain her kid’s clothes and they were all very spoiled to wash their own clothes, causing my great grandmother to wash them for her children.

And to this day, my grand uncle has that bad habit, now his wife washes his clothes for him.

JUGINDER
Participants from different generations have pitched in to add colors to paining, thus making a lifeless, now demolished house, come alive with so many fond memories. Almost all of them are glorious. There must have been lot of stress for Mataji. In the late 1950s I remember her locking herself in the store room and hearing sounds like Mataji was slapping herself on the thighs. I never asked anyone about it, but did wonder. We never heard her complaining to us or in public to anyone about what the fate had served her and her family. May be this was a way to get her frustrations out. Another very vivid nightmare I used to have when I was a kid – I was a very small ant going about in my merry ways and a huge wheat filled sac was tumbling down and ready to crush me and then it would stop. It occurred frequently enough to leave such unforgettable memory. 
