\chapter{Survival}
Our body is a miracle of non-stop self governing, defending and repairing
machine. It serves us for all our needs and demands. It has amazing
mechanisms to protect itself and survive despite getting attacked by
numerous enemies. Most of the enemies enter the body through the mandatory
breath and food. Some enter through breaks in the otherwise impenetrable
skin. Some enemies attack us from within while their origin may be due to
external irritants such as smoking, drugs, alcohol among others or
governed by genes while the cause of some remains illusory so far. Even
the strongest tend to buckle under giants. 

I feel fortunate to have survived, rather flourished, when I think about
numerous enemies I had to surmount living in Panipat, India in 1950s.
Obstacles can bring us down or make us stronger. Our immune system learns
to recognize the enemy and its soldiers stay alert and are ready for the
next invasion. 

Growing up in that era exposed our bodies to innumerable invaders. I just
have to close my mind and the vibrant images emerge from deep subconscious
mind. 

There is the unavoidable dust in the air due to vast dry land, constant
construction as the newly developing Model Town keeps constructing
thousands of homes and shops to accommodate the influx of hundreds of
thousands refugees uprooted from newly created Pakistan. The cilia in the
trachea and bronchi are working 24/7 to eliminate the dust through cough
and sputum that is considered a norm. Noises over the sink outside the
front door or spitting anywhere is proof of machinery clearing itself. It
is most noticeable during quiet gatherings. . 

At times the minor cough turns into painful sore throat. Gurgling sound of
salt water gargles is commonly heard over the sink outside the front door.
Our mother, Mata Ji, mixes salt, charcoal ashes with mustard oil and
applies the concoction, called ghundi, to the throat with her fingers. She
does it ever so gently, without inducing a gag reflex. She has to be an
expert. After all, she raised 10 children, even though two did not make it
beyond the age of six years due to small pox in what had become Pakistan.
These home remedies are effective almost all the time. 

Summer heat takes its toll by making life uncomfortable for us. Tiny
reddish dots, called pith, spreads all over the body. We are instructed
not to scratch them lest they get infected. These spots resolve
spontaneously when the temperature cools down. 

Those people working outdoors without access to shade and drinking water
are prone to get heat strokes. Pyaaus (drinking water stalls), set up as
fixed structures or portable ones on wheeled carts, are a common scene.
Many of them run as businesses while some are set up as means of charity
by the well-to-do, good hearted people. 

Everyone eagerly looks forward to the cooling effects of monsoon rains.
Nothing comes free. The rains provide the much needed dip in the
temperature and supply water for everything thirsty to its core. As a side
effect they also create ponds and puddles which became breeding grounds
for mosquitos. They love to bite and suck blood from us. The female
mosquitoes are the carrier for malarial parasite. They buzz around and
bite the sleeping folks in the evening or at night. To save our skin and
sometimes life we set up mosquito nets over our beds.  Swarms of them
follow above our heads as we walk around in the evenings. No matter how
hard we try to break up the assembly by swaying our hands, they regroup
immediately. Even running cannot not match their flying speed. 

Mosquitoes rush to get indoors in the evening looking for victims. We
spray Flit with a hand held pump to kill any invaders and quickly close
the doors. Lizards on the ceiling and walls are periodically darting their
tongues out. Insects gather around the light bulbs and that spot becomes
the dining area for the lizards. 

I still remember going through the classic cluster of symptoms and effects
of malaria. It starts with fever and intense shivering for a day. Thinking
that body is cold, I get tucked in the bed and covered with cotton filled
covers, razaai. Cloth soaked in ice cold water is applied to the feet and
forehead. Profuse sweating is seen as a sign of relief. The fever and
chills subside for several hours only to return in a day or two. The
diagnosis of malaria is made without consultation with a doctor.  Quinine
tablets are started on our own. I remember getting my tilli enlarged on
the left side of upper abdomen. That area, normally soft, has become very
hard upto mid abdomen.  Later I learnt that we were feeling the enlarged
spleen. 

When the fever does not subside, Mata Ji walks me over to Dr. Parma Nand,
pronounced by us as one name--Parmanand, who lives three houses from us in
5 Model Town. As children we love going to him. All his medicines are
sweet powder or tablets. 

After several days of taking his medicine and bitter quinine, hydration,
activated immune system, prayers, my symptoms subside, hard tilli shrinks,
a common fatal illness is cured. Malaria is a dreaded disease which kills
hundreds of thousands people. We are lucky that it did not kill any member
of our family close relative. 

Not only for my episode with malaria, Dr. Parma Nand is our first line of
medical intervention. He is a self taught homeopathic doctor. All his
medicines look same to us. They are small sweet white tablets sitting in
clear glass bottles on a few shelves. 

After listening carefully to our complaints he holds our wrist feeling the
pulse, have us stick our tongue out, "Say aah", pulls down the lower lid
and looks at the eyes. With this much information he picks up two or three
labeled bottles, shake out a few white tablets on a white piece of paper.
Sometimes he crushes the tablets into powder form. Carefully he  folds the
paper and asks us to take one pudhiya (folded white paper), or tablet two
or three times a day. He always ends up by saying encouraging words "Take
this and you will be fine in no time."  Children think he is a quack
giving same sugar pill to everybody from different bottles to simply
impress the patients. 

Most illnesses get cured by Dr. Parmanand's medicine, his encouragement,
placebo effect, simply time for the body to heal itself or a combination
of all. 

If there is no relief, the next line of defense is Dr. Lal Chand,
a partially balding man wearing a hard khaki hat, who makes house calls on
his bicycle. He lives in 56 Model Town, not far from on Bosa Ram chowk.
He does not have M.B.B.S. but has some diploma attached to his name,
enough to get the title of a doctor. 

A message is sent to Dr. Lal Chand. Hearing the bell from his bicycle
outside our door, one of us runs out to carry his brown leather bag as he
is greeted into the house "Come in Doctor sahib. Will you like to have
some water?"  No one refuses water in the summer days. 

His examination, in addition to the tongue and eyes, extends to check the
pulse, palpate the ticklish abdomen and placement of stethoscope on the
chest. His medicines, dispensed from his brown bag or later picked up from
the shelves in his clinic at home are much hated bitter tablets or
partially sweetened liquid dispensed in light brown glass bottles with
raised markings as a measure of the dose, called khuraak. He gives
instructions, never failing to give advice about what to eat or avoid
during the period of sickness. Reluctantly we gulp down the bitter pills
or liquid medicine followed by a spoon of sugar to wipe away the bitter
taste from the tongue. We are consoled "If you hate the taste so much,
imagine how it would taste to the little malaria or other bugs.”

Next line of medical defense was Dr. Nehchal Das, 63 Model Town. He is
also not an M.B.B.S. but has better diploma or the one with more letters
in it. He has a reputation better than that of Dr. Lal Chand. Occasionally
we end up with him and are treated the same way as by Dr. Lal Chand. He
just charges more and does not make house calls; may be that is why he is
considered a better doctor. No lab studies are done. I did not see an
X Ray machine till I joined medical college, Amritsar. There is not even
one pharmacy store in Panipat. Doctors or their compounders dispense the
required medicines. 

Last resort is with the only doctor in Panipat with M.B.B.S. written next
to his name. He is Dr. Dhamija who lives in 264 Model Town, which to our
little legs, seems miles away from home. After similar evaluation he
writes a prescription that we take to his compounder sitting in the next
room. The compounder takes out pills from different bottles lined up in
the shelves on the wall, puts them in a pastel and crushes them with
a mortar. Folded recycled news paper pudhiyas are given with instructions.
Liquid medicines from large bottles also dispensed in small brown bottles.
The compounder acts as a doctor when Dr. Dhamija is out of town. 

Once someone was really sick at our home. We bypass the first three tier
of doctors. Two of the boys run to Dr. Dhamija's house to request him to
come to our house in his car, one of the few people in town with that
luxury. 

“He pulled his back muscle this morning. He is resting but will surely
drive up to your house in the afternoon to check the patient", the
compounder told us. The doctor had misdiagnosed the symptoms of his own
heart attack and died that afternoon at home. 

Food poisoning is prevalent due to improper hygiene as well as absence of
refrigerators. Almost always it is a transient issue resolved by the
body's innate mechanism of vomiting and diarrhea to eliminate the
offending organisms or toxins before they have time and ability to get
absorbed into the body from the intestinal tract. Sometimes we are not so
lucky and sickness can last days and requires pills. 

Cholera (Haiza) is a highly contagious and serious disease. Hand washing
is a rare phenomenon in India. But in our home we are instructed to wash
hands before meals but otherwise nobody cares. 

Summer is water melon season. It is considered a cold fruit which also
provides much needed hydration. When eaten at home, the family gathers
around a freshly washed big tarbooze, cut slices and finish the whole
thing in one go, biting the slice down to the white of inner wall. 

Vendors set up stalls displaying cut red, juicy slices. Flies love the
sugar and sometimes hard to distinguish from the black seeds. They swarm
all over but our mind does not register their presence. It is simply
a normal thing to see. With open out houses and uncovered sewers the flies
become an efficient and fast carriers of the cholera causing bacteria.
Intense nausea, vomiting, and diarrhea ensues. Fortunately lot of fluids,
pills and syrups from the step ladder approach of doctors cure us. 

The episode curtails our eating watermelon from the street but we cannot
resist sweets from Bosa Ram? Flies like the same exposed sweets but we do
not pay attention to them. Consequently we pay the price in the form of
Typhoid. It is almost mandatory for all of us to get at least one episode
of this malady. High fever, vomiting, diarrhea, abdominal pains are the
punishment for our weakness for eating openly displayed sweets and other
outside food. 

This does require visit by or to the doctors. Antibiotics and hydration
cures it. At least for the occupants of 2 Model Town. 

Outdoor summer activities, luckily for us, cause only minor falls from the
trees and bicycles. Bruises and small cuts heal very fast, thanks to our
youth and strong immune system. 

Even though scorpions and snakes are in abundance in vastly open fields,
we never had poisonous bites. We do see snakes but leave them alone, and
they do the same. 

Doctors are very busy in the summers. We do not always have money to pay
them. They happily accept home grown guavas, mangos, or vegetables. At
times, simply saying thank you is good enough except for Drs. Nehchal Das
and Dhamija. Times are hard for everyone but some doctors just don’t get
it. 

I see that even now, even here in the land of prosperity. They forget
a simple lesson later taught to me in 1973 by the most respected
professor, Dr. Prem Chandra "When you look at the eye of the patient, do
not look at their wallet.” This lesson should be universally applicable,
not only in Medicine but in any profession and all spheres of life. 

It seems strange now but in the 1950s it is considered normal to see tiny
wiggly white worms moving around in the potty. Sometimes we see worms that
are like small earth worms. It does not scare us because everyone else
reports having them. Surely we get treatment for them too and get cured.
Parents repeatedly tell our deaf ears to wear shoes when we play in the
dusty fields. Infections invade through minor cuts or bruises in the feet. 

While playing outdoor games of all types it is not uncommon to get bruises
and cuts. Aak, also known as milkweed, grows wild in the open fields that
are rapidly disappearing giving way to new homes. We break off a branch
with oozing thick milky juice and apply it to the cut. Apparently it has
antiseptic qualities. Very rarely the wounds get infected. If aak is not
available we use dirt to cover up the wound, oblivious to the danger of
tetanus. Luckily none of us got that dreaded disease. 

Miraculously only one major injury at 2 Model Town, beyond the cuts and
bruises, is a fracture sustained by Virinder while playing cricket. There
is not even one orthopedics doctor in town.  In the newly expanding
Panipat there is on formal hospital either. 

The fracture happened while playing cricket. When I asked Virinder how it
happened, he laughingly recollects: "It was the new pajama, too strong to
tear. When I tripped, my leg got caught in the bell bottom end, as was the
fashion those days, and the damn thing did not tear. All other pajamas
were old and tattered and on the brink of being torn at the slightest
force.  Our clothing inventory, as mentioned before, was basic minimum.
I had just got the new pajama stitched from our family tailor in house
number 40.  Any old pajama that day, you will not be writing about my
fracture.  Pajama would have torn and I wouldn't have fallen."

Pitaji happened to be home. Going to Karnal, which had a hospital, 25
miles away by bus, in so much pain, was not possible. At someone's
recommendation, Virinder and Pitaji hail a rickshaw, a rare luxury, and
head for a man known for his expertise in fixing bones. He lives in old
Panipat. They are asked to wait while the 'bone doctor' is called.
Virinder recalls this episode years later. "I saw an old man, everyone ten
years older than us looked old to us those days, walk in with assistance.
He was about 80 and blind!" The blind old, so called unqualified, bone
doctor held and felt the left forearm where both the bones had broken,
displace about an inch but skin was still intact. His one hand held the
proximal and the other held distal end of the forearm, pulled the farther
end out with a jerk and then snapped it back. Ouch, I screamed. But in
that one movement the blind man had put the bone ends together. No
anesthesia and no pain medicine. Having done the miracle, without X-Ray or
scans, he put wooden splints held together by a bandage. A sling around my
neck and the arm has been like new ever since."

The malshi, (masseuse) did the follow up care. "After a couple of weeks,
he periodically removed the splints, gave gentle massage to the area and
put the splint back. After about 4 weeks the fracture had healed, splints
were removed and I went back to my college in Ludhiana. Dr. Ram Lal
Narang, our Bhaaiya Ji, (Pita Ji’s sister's husband) arranged follow up
care with one of his doctor friends."

That is how our brother Virinder spent his summer vacation in Panipat at
age seventeen in 1954. And spoiled our clean, serious injury free record! 

I personally do not have any memory of our first house in Panipat where we
lived from 1947 to 1950. 

Virinder still gets nervous talking of the day when our Mataji almost
died. 

It happened in our original occupied house in the 'Shehar' (City). It was
a two storey house previously occupied by a Muslim family which fled for
their safety. Following their departure we occupied it for our family. It
had open verandas but no windows. Light came into the rooms through the
open doors. There was no electricity. Store rooms had no ventilation and
were pitch dark. One had to carry an oil burning lamp to go there. 

On one fateful day, in 1948, at the young age of 36, having survived ten
home deliveries and all the prevalent diseases, she developed an attack of
asthma. Pitaji was away to Palval. Children, their Nani and Shiela bhua
were at home. Virinder can still hear the wheezing sounds as Vidya, our
Mataji, tries to gasp for the precious oxygen. She is struggling for
breath and getting paler by the seconds. Nani frantically rubs her
daughter's back but it is not helping. 

Suddenly the wheezing stops. There is no breath. Nani starts wailing,
beating her chest as ladies in India do when someone dies. She starts
crying loudly and saying "Vidya, thu aidee jaldi chodh gayee " (Vidya, you
left us so soon.) Virinder bolts out running down the streets of the
strange new town frantically asking everyone if they know of a doctor.
Fortunately he finds one within few seconds-Dr. Hans Raj. The doctor
rushes wasting no precious moments. There is no breath but heart is still
beating and pulse feeble but present. He quickly gives some injection and
our Mataji starts breathing again. There were no after effects of this
episode. It had never happened before and fortunately never recurred. Most
likely it must have been the mold in the store room which initiated the
almost fatal asthma attack. 

Soon we move to the spacious, open house with many doors and windows. We
have electricity. And we have our mother, hale and hearty. She never told
us this story. May be she was busy living her present. May be we did not
make time to listen to her. 

Considering all the outdoor sports, flying kites on rooftops, running,
bicycling without any guards, climbing trees, jumping in water without
knowing how to swim, playing day and night on the grounds infested with
snakes and scorpions and fights with axes and bamboo sticks, it was
a a miracle that we survived and that also with only one major incident. 

Even though we do not have any spare money, there is just enough to buy
food. This is amply supplemented by home grown vegetables and fruits.
Fortunately we do not suffer the effects of malnutrition, a problem faced
by millions of Indians. Similarly due to reasonably good hygiene and open
spaces we do not suffer the effects of trachoma, an epidemic eye disease
and common cause of blindness prevalent all around us. 

Surma, also known as Kajal, a black powder made of lead, carbon, antimony
and zinc is used by girls as eye liner for cosmetic reason. It is presumed
to have antiseptic properties. It is also applied to the eyes of newborn
children for presumed antisepsis as well as to protect the baby by warding
off any evil eye. 

Whether it is brushing with the neem tree branches, rinsing of mouth after
meals or lack of chewy candies, we never visit the only self taught
dentist located in the old part of Panipat. There is no qualified dentist
in Panipat. No one knew the word Orthodontist and nobody had wires on the
teeth. We just leave them alone and teeth get aligned for the job nature
prepared them to do. And if the teeth are crooked, no one makes a big deal
about them. 

Small pox and polio are prevalent all over India.  Our two older siblings
had succumbed to smallpox around the age of six years. From that tragedy
onwards our parents learnt about these diseases and did get us vaccinated.
Two large marks on the left arm from smallpox vaccination are evidence of
their wisdom.  These are much better than the pock marked faces, blind
eyes, death and occasional limping child we see among our school mate.  We
are the lucky one who have lived through the onslaught of these virulent
viruses. 

Everyone experiences attack of chicken pox. We get rash all over the body
and it is contagious. This is uncomfortable but not fatal or accompanied
by any long term complications. So we thought then, till I learnt later
that the Herpes Zoster shows up in later life from the sneaky chicken pox
virus which hides in a cocoon and comes out swinging when our immune
system weakens in old age. 

Bed bugs and lice are mere nuisances.  Itchy red spots on waking up in the
morning prompt us to wash the bedding in hot water, shake the beds, hit
the jute or cotton strips of the charpai. Then we look for the bugs on the
floor and also search for the hidden ones in the crevices of the beds.
Surely they are on the floor and crevices. They are promptly squashed with
shoes. We check and recheck the beds till no more are seen. 

In the neighborhood we see mothers or elders have youngsters sit in front
of them as they merrily chat away and search for the lice at the roots of
hair on the heads. Special combs are there to trap the tiny bugs. Picking
them out of the hair and then squashing them between the two thumb nails
is considered simply one of the routine chores. 

Thanks to no over crowding, daily bath, sometimes twice a day, very
infrequently anyone in our home has issues with the lice. When detected it
gets dealt with right away. 

Fortunately no one smokes or drinks in our home. We are spared of the
second hand smoke and the subliminal signal that such activities are not
good. We never hear of anyone at our home or in the entire neighborhood
using recreational drugs. In fact we learnt much later that there was
a difference between drugs and medications. 

Winter months send flies and mosquitoes into hibernation thus reducing
many of the summer diseases. Multiple layerings protect us from extreme
cold weather in January and February. Woollen pants, jackets and long
winter coats, donated by affluent Americans, are sold by vendors on
wheeled carts.  We eagerly sift through the big piles, find one that fits
reasonably well and buy these at very reasonable rates. The purpose is
protection, not fashion. In the absence of any source of external heating
we have to find ways to conserve our body heat. Thick cotton filled quilts
(razaais) and layering with sheets protect us during the freezing nights. 

Some less fortunate people without these basic needs sometimes freeze to
death. It is not uncommon to see rickshaw pullers curled up at night on
the seat barely covered with a sheet or a thin blanket. After finishing
undergraduate college when I returned home by train and then rickshaw,
I gave my razaai to the emaciated looking rickshaw puller "I have one at
home, you need it more.”

Being mostly indoor and many occupants, only one has to get flu or other
respiratory bug. And we all share it. Luckily these are self limiting and
clear up in a week. Due to salt gargles or due to our strong immune
system. 

Before we know, the winters are over and spring brings wonderful weather.
New leaves sprout and unfurl on the trees, flowers and fruits follow which
instill signs of vibrant life all around. 

With each passing year we get exposed to almost all bugs, thankfully in
micro doses which act like vaccinations, thereby making our immune system
stronger. 

We are ready to jump outdoors, eat gol gappe (Panipuri) and chaat from the
street vendors who rarely wash their hands. Bosa Ram again makes
irresistible sweets. In good health and spirits the family is ready again
to face the challenges of survival in Panipat. 
