\chapter{Forty Kilograms}

The receptionist at the front desk was rolling her eyes and throwing up
her free hand in the air, seemed frustrated, apparently from being
screamed at by the person on the other end of the telephone line. She
would have loved to hang up and before doing so would have surely told
that fireball where to go and burn, in no uncertain terms, but could not
for fear of losing her job, especially in these difficult times when
many of her friends and family members were getting laid off with no
prospect of getting another job. A small droplet was trying to make an
appearance from the ever so slowly rising lake of tears in her left eye
as she had hooked the telephone between her left ear and shoulder so
that she could keep working on the computer partly to tune out the
greyhound, who would become a poodle in the examination room in front of
the doctor, and also to finish her day's work lest her off site manager
presumed tardiness.

The line at the check in counter had gradually become five deep. The
lean, partly balding man in well-ironed light blue shirt, grey pants and
hurriedly polished shoes was at the head of the line. The shirt matched
his unusually blue eyes which were highlighted by his white hair's
backdrop. He had Indian features but with the fair skin and blue eyes he
could easily pass for a Caucasian. His wrinkled forehead betrayed the
worries he was trying to hide with his twinkling eyes while making jokes
with people behind him.

His keen sense of observing others in distress picked up the anguish the
receptionist was experiencing. He pulled out a tissue from the Kleenex
box for the receptionist to wipe away that droplet before it embarrassed
her by spilling over her mascara in front of now 7 persons fidgeting to
sign up and start the long wait before seeing the doctor for maximum
five minutes. This was the first time he had come to the office and had
never met the receptionist before. As much as he wanted to help her,
just as he always would go out of his way to lend a helping hand for the
needy, he folded the tissue into his palm as he was well aware of the
strange laws of presumed sexual harassments in this new country. He used
the outstretched hand to pick up the pen tied to a string and wrote his
name in the sign-in sheet in a neat, straight handwriting---Prem Luthra.

He was quite content with the diagnosis from his family doctor in India
but his son, Ashit, who is settled in New York, insisted ``Papa, you
must come and get a second opinion from Joan Harding Cancer Center, they
are the best. Unexplained weight loss can sometimes be the first sign of
a hidden cancer.''

Prem had come half an hour early, a childhood habit. He was never able
to figure out the origin of this reflex. No matter where he had to go,
the alarm clock was set hours before the departure, check lists were
made, a series of questions had been prepared and revised till he was
satisfied that every possible situation had been considered, mentally
solved and put on legal sized paper in neat double spaced lines. Before
taking the final step out of the house every item had to be checked and
marked off.

The fancy waiting room filled with leather chairs, high definition
television, coffee with assortment of cookies on one hand made him
relaxed and confident that the doctor must be considerate and
compassionate while on the other hand might be very expensive. After
all, the customer pays all the bills. Having no insurance was also
playing on his mind and was one of the reasons he would rather let the
unexplained weight loss play out its course than bankrupt his hard
working son who had a wife and two children to support. Ashit had again
refuted the argument by saying ``Papa, remember, you used to forego all
the pleasures of your life so I could have everything I wanted, needed
or not. You never said no, not once. It is my turn to pay back,
especially when we feel guilty leaving you and Mummy behind in India,
alone.''

Over the last year or so he had noticed a gradual decline in his weight.
It used to be steady 40 Kilograms in his college days and for a few
years after. And then he got married. The excellent cooking and company
of his wife made him eat more in quantity and regularly. The cheeks as
well as the abdomen puffed up reflecting prosperity in happiness and
wealth.

Unlike his other siblings, who all went on to professional colleges due
to the determination of their father, he had chosen the path of getting
a job selling radios and parts. He had no interest in pursuing further
studies in the two professions commonly talked about\ldots{}medicine or
engineering. When someone would ask him ``Beta, what are going to do
when you grow up?'' his standard answer, partly true and partly to get
them off his back, he would say ``I want to do business.''

At that time he was old enough to see and feel the stress his father,
addressed as Pita Ji, was going through from raising and educating eight
children single handedly. That number is daunting in itself and to top
that the hardships had multiplied due to the unforeseen partition of the
country resulting in loss of the home and steady income from farming in
Khanewal, India which all of a sudden was now named Pakistan.

In May, 1947 the family had come to a hill station called Sabathu, 20
miles south of Shimla, where Pitaji had purchased a summer vacation home
in early part of 1947. In August,1947 the country got divided based on
the religion with Muslims claiming Pakistan whose boundary was decided
by a line arbitrarily drawn by an English bureaucrat. Unfortunately
their ancestral home and part of the country where several generations
had lived was suddenly not their home any more. Sad as it was but they
were blessed that unlike the 12-18 million people who migrated on both
sides and about 1.5 millions were massacred trying to do so, they
already had at least a roof on their heads and they were safe and all
together.

Prem had frank discussions with his father resulting in the decision to
supplement Pitaji's meager income by getting the job in Madras. The
positive effect of such a huge physical distance, in the absence of
telephones and air planes and short train journey, instilled a fantastic
lifelong habit of writing letters. Letters became longer and more
frequent to make up for the vast distance created by the move to Madras.
Prem had a unique quality of visualizing the intended recipient on the
pages of his long letters, vanishing the physical gap, which would make
him less homesick. He would then simply talk to the intended receiver of
his letter while the pen automatically put the words and feelings on the
paper. Being the only one flung thousands of miles away, he wanted to
bottle up and store every possible semblance of an object that would
connect him with the separated family. The other members were not
deprived of the family structure and therefore never realized or
comprehended the pangs of separation that Prem was going through. They
did not write back as frequently as they received his frequent and
voluminous letters. The letters were his life-line for survival and he
hung on to all the letters he wrote and received. He started a unique
habit of making a three ring binder for each member of the family,
friends and even strangers who he wrote to or received a letter from.
Meticulously, he made copies of every letter that he wrote and filed
carefully into the binder. He went even farther by making an index
showing the date and summary of the main subject of the letter. Then he
would patiently wait, check the mailbox at least once and sometimes
twice a day just in case he missed a post card sticking to the bottom of
the familiar red box. Sporadically as they trickled in, the original
letters from the family members, which he had so longingly awaited, were
read and re-read and then lovingly filed them in the appropriate folder.
At the first possible opportunity he would sit down on his usual chair
and table, with his ink filled pen and legal size papers and shoot off
another monologue to the face he clearly saw in the paper. Before
mailing, he fondly made copies and carefully filed them in the binder.
The collection of his cherished memories traveled to Calcutta when the
job brought him to the new and life long residence. One person who added
the most to his collection of letters was his soon to be wife, Shashi.
They promised to write a letter to each other everyday from the time
they got engaged in December 1959 till they would get married which they
finally did on October,8,1960. Marriage not only brought prosperity to
Prem's cheeks and abdomen, now a healthy 70 Kilograms, but also added
weight to the ever increasing collection of letters written and received
by him to another group of relatives who came with his marriage.

As lives became busier, postage more expensive, incomes shrunk, lack of
expected replies, the rate of weight gain of his collection of letters
slowed down. The final nail in this decline was the invention and rapid
explosion of use of internet.

He and his grandchildren, Disha and Tanuj, one day marveled at the life
long collection. Prem, in his usual manner, wanted to place a bet with
Disha and Tanuj about the weight of the mail. The one closest to the
actual weight would do 21 salutes in front of the friends at Saturday
Club. Excitedly the grand children took on the grand father, fondly
called Babu Ji. To the children the huge pile looked massive and they
bet it was 50 Kg while Prem under estimated it and placed his bet at
35Kg. Actual weight was an astonishing 40 Kilograms, the same as Prem
used to weigh in his college days.

Grandchildren were surprised by the coincidence and Prem in his mind
quietly wondered about its significance. His inquisitive mind always
wonders about each and everything and this figure of 40 Kgs of his
college time weight and the current, ever slowing rate of growth of the
mail made him wonder if there was a hidden meaning of nature. To confuse
the matter even more was his own decline of weight over last two years
to 49 Kilograms. He wondered if the convergence of the almost stagnant
weight of the letters and his own declining weight had a mortal
significance. The question still bothered him and he put it aside to be
solved at a later date. He would relish the salaams of his grand
children at the Club and with that thought causing a smile on his face
he sipped a perfectly chilled beer, resolved to solve the mystery later.

``The doctor is running late due to an emergency at the hospital'', Prem
heard while half asleep in the comfortable chair as the nurse made an
announcement in the waiting room. Having read every conceivable book,
medical and otherwise, he knew that the doctor was held back in the golf
course and the announcement of an `emergency' was just a cover up. He
was willing to wait, only hoping that the doctor had not celebrated his
victory with a beer at the club's bar. The waiting room was full by now.
Some of them looked fairly healthy and out of place in the cancer
clinic. A couple of them were emaciated to the bones, again raising the
questions in Prem's mind about the possibility of cancer silently
growing, evading every conceivable test that his family doctor along
with the two greedy specialists had done in India. This had resulted in
similar loss of the weight of his wallet and he would jokingly say ``I
will die when these sharks have stolen every paisa and my wallet is
squeezed out of money and life is squeezed out of my body''

After another 45 minutes wait, the nurse shouted out over the sound of
television and a constant chatter in the waiting room ``Prem Luthra!''

Prem had made sure that morning to put a new battery in his hearing aid.
Despite the noise, he heard it clearly and nervously made his way to the
examination room pointed by the nurse. The folder full of copies of the
reports and a long list of questions was carefully held in his right
hand. She is really pretty, he thought as he followed her, being careful
that she did not notice the direction of his sight which admiringly
followed the curves of her body. The age and perceived sickness did not
bar him from appreciating the beauty when he saw it and he saw beauty
more often than not.

As ordered, he took off his clothes and donned a thin, worn out gown and
felt conscious about his thin body with wrinkled skin, even though he
felt like a sixteen year boy in his mind. The nurse did the preliminary
tests of weight, which was 47Kg, pulse and temperature, both of which
had gone up a notch at the touch of the beauty that was so close to him.
She noted down the chief complaint of unexplained weight loss over the
last two years. She also made a note about Prem's question of possible
link between the convergence of weight of the letters and his own weight
and whether a life ends when the purpose to live is no longer there.
Prem knew that this was not really a medical question but rather a
philosophical debate but he still respected medical professionals for
their knowledge and insight of the unknown.

Having done the preliminary work up, the nurse assured that the doctor
will be in shortly and she was sorry for such a long wait. He pulled out
his thick folder along with the list of concerns and questions and
nervously started the waiting game again. This time it was only 12
minutes, which was still long enough to start a shiver from the nervous
tension, loss of muscle mass and cool temperature which the doctor liked
in his clinic, forgetting that the patients were sitting almost naked,
barely covered by the old, flimsy gown.

Doctor Huffman appeared to be in his early fifties. He was well dressed
in a tailored blue striped suit and a cheerful red necktie. He did not
wear the dreaded white coat to prevent the additional anxiety for his
already scared patients, most of who were stricken with some type of
cancer. He had an easy reassuring smile, an unhurried manner and a warm
hand shake.

`` Good morning, sorry I made you wait so long. ``

`` Good morning, Doctor. Wait never bothers me. After all, where are we
going to go anyway, rushing around raising our own blood pressure and
also those of others around us? My wife, Shashi, and I use such time to
make friends with strangers and have some fun along the way for
ourselves and who will soon become strangers again. But memories linger
and that is what life is all about anyway.''

Doctor Huffman by now had pulled a leather chair next to the examination
table and was scanning through the chart. As always, even having read
the notes written by his nurse, he still wanted to hear from the
patients their complaints and history of illness. Patients, especially
males, had a way of giving only patchy history and needed prodding to
come out with the details. They tried to trivialize some of the
complaints, whether out of denial or pride that `nothing can go wrong
with me.' That is why he moved slowly and tried to get to know the
patients more intimately and also make them comfortable with sharing
their deep fears and concerns about disease, and mortality. `` So, what
nationality is that, Prem? Did I pronounce it right?''

`` Oh yes, you did fine . You can't mispronounce that name. It is not
like some other Indian names like Juginder, poor fellow, my brother in
West Virginia who gets mercilessly hacked by getting called Huginder,
Jugainder and many times simply---what the hell is that?. My name is
Prem Luthra, call me Prem just as James as in James Bond. Delete the S
and that is how Prem is pronounced and you did a fine job with that. I
am from India and visiting our son and daughter-in-law who now live in
New York.''

`` What brings you to see me today?''

`` I told my son that there was no need to make this appointment because
I know my diagnosis but they insist that I be examined by the best in
the best cancer institute in the world''

`` Oh, you already know that you have cancer?''

`` No, I know that I don't have cancer but they feel that with my
progressive weight loss and in the absence of any other detectable cause
the bugger must be hiding somewhere. I have been put through the hide
and seek game by some necessary and many unnecessary tests in India. My
son feels that you guys are the best detectives to seek out the
sneakiest hiders. That is why I am here, bankrupting my children''

`` Hum, If you know you don't have cancer, then you must know why you
are losing weight and why you are worried about impending death?''

`` You see doctor, everyone needs a purpose in life to live. Nature puts
us on this earth to become a piece of the giant jigsaw puzzle. Each one
of us figures out, makes up or simply believes in divine destiny and
finds a purpose which makes him/her feel useful and the life becomes
purposeful. Studies at Rush University Medical Center in Chicago in 2009
along with some other studies are now scientifically showing that if you
have a purpose in life and lead a meaningful life, you live longer. You
take away the purpose and the man withers away. As if nature is saying
``I don't need to waste any more resources on this useless piece of
flesh. Let it perish and convert the atoms into something that will add
value to my purpose.''

After a short pause Prem slowly, almost in a whispering tone said ``And
I have lost my purpose in life. As if directed by nature my appetite is
diminishing and accordingly I am not eating enough, resulting in weight
loss, but no one believes me.''

Dr. Huffman listened to this interesting man and his self-made
diagnosis. His curiosity was getting the better of him. He knew that he
was already running 45 minutes behind schedule but he had never been
more curious and eager to learn more of this patient's unusual story. He
mentally made a plan to give Prem a complete physical examination now
and then, at no charge, have him come on another day when he could
easily spend time without worrying about the angry looks from 12 sets of
eyes in the waiting room. With help of the nurse, who had just walked in
indicating with her hands and eyes to hurry up, he removed the flimsy
gown from Prem's thinning body where the loose skin was folded in
places. The glowing skin at one time used to be smooth and taut to cover
up the abundant muscular mass.

After all, Prem at one time was the best badminton player in his college
and for many years after that. Girls would hawk at his swift movements
and muscular contractions easily seen through the sweaty white cotton
tee shirt. The looks and the victory ribbons would make him practice
even more and he stayed number one badminton player through out his
college. The muscles were now just a faded memory, though the silver
victory cups still reminded him, his son and grand children the golden
days of his body.

Dr. Huffman methodically, like he had done thousands of times before,
examined Prem from top of the head to the small toes. His eyes were
keenly looking for any hidden enemy, fingers were palpating all the
tissues they could reach. Percussion, even though being displaced by
fancy scans, was still a part of his armamentarium. Stethoscope again
was on the verge of becoming extinct, being replaced by well paying and
presumably more accurate echo-cardiograms and C T scans. Dr. Huffman
still believed in these gadgets because they added to his diagnostic
skills and also gave him a chance to spend time with the patient. He
listened with his instruments but more importantly he listened to the
patient very carefully with his ears. He knew that almost all the time
the patient in some verbal or non-verbal language was going to tell him
the diagnosis. The reason we are becoming more dependent on the tests is
that a time consuming examination does not pay the bills and also does
not leave behind a trail for the insurance companies and lawyers that
all that could be done `was' done. Dr. Huffman had decided to cover all
the angles by an exhaustive history taking and a comprehensive physical
examination. Even though this was not the protocol of his system, he
decided to just go through the examination at this time and planned to
do a detailed, non-rushed history taking at a later date.

The physical examination failed to reveal evidence of any obvious
disease that would explain the gradual, progressive weight loss. Orem
had made sure that copies of the tests and reports of examinations done
in India had been carefully cataloged, placed in a folder and arranged
date-wise. The front sheet enumerated the date, name of the test,
results and the name of the `Thief' who ordered and profited from the
test. This was Prem's way of getting even with the `thieves' as he would
describe the Indian doctors. Even the amount of money stolen from him
was entered with varying number of ? marks next to the entry. Huffman
carefully and with a smile reviewed the reports and copies of some of
the scans that Prem had coaxed out of the doctors who were reluctant to
do so for fear of getting caught having misread the test or someone
really finding out that the test was not indicated at all considering
the type of history and the findings of the examination.

The tests and the reports did not show any indication of the lurking
enemy. Huffman asked the nurse to arrange a couple of more modern cancer
detecting tests not available in India and a scan of the kidneys. Kidney
disease sometimes can cause no obvious symptom and yet cause unexplained
weight loss. Noting that Prem had no insurance and also the fact Dr.
Huffman had liked his friendly nature, he directed the nurse to have the
billing clerk give a 50\% discount on the visit and also on the tests he
had just ordered. Prem loved discounts and bargains. He promptly said,
``Since I am getting 50\% off, it means I can have my wife examined for
free. Like the signs we read in the shopping mall---pay for one , get
second free!'' Shashi had survived an Osteoclastoma at a very young age.
Decades had gone by without any recurrence. This was a miracle for which
Prem would thank God, although many times he would debate that there was
no such thing as God who would be busy managing our daily lives and
every move. In later years his opinion was changing toward accepting the
existence of a power beyond the physical body. He had started going to
the Ram Sharnam in Panipat at every chance he would get.

With a smile, Huffman said good bye for now and asked the nurse to make
the next appointment for Prem, making sure that he would be the last
patient of the day in case the story got more interesting.

The nurse left him alone to get dressed up and come out to the front
desk for billing purpose and next appointment. The dates for the tests
were arranged on a Wednesday afternoon when his daughter-in-law, Jyoti,
would be free from her job in the music recording studio.

Prem knew that the examination would be negative for any cancer but
still felt a sense of relief that the best detective of cancer agreed
with him. No matter how safe we feel, a visit to the doctor is always a
nerve wrecking experience. You never know, whether correctly or just to
make up a presumed diagnosis to order tests at the lab or facility where
they have financial interests, the doctors will add a dose of anxiety
and additional bills. Sometimes the doctors will add a pill here or an
injection there just to treat a symptom or finding that might even have
cured itself. Prem loved the cartoon where a doctor is leaning over a
patient in his bed and the doctor is saying ``Mr. Jones, we did the
operation just in the nick of time; two more hours and you would have
cured yourself!'' The cartoon can be funny as long as it is just that
but the fear of yourself being the hero of that cartoon is real when you
are in the examination room completely at the mercy of the trusted
doctor.

Prem loves to read at least two books a week. Any subject is welcome but
he truly loves to read legal and medically oriented books. Everyone
knows that there are good and bad lawyers and doctors but somehow they
all believe that their own is the best in the whole world. Most of the
times they are trusted as much or even more than God. Huffman fitted the
description of being the best in Prem's mind.

Having made the appointment two weeks later, settling the discounted
bill and wondering if he would use the savings for a nice chilled beer
before heading home, he said goodbye to the receptionist. With the
additional help having arrived and the office machinery moving smoothly,
she seemed to be calm and collected. In addition to the discounted
charges, she gave a generous smile to Prem. His gesture earlier to help
her tears was duly perceived and mentally appreciated by her.

Folder was carefully closed and tied. He wished that the doctor had more
time to discuss the history rather than spend time on the examination.
Doctor's office is one place where he, like most patients, are tongue
tied out of respect or fear of offending the life saver. Any other place
he would have told the guy to sit down and listen, especially when he
was going to pay a hefty amount of money just to listen but not to a
doctor. When paying the expensive dollars he said only in his mind
``They are all the same, here or in India. At least here I am getting a
50\% discount.'' But his mathematical mind quickly found a flaw in his
reasoning. Here he had to pay in dollars and even with the discount it
would translate into a big bundle of rupees. Slowly he made his way out
of the office and sat on a bench under a canopy to start the wait for
his son who was to pick him up during lunch break.

Lunch break reminded him of his friends in Calcutta, longish lunches,
chilled beers, making fun of strangers and friends with bets on every
conceivable subject. Malik, Rajpal, Singh\ldots{}all came alive in his
mind. His friends had called to wish him luck for the upcoming visit
with Dr. Huffman. They would have even placed a bet whether the doctor
would be late by more or less than one hour, the doctor would give Prem
more or less than 5 years to live etc. Malik would even jokingly tell
Prem that the doctor will never let him die before he paid the bill and
therefore the best way to live long will be to not pay the bill.
Laughter would be in the air for hours.

There was never a dull moment except one fateful day when someone in the
group jokingly told him `` Prem, don't you realize that no one is
sending you letters any more. Internet thaan ik bahana hai; people are
just not interested in receiving your same old letters and don't want to
waste their time writing back hoping that this will put a stop to your
long time-wasting letters.''This was said in the spirit of jest but
somehow Prem took it to heart. Even though he was told repeatedly ``Come
on Yaar, this was just a joke. You are taking it too seriously.''

No one should make fun of something so close to the heart, something
that has become a purpose of life, a reason to get up in the morning,
get dressed, go to the office, less for work and more for seeing the
faces of your loved ones in the white striped pages, something Prem had
done practically all his life. This one sentence pierced through his
heart. He wanted to believe that it was just a joke but then he would
wonder in the wee hours of wakeful nights that where there is smoke
there must be fire. ``Someone must have told my friend just what he
repeated to me'', he would mutter under his breath for fear of not
waking up Shashi and letting her know how one single sentence had
changed his life forever. That was the day when he started wondering if
there was a link between the stagnant weight of the mail and his own
declining weight. Behind closed doors he started weighing himself daily
and sometimes twice a day. His appetite became poor resulting in
declining weight. His analytic mind connected the dots between the mail
getting stalled at 40 Kg and his weight having come down to 49 and
falling. He got convinced that the mail was the purpose of his life and
it was nature's signal that when his weight would also reach 40 Kg, that
would be his last day on earth. He even envisioned a scene where there
are two pyres burning side by side. On one is lying a 40 Kg male and one
other there is a pile of 40Kg mail. Here he saw a merger between a
completed life and a completed purpose.

``First thing first, let us rule out any medical illness causing the
weight loss'', Shashi had emphatically said to him. To appease his wife
and just to be sure that there indeed was no cancer or such dreaded
disease causing the progressive weight loss, he had made appointment
with his friend, Dr. Minocha. Initial examination was negative but Dr.
Minocha did not want to take a chance of missing anything serious,
especially in a life long friend. Two specialists and many tests later
Prem was relieved that there was no cancer but was convinced that unless
either he put on weight or the mail started growing, his days would be
numbered.

His letters, now scanned through the internet, started reaching quarters
covering much wider audience and farther than the snail/expensive mail
ever did. The responses were generic and short in this fast paced life,
not counting as letters according to Prem's definition. The pace of
physical weight decline had continued even after he reached USA.

Honk, Honk!! Prem suddenly jolted from his mental journey back to India
with his friends, looked around and saw Ashit waving near the gate
leading to the parking lot, asking him to come over. Ashit was saving
the minimum \$5 parking charge, otherwise he would get an earful ``You
wasted Rs. 250 just for 5 minutes of parking. I am still strong enough
to run, let alone walk over across the road. When I was your age, the
whole family could buy food for one month with 250 rupees.'' As he stood
up he felt a little dizzy, held on to the end of the bench, rebalanced
himself and then started a slow, slightly wobbly walk towards the car. A
tear welled up in Ashit's eye as he watched the scene and also recalled
the badminton games the father and son used to play years ago.
Gradually, with will power, the pace picked up and the gait became
steady. He hopped into the front seat and methodically placed the seat
belt. In India, one could get away with a bribe of 10 rupees or it may
have gone up to 50 by now but in USA a ticket could be an expensive
lesson.

To the expected question from Ashit, Prem said `` See, there was no need
to feed more money to the already rich doctor. I am totally fine and the
big C has not made its home in my body.'' There was a distinct relief in
Ashit's demeanor, which could be seen on his face and now relaxed
shoulders. ``I am still glad that the best of the best has checked you.
A few dollars is worth the peace of mind'' said the loving son who
wanted to hang onto the umbrella between himself and his own death, a
thought that had lately started creeping in his mind.

After a few quiet moments, Prem said that the doctor wants to see me
again and with a smile said that he had charmed Huffman into not
charging even a penny for the next visit and some of the tests to be
done will also be discounted. There were still a lot of good people left
in this world, he thought. Reluctantly he even entertained the idea that
all doctors are not just after money.

After supper he pulled out the folder and carefully placed all the
documents in their proper places, wrote a summary of the day's events on
a single sheet of paper and stapled it with the medical reports. Shashi
got an extra tight hug of relief and dreams of long life together, only
if he could somehow control the progressive weight loss. With that
thought and a book by James Patterson in his hand he turned to one side
and before long the book fell down. Shashi carefully pulled away the
glasses, lovingly rolled his grey hair back, covered him with a soft
blanket and sat up for her regular meditation session in her bed. She
said an extra thank you to Ram and called it a night with a smile on her
lips. What a wonderful life, despite all the ups and downs of their
lives they were just as young today, together as when they were courting
many decades ago, which felt like it was only yesterday. Few blinks of
eyes and boom, the years flew by. She felt happy that they had not only
filled those years with open and private affection for each other but
also that they had enriched and filled many a heart with their genuine
care and love.

Filled with books, walks, visits to monuments and museums and
interacting with Ashit and Jyoti's friends and strangers, two weeks went
by fast. The weather in USA suited them well, not that they ever
complained of the oppressively hot weather followed by ocean pouring
down from the monsoon clouds over Calcutta choking the fragile veins of
the city's drainage system.

The flowers in the park were in full bloom, all the lawns in their
development were manicured better than most heads of many of the
youngsters they would see in the shopping malls. Trips to the malls were
purely recreational and for exercise. They had never been fond of
accumulating material stuff but lately even a mention of buying one more
thing could trigger a massive headache. Buying a gift for Disha and
Tanuj was a whole different story. Children need the excitement and use
of the gifts, he would often say. Some things never change and it was no
different when at the age of 25 he felt old enough not to need anything
for himself. Almost half of his pay would go towards a monthly check to
Pitaji for running expenses at 2 Model Town, Panipat, the newly adopted
home. About a quarter would be used for living expenses and the
remainder was used for gifts for the four younger brothers, age 7 to 16.
The youngsters growing up in poverty never were allowed to feel any less
than the most well -to-do folks around them. Prem's every visit to home
meant boxes upon boxes of gifts. Braino, Mechano, cricket set, Murphy
radio, transistors for constant cricket running commentary, Bianca feet
Malay, among numerous others kept children occupied and happy. When
their mother, Mataji, died in 1990, several of Prem's gifts were fondly
spread between his siblings to lovingly keep in their homes, reminding
them of the glorious childhood. The south Indian temple made from the
pulp of bamboo shoots and a large sea shell which had adorned the
cupboard along with the picture gallery of the family, are now prized
possessions of Juginder and Dolly. The brown electric clock on the wall
across from Mataji's bed was always in demand as no one else had a
watch. Keeping up with the age-old habit, Prem and Shashi would always
be on the look out for gifts for their grand children. One day before
his visit with Dr. Huffman they bought one thank you card for the doctor
and one for his staff along with seven picture post cards with a
resolution to write to Disha and Tanuj every day for a week.

Appointment with Dr. Huffman was at 3.45. Prem asked Jyoti to drop him
off at 3.15 so he had enough time to organize his thoughts and thick
folder for the doctor. He had not heard the results of the tests that
had been done a week ago. No news is good news, especially when it
relates to the report of the test results, he thought. Blissfully he was
oblivious of the fact that many doctor's offices do not call back with
the results because they don't have a tracking system to follow up on
all the tests that had been ordered. If the lab does not send the
results or the reports get filed away or lost, doctor would never
remember to call the patient. Some doctors wanted to discuss the report
in person especially if it was going to be a bad news but many times
just to collect fees for another visit.

At 4.00 PM he was escorted into the examination room by the pretty
nurse. His pulse rate went up some but he did not care as there was
going to be no examination and nobody will know or suspect an old man
having any sexy thought. He even escaped the embarrassing gown. He used
to say that such gowns are like insurance companies; they guarantee
coverage but leave many cracks open in critical areas.

Dr. Huffman soon followed and pulled a chair close to the one occupied
by Prem. `` I am so happy that you came back at the end of the day'' he
said as he shook Prem's hands without first washing them. Prem made a
mental note to not touch any part of his body till he had thoroughly
washed his own hands at the conclusion of the visit. He had read that
about 98,000 American die annually just because they were in the
hospitals. Number one reason was cross infections spread by providers
who did not wash hands between patients.

Huffman reviewed the test results in the folder and with a reassuring
smile said ``All the tests came back negative. One of these tests is
super sensitive about the presence of any type of cancer in the body;
even at the very early stages. So, we can safely say that your doctors
in India may have run a few extra tests but were correct in their
conclusion that the big C has not chosen your body as a host. Now let me
hear from you about this conjecture of relation between your life and
weight of the mail in your possession.''

Prem carefully opened one of the many folders he had brought along and
pulled out the index cards. Each one had the name of a relative or
friend on top of the page. Underneath was a carefully made Excel style
sheet made with a pen. It depicted in chronological order the date the
letter was written or received and a short note about the contents of
the letter followed by any special comments. Some index sheets ran
several pages. Each index sheet was linked to three ring binders
containing copies of the written and originals of the received letters.
Some letters were dated as far back as 1952. No one goes through such
laborious work unless they are convinced deep in their hearts that the
efforts are worth the trouble, will make a difference to someone's life,
may be an accurate source of history of the family, the city and the
country.

Carefully keeping the sheets and some folders on the examination table,
Prem gently caressed his treasure and then with a sad smile said ``You
see doctor, these letters have been a purpose of my life, something I
looked forward to doing every day. It made me look forward to living
fully. For various reasons, the amount of out-going and in-coming mail
is rapidly declining. Coincidently or by nature's design, my weight also
started to decline at about the same time. My belief is that nothing in
this universe happens by chance. Everything is governed by a plan,
unseen it may be but surely there is a designer creating the design.''
He took a short pause and then with sadness expressing through moist
eyes he continued ``Once the weight of the mail got almost stagnant at
40 Kg. and my weight declined from 70 Kg to about 49, the meaning was
obvious---end is near, only 9 Kg away.''

Huffman had diagnosed and treated thousands of patients but today he was
stumped. He did not know whether to whisk away this stupid notion or
were we all a pawn in the grand scheme of things and events coordinated
by the unseen hand with unseen powers. What if there was a real message
coming or was there a place for placebo effect playing a role here. Mind
can be programmed into sickness or health based on the data fed into the
mind. It is effective as long as the mind believes the data to be real.
Even if there are no external chemicals affecting the cells, the tissues
and the organs, the mere focused, firmly believed thought will generate
internal chemicals actually affecting cells and organs in a constructive
or destructive manner depending on the basic nature of the thought.

Researchers estimate that 80\% of all major illnesses like cancer, skin
disorders, cardiovascular disease and even backache are related to mind
and behavior. Stress is perceived to be a psychological problem but it
has very real physical effects. Increased secretion of adrenaline,
acceleration of heart beat, greater tension in the muscles, slower or
improper digestion are the results of such physiological changes. Blood
pressure and blood cholesterol levels may rise, there is thickening of
blood and making it more prone to clot formation. This in turn increases
the risk of heart attacks and stroke. He could sense that in this
patient the self destruction was in progress. Simply telling Prem ``What
nonsense, there is nothing wrong with you. Go home, stop wasting your
money and our time''---will not solve the problem.

He leaned over and put his hand gently over Prem's left shoulder in a
sympathetic and almost fatherly manner, even though he was half Prem's
age. The chair one sit in adds years to one's demeanor and authority.
Every word delivered has several times more weight compared to someone
else, without the aura of authority, saying the same thing.

`` I fully understand your dilemma, Prem. I am sure you and your family
are as happy as I am that you do not, I repeat, do not have any organic
disease, including cancer, causing your body to whither. Your friend was
obviously joking when he made comments about your declining mail. You
took it much too seriously and let it get to you. May be now you will
get burdened with the guilt of affecting many innocent people who were
the butt of your jokes and pranks! I am only kidding.'' They both had a
big laugh of relief and joy.

``Now you go back to your life and remember that a purposeful person
will always find a niche where nature is calling for help. You have so
much more to add to the lives of your family and friends.''

They shook hands and Dr. Huffman made Prem's day by saying that there
would be no charge for today's visit. Prem said `` Thank you. Even if
you had billed me I would not have paid any way. You know why? Because
one time a doctor gave a patient 6 months to live. The patient did not
pay his bill so the doctor gave him another 6 months!'' They both had a
hearty laugh as they went their separate ways, both feeling sad that
they might never meet again.

Jyoti was in the waiting room and could not wait to hear every word that
was uttered behind the closed doors. On receiving the good news, she let
out a loud `yipee' startling some lingering patients and their families
in the waiting room and the staff behind the glass window. Happily,
Jyoti and Prem left the office to head home and celebrate the good news
by going out for drinks, dinner and ice cream.

With new lease on life, the first thing Prem did next day was to buy a
new legal size note book, a new pen--made in China, and sat down to
write every detail of the story plus all that was happening in the lives
of Disha, Tanuj, Ashit, Jyoti as well as the Nation. After 21 pages and
with tired hands he went over to the fridge, pulled out a cold Heineken,
put his feet up and looked out of the window at the white fluffy clouds
gently rolling across light blue sky. Just as he had taken the last sip,
Jyoti walked in with a large plastic bin with the markings of U S Post
Office and placed it on the table.

`` You wouldn't believe, Papa. I got a call from the post office that
the amount of mail to be delivered is far too much to fit in our small
mailbox. They asked me to come and pick up the box full of letters from
all over the world.''

Unknown to Prem, Dr. Huffman, with the help of Shashi, Ashit and Jyoti
had taken the addresses of all of Prem's family members and friends. He
then dictated a letter stating the diagnosis of mind over body playing
tricks with Prem and the obvious treatment was to increase the weight of
the mail. This in turn will encourage Prem's mind to tell the body to
eat and exercise more, drink a little to keep up with the increasing
load. He ended by saying `` Rx: This is Doctor's order, please comply.''

Every member of the family, including grand children and circle of
friends not only wrote a letter making sure that they were not of the
kind which said ``I am well and hope you are in the same well. Rest all
is fine'' They were long, juicy, full of news of themselves, their work,
family, hobbies, good news and bits of sad news they wished to share.
More letters went out and more boxes were carried in periodically,
initially in USA and then for many years in Calcutta.

Years went by, folders got heavier and their numbers multiplied. Another
steel almirah was needed to store the treasure.

Bets were floating via Facebook and Twitter whether Prem, who was by now
at age 99 , will make century or not. Bets of all kinds were being
placed, various permutations and combinations brilliantly thought out by
Raju, Umang, Rohit and others. The betters' names were kept secret, only
the numbers for and against were known. At age 99 and 11 months a
century was being considered a safe bet. With all the bets in, the
fateful day of June 1, 2031 was being awaited with great anticipation.
Over 2 lacs of rupees had been placed in the kitty and everyone was
making plans how to spend it.

And then, at different times, in different time zones, in different
countries, the news traveled via telephones or email, Facebook, Whats
App and Twitter in the form of the following letter.

My Dear Family,

Obviously it is impossible, but wishful thinking has no limits or
boundaries. I wish Mataji and Pitaji were here to see this day with us.
The bets have been placed. 99\% are betting for the century up but the
deep desire to win the bet and to get out of the frail, hurting body are
telling me that as much as I would love to see the century completed, it
seems that the innings is over. I did have a cancer for a number of
years that was not easy to detect and fortunately was very slow growing.
It has finally permeated every part of my body.

I have no regrets now, had some as we were going through life but in
some strange ways it all evens out. The blessings were way more than the
shortfalls, balance sheet was overfilled with abundance. Looking back, I
would not change a thing\ldots{}may be I would not have worried as much.
Blissfully I was granted a positive attitude and outlook to life. I
would focus on only the good qualities of everyone I came in contact
with. My inner passion was to learn about others' life, and helping
wherever I could.

Love has been my guiding force in life. Money has some value but love
trumps it all. Excess money tends to cause more problems than it solves.

I never believed in it but now I do want to believe in reincarnation and
hopefully will see you all again, soon at the next FGT\ldots{}or when
you see a long letter, hear wind blowing, a thunder or lightning in the
sky, a sound of a bird, a cold beer, a loud laughter, a zindabad shouted
out somewhere, it may just be ME!!

Love and Bye,

Fondly,

Prem

This is how the family got the news of the outcome of the bet\ldots{}via
a letter with slightly shaky writing but still beautifully crafted
letters and still never a word crossed out or re-written, scanned and
sent as an attachment. On the night of May 30, 2031 two days shy of the
century, the giant, the glue of the family, Dear Prem said good bye to
the family. He passed away peacefully with a smile on his face, holding
one of Shashi's hand who held Amritvani and a Mala in the other. The
family will celebrate his life and hope that they can get at least some
of the number of qualities that came so naturally to him. Without having
a firm belief in God, he indeed was blessed. So was the Luthra Parivar
to have had such a love filled and love giving person, appropriately
named Prem, in their family.

Prem was the only one who had placed his bet against the century being
up and ended up winning the last bet of his life. The partial
benefactors of his Will turned out to be, in equal share, every member
of the Khandan of Vidya and Kundan Lal Luthra, right down to the latest
addition of Mataji Pitaji's great, great grand children. They all
divided up the eight lac rupees . They decided to have the largest FGT
of all times, rented the whole movie theater, spent two lacs on tickets
and a few bags of popcorn. Right in the middle of the movie, the lights
went out completely, a pre recorded short movie of lightening and
thunder lit the screen ending with sounds and sights of rain and then
the words mixed with flowers started descending with a picture of Prem
filling the screen in the background and the captions reading \ldots{}
Presented Fondly by Prem Luthra. A voice commanded everyone in the
audience to repeat five times---Prem Bhapa Zindaba, Prem Chacha
Zindabad, Prem Mama Zindabad, Prem Papa Zindabad, Babu Ji
Zindabad\ldots{}\ldots{}.Pictures of Mata Ji and Pita Ji gradually
descended on both sides of Prem's picture, rose petals gradually falling
and they said-Prem Beta-Zindabad,

Prem Beta---Zindabad!!!

His grand children, Disha and Tanuj had orchestrated this portion of the
movie.
