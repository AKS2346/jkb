\chapter{Dreams Don’t Die}
There was a flurry of activity all over the house. "Is the suitcase ready?
Did you pack enough mango pickle and praanthe with cooked dry potatoes
placed between them? Are your shoes polished? Do you have enough money for
the journey? Do you have Dr. Chitkara's (Pitaji's cousin) address, at
whose house you will be staying for the first 3 or 4 days?"

The last question was planned to avoid the notorious, scary ritual of
ragging which every new student received from the seniors.

"Make sure your shoes match; one black and one brown will look really
funny. Don't stick your head out of the moving train, you will get a coal
particle in your eye. Above all, no more mischief; you are a big boy now."

Questions and advice were coming from every corner and everybody.  Mataji
(as my siblings and I called our mother), with her nose dripping and tears
in her eyes, haltingly, in a raspy voice, said, "Make sure you eat
properly; I will not be there running after you with butter soaked
praanthe and your favourite bhindi and karele." She kissed my forehead.
Our tears mingled.

What was the fuss and occasion for all this hoopla in September 1961 at
House Number 2, Model Town, Panipat?

I was on my way out of the home, at the age of 17, alone, plunging into
the large, unknown world. I grew up with the luxury of being sheltered, so
far, by my mother, brothers, sisters, Dadaji and, mostly physically absent
but in spirit always around, my father (Pitaji). We all knew that, felt
that.

After the partition of India and our migration from what now had become
Pakistan, the responsibilities of supporting the large family was on
Pitaji's shoulders. Money earned from the newly learned trade of
brickmaking was never enough. First, the eldest son, Suraj, followed by
the next one, Prem, and later, Virinder, got jobs and supplemented
Pitaji’s meagre income. All of this combined money was barely enough to
cover the family's basic needs of food and shelter.

Education was considered to be as essential as food by Kundan Lal Luthra,
my father, whom we called Pitaji. That was the main goal of his life for
his children.

The family had distributed 11 boondi laddoos as gifts to all the
neighbours, and appropriate donations had been made to the needy woman,
Brahmni, who visited us weekly to receive money and food for her recently
uprooted family. Ganesh, the god who removes obstacles, had been invoked
with indecipherable mantras. We had also obtained proper blessings from
the family gurus - Swami Satya Nand Ji and Shakuntla Behan Ji.

Yes, indeed, it was a big occasion for the family, and above all for
Pitaji. I was the first child going to medical college and, hopefully,
becoming the first doctor in our family. My six older siblings had become
engineers, businessmen, and teachers. The one younger than me, the last
one of the eight siblings, had clearly declared his intentions of becoming
an engineer. I had also wanted to become an engineer, not a doctor
- because I hated blood! In hindsight, what a foolish reason that was; but
at that age, it was a major one that caused me to join pre-engineering
courses after high school. In the two-year course, I had already attended
pre-engineering classes for three months in 1959 at Government College,
Rupar, Punjab.

A two-week all-India trip, arranged through the college, gave Pitaji
a window of opportunity to change the course of my life. While I was on
the trip, he went to my college, cancelled my Physics and Math classes,
and enrolled me in Botany and Zoology. He also arranged for evening
tuition and extra classes, to make up for lost time.

He was pleasantly surprised and happy when I did not object, but quietly
followed the path he had carved out for me. After getting sufficient
grades, I was accepted to Medical College, Amritsar, only 20 miles from
Lahore. Since then, I have been forever thankful for his foresight.

We later learned that there was a whole different dimension to this
action: the fulfillment of a long, hibernating dream of the architect of
education of the whole family, our Pitaji.

Before Independence, Kundan Lal was one of the two sons of Lala Gokal
Chand. Lalaji had been gifted 10 acres of land by his British officer.
Kundan's brother, Karam Chand, was meant to take over the farming. Kundan
loved to help people. “What better way to serve than to become a doctor?”
was his thinking. The year was 1921. The handsome young man had worked
very hard to achieve his dreams. With this goal in mind, Kundan took
pre-medical courses at Daya Nand College, Lahore. He completed the
prerequisite courses and earned higher marks than required to get admitted
to the prestigious Glancy Medical College, Lahore.

Much to Kundan's disappointment, a Muslim boy with lesser marks was given
the one remaining spot due to the quota system in Muslim-dominated Lahore.
Not one to be defeated and give up, Kundan continued his studies to
complete his B.Sc. to ensure admission to medical college the following
year. His actions and courses were on track.

No one can foresee or predict the hand of destiny which is stronger than
one’s efforts. What is not meant to be will not be. Defeatism, a rational
defense mechanism of acceptance, or ingrained deep faith in destiny and
the will of God: such has been the Hindu philosophy.

At that time, infectious diseases were prevalent and a common cause of
death in all age groups. Kundan's brother, Karam, fell victim to one such
disease and died at the age of 21. Their mother died shortly thereafter.
Kundan Lal got K C engraved in green ink on his forearm in memory of Karan
Chand.

Lalaji summoned Kundan back from Lahore to Khanewal, our ancestral
village, to look after the land. Kundan’s pleas to let him continue his
education fell on deaf ears. One could cry, make a scene, but the dictates
of the elders could not be denied. Such was the culture in the 1920s. The
budding doctor, overnight, became a doctor of crops. And what a farmer he
became! Farmers from all around Khanewal sought his advice on all aspects
of farming.

The country got divided, the family got uprooted, displaced, and became
homeless refugees. Pitaji taught us that all you have today can get
snatched away in a moment. Don't give up; hard work with a smile can build
a full life.

Pitaji's suppressed, dormant, hibernating seed of becoming a doctor had
never died. One day it had to sprout. He must have seen something in me
akin to himself and saw a doctor in me, serving humanity. The one to carry
the torch and to keep the man's dream alive. His surviving 7th child was
about to enter medical college and become a doctor. The dream lived on.

Now I was finally ready to embark on the daunting, lonely journey to
Amritsar on a newly started train called the Flying Express. After the
flurry of activity at home, goodbyes, tight hugs and tears, Pitaji
escorted me to the train station.

The train, as usual, was running late. He did not talk much, but looked at
me with his usual style of tilted head with a slanted gaze, a tear
building up in the corner of his eye. He carefully wiped it with the same
motion he would use to straighten his hair, hoping the train would be
delayed a little longer.

A loud whistle, a down green signal, and a plume of smoke far to the left
broke his trance. Overhead speakers announced the arrival of the train
that was rushing in to take his son away to finally fulfill his long
cherished dream. The held back tears could not be subdued any longer;
a mixture of joy and sorrow wet my cheeks and collar as he gave me a tight
reassuring hug.

The train rolled out gently. His blessings to serve humanity, and become
as good a doctor as he had dreamt to be, kept following me through his
loving gaze, the gradually fading smile on his face and the waving of his
hands.
