\chapter{Vendors and Hawkers}
Silence is the first things that hits me when I return from India. The
orchestra of sounds created  by animals, birds and human vocal cords,
along with a variety of man-made machines and amplifiers suddenly
vanishes, making me wonder if I am losing my sense of hearing or facing
deafening silence. 

This is particularly true of the era long gone, the years of Panipat in
the 1950s. The years when the flower of life, as I know, started blooming
and absorbed the panorama of the world around me. Times change but the
memories of those days, somewhat faded, still linger. 

Among the constant background sounds, the one missed the most is the
periodic shrill yelps of numerous vendors and hawkers selling products in
the trains, buses, movie theaters, on the streets and roads. 

Who can forget the shrill high pitched sound of 'Chai garam' at the train
stations? Just as the train begins to slowly pull in to the station,
numerous young boys start the chorus, trying to outdo one another, to seek
attention of potential customers. The bolder ones hop on to the steps of
the moving train, precariously holding onto the metal handle bar while
balancing their aluminum kettles and reddish brown clay mugs. They briskly
walk up and down the aisles to get a first shot at the thirsty, tea loving
customers. Various tones and pitches saying Chai Garam fill the air inside
the train and along the windows partially blocked by horizontal bars with
enough space to exchange a steaming hot mug with money. 

Some vendors carry the tea in aluminum kettles and some in open buckets.
Flies follow the sweet smell of pre-mixed sugar. 

The tea sellers are soon followed by sellers of moongfalli (peanuts) still
warm in their shells. They are served in folded bags made out of old
newspapers. It is a common scene to have the empty open pods scattered all
over the train floor. Some snacks like fried yellow moon daal, roasted
chick peas are served in conical containers created from old newspapers.
Even the thought of spicy fried lentils in the cones with freshly squeezed
lemon juice induces Pavlov's reflex of saliva trickling in the mouth. 

Moongfalis were gradually getting replaced by pre-packed nuts, spicy
snacks and biscuits. 

"Santre kele le lo!" (get oranges and bananas) is hawked by another set of
sellers. Fruits are stacked on round jute trays deftly balanced on one
hand as the other hand exchanges the goods for money. Seasons dictate the
types of fruits; cut slices of watermelons are popular in hot summers. 

All of them manage to squeeze through the narrow aisles rubbing shoulders
with  the constantly moving passengers and fellow vendors. Somehow they
deliver the merchandise, collect money and almost always jump off the
moving train, maximizing their time in the compartments. 

Alongside the train are also boys walking briskly and hawking garam samose
and pakode. They make a great combination with garam chai. They are
dispensed in the double layered pieces of newspapers. Imli chutney adds
the tangy taste and cools down the piping hot samose and pakode. 

"Boot Polish kara lo sahib" is another sound that fills the railway
platform as one waits for the arrival of the train. Young boys with array
of Kiwi black and brown boot polish in round metal containers, brushes,
cloth pieces and a metal ankle size pedestal to park the foot on. Needed
or not, I get the shoes polished, partly to get rid of the pervasive dust
that settles on the shoes and partly thinking that at least the boy is
trying to work to earn money and not begging. 

When our rickshaw or tonga with luggage approaches the railway station or
when the train arrives at our destinations, we get surrounded by the
sounds of "coolie sahib" from men wearing red shirts with a brass badge on
the sleeve and a rolled piece of cloth on their heads to support some of
the luggage as the rest hangs over their strong shoulders. No one tells
you how much they will charge; they attempt to grab the luggage to beat
the competitor. They do put the luggage properly and then the haggling
starts. Coolies are helpful in knowing exactly where the third class
compartment will stop. When a train arrives at the station, coolies hop
onto the moving trains and vie loudly for grabbing your luggage to take it
down with a standard statement "give whatever you please.” There is also
quibbling among coolies “I touched this suit case first. This is my
customer.” Sometimes both do the tag game at the same time. It is best to
stay out and let let them settle. After a yelling match and friendly
Panjabi abuse they split. 

After the delievery of luggage into the train or out of it, the rigorous
process of haggling starts. Finally one succeeds, ending the fake bitter
argument over the fair price for the services.  If you know the system it
is quite an enjoyable experience but to a novice it may be annoying. 

Now a days this old tradition has gradually vanished from large urban
train stations but still prevalent at smaller stations. It is more orderly
now but I still miss the old drama. 

Train stations are not the only places where hawkers thrive. Four wheeler
carts, filled with fresh vegetables, some pushed by hand and some pulled
by a bicycle, start making appearance on the streets at sunrise. The
sellers move slowly, periodically howling "Aloo pyaaz sabji le lo.” Women
and sometimes men come out to avail these services, thus avoiding trips to
the market. The vegetables are weighed in the hand held traazoo (scale)
with the produce placed on one side and different weights in the other.
The customer is always wishing to see the produce side tipping more while
the seller trying to keep it even and sometimes tipping the scale down on
the side of the produce with an invisible play of the hand holding the
scale, giving false pleasure to the customers. Back and forth accusation
and defense about this hand trick a always a way of benign bickering.
These scales are now getting replaced by the stationary weighing scales
which still need different sized weights on the other side to weigh the
goods without the risk of tipping the scale with hand. More recently the
automated scales with the moving needle or even digital scales are making
appearance. 

Just about the same time one starts hearing "Ande double roti makhkhan le
lo", the last one getting a prolonged stretch. Some of the carts have
bells making their presence known to the potential customers. Home
delivery of fresh eggs, bread and butter is much appreciated. 

In hushed voice the vendor says, “Painji tusi sudiyan Mehta sahib di kurhi
musalman naal bhaj gayee aye. (Sister, have you heard that Mr. Mehta’s
daughter ran away with a Muslim.”) The customer has already heard it but
pretends not to know. “Then what happened”, she says in an attempt to
squeeze more juice out of the gossip. These vendors not only sell their
products but also become a venue for social news and gossip exchange among
the customers as well as the sellers who keep track of what is going on in
the Mohalla  (local area). After a fair share of gossip, bargaining and
purchase, the hawkers move on to their next stop. 

Doodh vallaaaa sound accompanies a man on bicycle with two drums hanging
on each side of the carrier over the back wheel. Gentle arguments over
water having mixed with the milk are commonly heard. Later the bicycles
gave into motor cycles, then delivery in glass or plastic containers. Now
they have also mostly vanished, thanks to super markets. 

Medical system is rudimentary, especially in rural areas. Some vendors in
the buses and occasionally trains hawk powder in a bottle claiming it will
cure stomach pain, hemorrhoids, heart diseases and even marital problems.
Surma with an applicator can cure kala motia (glaucoma), safed motia
(cataract) and trachoma. Powder in small packets can cure all dental
problems. Even as children we know they are fake but many people fall for
these shysters. 

As the day matures one starts hearing Raddi, botlen!! (Paper trash and
bottles). Recycling has been used decades before this word came to Western
vocabulary. These are buyers of used newspapers, magazines, books, bottles
and metal objects. They come on bicycles with two bags hanging on each
side of the carrier-one for paper and other for glass/metal. People save
these items in piles during the week. On hearing Raddi botlen, they rush
out with the recyclables, get them weighed and collect money in exchange.
Most vendors give money but barter system is also common, exchanging
valuable trash for vegetables, fruit or toys. 

'Bhaande Kalayi Kara Lao' (get your utensils polished) is sound of the
announcement of a team of two or three men coming to apply a shine on the
inside of cooking utensils. They come prepared with soft charcoal,
a manual machine to blow air to keep the burnt coal red hot, naushadar
powder, sticks of tin, a pan to hold cold water and woolen as well as
cotton cloth. The brass or copper utensils are held by a metal tong, made
burning hot, naushadar powder is applied and rubbed with thick folded
cloth. It creates white smoke. The tin sticks are sporadically applied to
the inner surface. Melted tin spots are spread out quickly with folded
cloth to give a uniform luster to the inside wall of the utensils. The
utensils are quickly dipped in a pan of water. I can still see the steam
rising along with the sizzling sound of red hot utensil cooling down in
cold water. Utensils look new for about six months till the cooking
gradually melts the tin, which the hungry growing bodies gorge down with
the delicious food made by Mata Ji. No side effects occurred unless you
count waking up at 3 AM and writing all this on iPhone as one of them.
Seventy years later. 

Loud speakers attached to rickshaws are used by politicians to beg votes
just before the elections. Their names and party symbols are prominently
displayed. Then they go into hibernation and show up before the next one. 

Movie theaters promote the currently playing movie with large posters on
three sides of rickshaws with loud speakers blaring songs from that movie. 

In the evenings, after the children return from schools, one starts
hearing calls from ice cream sellers. These are particularly popular in
the summer. The vendors entice children out by their calls of "Ice Cream
le lo" as well as loud bells announcing their arrival. A variety of ice
creams are stored in the cooler pushed as a cart or pulled by a bicycle.
The children beg their parents for money and merrily run out to lick their
favorite flavor and cool down the sweltering heat. 

Even though everyone in Model Town is a refugee, I don’t remember seeing
a beggar. People take pride in doing any job, scrape and save money and
start the life from scratch. 

Hari Ram is barely able-to -see-shadows blind man with pocks marked face.
He lives with his family near our snatched house in the city. They own
buffaloes and sell milk and butter. Hari comes to our home for the milk
and butter delivery. He chats with us while MataJi graciously serves him
hot meal. Few years later Virinder was traveling in a train. To his
surprise Hari Ram comes in with his harmonium and starts singing, and
begging money. Virinder says “Namaste Hari Ram. This is Virinder.” Hari
felt ashamed of being seen begging by someone known to him. Without
a word, Hari runs and disappears into the crowd. We never saw him again.
Even without eyes he could not look at us, due to shame of begging. 

Children get a thrill when a family brings monkeys which perform tricks in
exchange for some money. They always include a scene of a male monkey
marrying a saari clad female monkey. 

Young boys and girls with rubber like flexible bodies dance and do
acrobatics in anticipation of some money in return. Most people oblige but
some sheepishly walk away before the show ends lest they get too
embarrassed to pay the performers. 

Loud shrill sound of Moongfali  just a few seconds before the interval in
a movie theater seems to be an integral part of movie. Sounds of Chai
garam fills the air here also.  Chana chor garam, the spicy roasted chick
peas, are also our favorite snacks. 

The humdrum of sounds of vendors and hawkers fill the air in the markets,
festivals, rickshaw stands, railway stations, and the streets of Panipat. 

A mere 17 hours of flight from India transforms me back to an orderly but
eerie silent life. Adjusting to the jet lag takes a few days but the
adjustment to the missing orchestra of sounds takes me several weeks or
even months after I return from India to the USA. In fact, the sounds of
home sit on the back burner but never vanish completely. 
