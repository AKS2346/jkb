\chapter{Customs, Traditions and Superstitions}

Indians may leave India but India never leaves Indians, no matter which
part of the earth they migrate to. 

There is hardly any country where Indians have not made a home outside the
native homeland called Hindustan, India, or Bharat. They carry their
genetics, ingrained thinking, expressions, customs, traditions, and
superstitions with them. While practicing these at home, by and large they
merge with the adopted society and its norms. 

In the fluid global international world, the firm boundary lines are
gradually fading. Indians transmit their native culture and traditions to
the adopted one and add their unique color to the tapestry. And they
imbibe the language, customs and traditions of the adopted countries. 

Many customs, traditions and superstitions while living in the
post-partition Panipat in the 1950s come alive in the mind even after 70
years. 

Crowing of the rooster wakes us up but we cover the heads for a little
more snooze. Virinder, our responsible elder brother and mentor Kanchan,
elder sister pull the sheets down. Now there is no choice. 

We bow down, touch feet of the elders as they keep their hands on our
heads and bless us with wishes of long life (jug jug jiyo). The hand which
touched the feet is brought to touch our hearts as if transmitting their
blessings to our heart. It is the first thing newly married couples do to
their priest, parents and elders. The feet touching is done to all elders
any time relatives, or friends of Mata Ji or Pita Ji visit. But it is
a must when Shakuntla Ma periodically stops after finishing her Satsang. 

Even though there are several places of worship including temple, Gurdwara
and mosque nearby, most homes have a small place of worship in their
homes. Children are busy with schools, little bit of homework and then,
just being kids, busy with outdoor games. Elders read scriptures, sing
devotional songs and meditate. Every so often, especially on auspicious
occasions, children are asked to join the religious rituals in our temple
at the house, located behind the swing and attached to the store room. 

We wash hands and feet, take off the shoes and sit, then copy what elders
do. 

For more elaborate and special functions, a priest (Pandit Ji) is invited
to conduct the religious ceremony. Books of devotional mantras and
devotional songs are handed out to everyone sitting on cushioned floor.
A harmonium and dholki (two sided drum) player are invited. The
congregation follows the lead singer or the Pandit Ji conducting the
pooja. At the end of the ceremonies, a bit of food is put on the mouth of
moorti of the Hanuman and then distributed to all as parsaad. Some people
place a thali comprised of all the foods we are going to consume in front
of the picture of the chosen, revered God. After serving food to God,
everyone else gets their plates. In later years, reciting Amrit Vani,
written by Swami Satya Nand Ji is the norm. Kanta leads the melodious
singing accompanied by her playing the harmonium. Everyone else sings
along, some by memory and some by reading the booklet. 

I learnt from elders that in Khanewal, we had separate utensils for
Muslims and Jamadars ( garbage, laterite cleaners). Hindus did not go to
Muslim homes. We have no Muslim visitors in Panipat but rules continue for
jamadars. They or their shadow should not touch the kitchen area. If it
does, the kitchen floor is washed with water. 

Menstruating women are also not permitted to enter the kitchen. They stay
in their room and food is delivered to them. That was a custom around but
we never heard it in our house. 

Following the trauma of the Partition, Pita Ji’s main goal is to make sure
all the children get best possible education. He keeps a distant but close
watch on our grades. Only two options are given to boys—Engineer or
doctor. Girls are educated to become teachers. Stern love is the driving
force. Fortunately all but one followed the dictates. Only Prem started
job after Bachelor of Arts, to financially support the family. 

Exercises are not done in a formal way. Walks to and from school and
unending outdoor games and cycling burn more calories than we consume. 

A crow cawing indicates that a guest is expected. Many times the
prediction comes out to be true. That is partly because it is common to
have unannounced guests come in. Some for a day and some for stay. Both
are always welcome. 

Formal Yoga is done whenever our Mama Ji, Chooni Lal Virmani visits. He is
son of Mata Ji’s paternal uncle. Being the only child, Mata Ji ties Rakhi
to Mama Ji, her first cousin. He has glowing fair skin, is jovial and
a devout practitioner of yoga. He calls us his little Chelas (students).
He teaches us Surya namaskar, sitting on our folded legs, separate the
legs and lie backwards till our flexible body and back of head touch the
ground. That’s all forgotten as soon as his fiat car is out of sight. But
even at age 79, I can lie down between the folded legs! And remember our
loving Mama Ji. 

Being on time is considered a strange phenomenon to Indians. It has become
ingrained in our system and we still run on Indian standard time. This may
be 15 minutes or even 2 hours late. And there is no apology or
explanation. The hosts also accept it as a norm because they do the same
when roles are reversed. When abroad, Indians do follow the normal time
when invited by caucasians. But when going to Indian home or function, we
default to IST. 

We eat home-made, almost always vegetarian food. Even after 50 years of
leaving India, we start missing Indian food after just one or two dishes
of other countries. 

We are asked to eat yogurt before heading for any school exam. “It will
improve your memory,” MataJi says. Right or wrong, no one questions it. 

The tradition of arranged marriage is the way to find suitable match. It
is said that marriage is not only Union of the couple but also the two
families. 

Friends or relatives start suggesting a suitable match for the marriage
age child. For girls it starts when she turns around 16 and boys around
age 20. Common sources are barbers, priests, relatives and friends. This
is arranged by the parents and children simply get informed. Our oldest
brother, Suraj was engaged to Urmil for seven years and they saw each
other at time of wedding. Virinder and Santosh also had seen each other by
looking at photographs taken in the studios till they met at the time of
wedding. Things did change. Dolly and I met on the stage doing a drama.
Little did we know that we would have to go through a family drama to
finally get together. 

The traditions are changing. Arranged marriages are still done, but boy
and girl get to meet in company of families and lately in privacy. Western
culture has entered in a big way and now most marriages are arranged by
the boy and girl, with ceremonial blessings of the parents. 

Dowry system is a standard part of marriages around us. As the society
became more educated, some times girls started earning as much or more
than boys, the tradition is gradually dying down. In our family dowry
system is is not followed. Prem returns even a fruit basket sent by Kalras
after he gets engaged to Shashi Kalra. 

The boy’s parents can stay with the couple, but the girls parents do not.
Our Nani pays rent to stay in our house. Stretching to extreme, girls
parents do not even drink water at the boy’s home. This happened as late
as 1960s. In our house, Shashi Bhabi’s parents never drink water, let
alone tea or food. This tradition has been shed, at least in urban areas. 

Barring few Anglicized Indians, regular clothing in Panipat is shirt/pants
for men; salwar kameez or saari for women. At Indian functions while
living abroad, this has not changed much. Men wear neck ties on formal
occasions. Kurta pajama is the traditional night wear. 

We grew up listening to Indian film music and that is ingrained on our
hard copies of the brain. A very special music starts the day on the All
India Radio, Vividh Bharati. Just listening to the first few notes,
Indians who were old enough to remember in 1950s, can recognize the tune. 

A musical program, Binaca Geet Mala, conducted by Amin Sayani from Ceylon
(now Sri Lanka) comes on the radio from 8 pm to 9 pm every Wednesday.
Family drops everything and gathers around the radio. His segments,
recorded on CDs, are still played all over the world. Lately the CDs have
vanished giving way to streaming and You Tube. Binaca Geet Mala lives on.
It is sponsored by Binaca tooth paste which obviously is very famous and
used by many. 

Even after living abroad for fifty years, Hindi music sounds melodious and
touches the heart. Dolly finds old Hindi songs as therapy or meditative. 

Among many happy traditions, some sad ones are prevalent but fortunately
are disappearing. A widower, after only a few months was allowed, rather
encouraged to get remarried. Sometimes to the sister of deceased wife,
relieving the parents of struggling to find a suitable match and to save
providing dowry. 

On the other hand, life of a widow is a living hell. They can no longer
wear Jewelery or colored clothes. By hitting her arms on a solid base, she
breaks the glass bracelets. Seeing a young woman in white sari is a sign
of her being a widow. She is scorned and shunned by the society, many
times including the family. It is considered a bad omen to invite a widow
to a wedding, lest her presence may invoke a curse and may somehow bring
early death to the boy about to get married. A common word often heard in
Hindi movies “You have already eaten one man, now you want to eat another
one.” This was the thought implied and sometime verbalized. 

We experience this first hand when my brother died at the age of forty
five leaving behind a forty years old widow. With three young children.
Widows do not choose marriage sometimes to protect their children from an
unknown man who may or may not treat her children well. She sacrifices her
life for children. As much this tradition is changing, it is still more
difficult for women than men even in the modernized society. 

According to the memory bank of our home, Kanchan tells that I was quite
a naughty child. I am told that I used to make small round balls of the
expensive and laboriously made wheat dough and throw them standing on the
railing-less roof top, throw stones and break clay pots. Virinder drops me
to the school on the bicycle. I run back and many times I am home befor
Virinder comes back. Pita Ji is mostly away to work. Mata Ji has numerous
jobs to do. She is at her wits end how to deal with me. 

She thinks of the Bum Bum Bolay baba. The last resort was to pay an
exorcist. This Baba is a man who goes around on the streets, stomps a long
bamboo staff on the road. It has many ghungroos tied to it. They make loud
sounds to attract potential customers. He is supposed to have powers of
exorcism. He wears only loin cloth; his body, including face, is painted
orange and blue. His lips are bright red. He is a devotee of Lord Shiv. He
has long hair, holds a wooden staff with attached bells in his right hand
and a broom in the left. One day Mata Ji brought the Baba in and asked him
to do something for the naughty son. I am brought out to our side bricked
yard. Bum bum bolay stomps the staff on the ground, goes around me and
screams words to the effect asking the ghost to leave my body. I find it
amusing. Apparently it worked. He advises Mata Ji not to wash clothes on
Thursday, Shiv Ji’s day of the week. Since that day till her death in 1990
she never washed clothes on Thursdays. Such was the influence of belief in
Bum Bum Bollay. 

Taking care of the elders is a norm. Pita Ji’s father lives in the house.
He wears a tight white turban with a long tail hanging behind. He walks
carrying a cane, not to help him walk, but to warn Mata Ji of his
approach. He also gives a gentle cough in case the sound of cane is not
heard. Mata Ji quickly covers her face with the part of saari covering her
head. This is called Ghund in Panjabi. 

This tradition stops as Pita Ji does not believe in this practice. May be
his progressive mind or the educated daughters-in-law have a role in
reversing an age old tradition of Panjab in our house. 

One peculiar way of head movement of saying yes by nodding forward and
also by moving it side to side seemed normal to us. Till we came abroad
and learnt that side to side means no. Even now I have to consciously be
aware of this habit. 

Growing up we have many superstitions. A black cat crossing the path we
walk on makes us stop and say Ram Ram seven times. Even after education,
and moving abroad this persists. Someone else sneezes when a person is
leaving for an important job is ominous. As is drinking water as we are
ready to walk out of the house. Number 3, for some reason is considered
unlucky number.

Many of these customs, traditions and superstitions have changed, some
linger. 

