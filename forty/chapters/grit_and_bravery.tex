\chapter{Grit and Bravery}
Growing up in the newly independent India, I did not realize or know that
my mother and father were very brave. Not only brave, they anticipated the
dangers ahead, acted upon them and in the process lost every material
possession except the most precious commodity they had produced,
protected, cared for and loved deeply. Together. That was the seven
children with the eighth and last one to arrive three months after the
move. 

The event was a massive, partly anticipated, partition of India resulting
in the free India and newly created Pakistan. India was built on a secular
system while Pakistan was based on choice of religion, a land for Muslims.
Our parents, Hindus, happened to be in the wrong section of India, and had
a decision to make: Lose all the material possessions by leaving their
home or risk their own and their family's lives by staying and hoping that
the turmoil would settle and division of the country would not happen. 

They made the right call. In May 1947 our parents, Vidya Wati and Kundan
Lal Luthra, moved across the potential boundary line yet to be drawn by
a British lawyer. This resulted in the greatest migration of humanity in
recent history. About 14 million people crossed the borders and 1.5
million were killed. 

Our family was safe, thanks to the foresight, wisdom and bravery of our
parents. They had purchased a vacation home in Sabathu near Simla in March
1947. 

They never mentioned the turmoils and tribulations they went through and
provided us the most love filled, hatred free childhood. There was not
a word of the enormous losses they had incurred, nor the fact that they
had no money, no job, no source of support or hope. Their only guiding
lamp post was to do whatever it takes to survive and do so with a smile.
They knew that the children are very perceptive and pick up on the
negative or positive vibes. They chose the latter. 

Early on, they figured out that shelter, love and education were the keys
out of misery. Without a paisa in hand, they bought a house with borrowed
money, gave unconditional love beyond imagination and made sure that all
eight of us got education to stand on our own feet. Four became engineers,
one a businessman, two teachers and one doctor. Not getting an education
was not an option for the children. 

While growing up, we never heard them complain about politicians, British
government, the circumstances or even the Muslims for asking for a new
country which usurped our home and possessions. There were no regrets of
the lost past, only grit to move on and survive to the best of their
ability. 

Over the last 68 years since the partition I received education and went
on to become a physician, an Ophthalmologist, got married, blessed with
a daughter, and emigrated to United Kingdom, blessed with two daughters
and finally reached the United States of America. Life has offered its ups
and downs. In the hardest of the times, I look back at what my parents
went through and realize that my problems, no matter how big, are
minuscule compared to ones tackled by Mata Ji and Pita Ji. That gives me
courage to stand up and face the challenges as life presents to me and my
family. My parents taught me that adversities will come, sometimes
predicted and at times unannounced. The solution is to be brave, assess
the situation and find the best possible way. Let not the past spoil the
present and the future. Fill the days with grit, love, bravery and
forgiveness. 
